\documentclass{beamer} 
\usetheme{AnnArbor}
\usepackage[euler-digits]{eulervm}
\usefonttheme{professionalfonts}
\usepackage{xeCJK}
\usepackage{pstricks-add}
\setCJKmainfont[BoldFont=FZBeiWeiKaiShu-S19S]{FZQiTi-S14S}
\newtheorem{lemm}{引理}
\newtheorem{theo}{定理}
\newtheorem{rema}{注}
\newtheorem{coro}{推论}
\newtheorem{defi}{定义}
\newtheorem{exer}{题目}
\newtheorem{exam}{例}
\newtheorem{question}{问题}
\renewcommand{\today}{\number\year 年 \number\month 月 \number\day 日}
\begin{document}
\title{Fourier变换及其相关理论} \subtitle{开题报告}
\author{叶卢庆}\institute[杭州师范大学]{杭州师范大学理学院数学112班\\
  \medskip 
指导老师:谢剑}
\begin{frame}
  \titlepage
\end{frame}
\begin{frame}
  \frametitle{选题背景与意义}
  \begin{itemize}
  \item Fourier分析博大精深.横跨代数,微分方程,分析,数论等多个数学领
    域.在力学,光学等物理领域有深刻应用.在图像处理,信号处理等工程领域也
    有广泛应用.\pause
  \item 我的学年论文也是写有关Fourier变换的理论,希望毕业论文能在学年论
    文的基础上进行深化和扩展.
  \end{itemize}
\end{frame}
\begin{frame}
  \frametitle{研究的基本内容和拟解决的问题}
  \begin{itemize}
\item Fourier分析早期历史.包括Euler,d'Alembert,Lagrange,Fourier等人的贡献.特别要参考
  Fourier的著作《热的解析理论》.\pause
\item 探究线性代数的相关理论和Fourier分析理论的紧密联系.\pause
\item Fourier分析与微分方程.以及离散Fourier分析与差分方程.\pause
\item Fourier分析与等周问题(Hurwitz).\pause
\item 探究为什么通过离散Fourier变换能导出四次以及四次以下方程的根式解(Lagrange).同样
  的变换对于五次方程为什么会失效.\pause
\item Fourier分析在弦振动,$n$个自由度系统的微振动,热传导等物理问题中的
  应用.\pause
\item 以上内容并不是孤立地叙述,而是互相联系,形成一个有机的整体.
  \end{itemize}
\end{frame}
\begin{frame}
  \frametitle{研究方法及措施}
\begin{itemize}
\item  看书,上网查阅相关资料和文献.记录笔记和感悟.最后整理成文.
\end{itemize}
\end{frame}
\begin{frame}
  \frametitle{研究工作的步骤与进度}
  \begin{itemize}
  \item 在必须上交论文的倒数第二个星期撰写好论文,并交由导师审阅.在此之前,一直看书,写笔
    记.
  \end{itemize}
\end{frame}
\begin{frame}
  \frametitle{主要参考文献}
  \begin{itemize}
\item [1] Elias M. Stein,Rami Shakarchi.Fourier Analysis.Princeton University Press,2003.
  \item [2] 陶哲轩. 王昆扬译.陶哲轩实分析[M].第1版.北京:人民邮电出版
    社,2008.
\item [3]M.Klein.张理京,张锦炎等译.古今数学思想.第1版.上海:上海科学技
  术出版社,2009.
\item [4]吴大猷.古典动力学[M].第1版.北京:科学出版社,1983.
\item [5]Joseph Fourier.桂质亮译.热的解析理论.第1版.北京:北京大学出版
  社,2008.
  \end{itemize}
等等.参考文献现在还没定型.得我论文写完后才知道具体参考了什么.
\end{frame}
\begin{frame}
  \frametitle{END}
\begin{center}
谢谢大家.
\end{center}
\end{frame}
\end{document}