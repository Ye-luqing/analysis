\documentclass[a4paper, 12pt]{article} % Font size (can be 10pt, 11pt or 12pt) and paper size (remove a4paper for US letter paper)
\usepackage{amsmath,amsfonts,bm}
\usepackage{hyperref,verbatim}
\usepackage{amsthm,epigraph} 
\usepackage{amssymb}
\usepackage{framed,mdframed}
\usepackage{graphicx,color} 
\usepackage{mathrsfs,xcolor} 
\usepackage[all]{xy}
\usepackage{fancybox} 
% \usepackage{xeCJK}
\usepackage{CJKutf8}
\newtheorem*{adtheorem}{定理}
% \setCJKmainfont[BoldFont=FZYaoTi,ItalicFont=FZYaoTi]{FZYaoTi}
\definecolor{shadecolor}{rgb}{1.0,0.9,0.9} %背景色为浅红色
\newenvironment{theorem}
{\bigskip\begin{mdframed}[backgroundcolor=gray!40,rightline=false,leftline=false,topline=false,bottomline=false]\begin{adtheorem}}
    {\end{adtheorem}\end{mdframed}\bigskip}
\newtheorem*{bdtheorem}{定义}
\newenvironment{definition}
{\bigskip\begin{mdframed}[backgroundcolor=gray!40,rightline=false,leftline=false,topline=false,bottomline=false]\begin{bdtheorem}}
    {\end{bdtheorem}\end{mdframed}\bigskip}
\newtheorem*{cdtheorem}{习题}
\newenvironment{exercise}
{\bigskip\begin{mdframed}[backgroundcolor=gray!40,rightline=false,leftline=false,topline=false,bottomline=false]\begin{cdtheorem}}
    {\end{cdtheorem}\end{mdframed}\bigskip}
\newtheorem*{ddtheorem}{注}
\newenvironment{remark}
{\bigskip\begin{mdframed}[backgroundcolor=gray!40,rightline=false,leftline=false,topline=false,bottomline=false]\begin{ddtheorem}}
    {\end{ddtheorem}\end{mdframed}\bigskip}
\newtheorem*{edtheorem}{引理}
\newenvironment{lemma}
{\bigskip\begin{mdframed}[backgroundcolor=gray!40,rightline=false,leftline=false,topline=false,bottomline=false]\begin{edtheorem}}
    {\end{edtheorem}\end{mdframed}\bigskip}
\newtheorem*{pdtheorem}{例}
\newenvironment{example}
{\bigskip\begin{mdframed}[backgroundcolor=gray!40,rightline=false,leftline=false,topline=false,bottomline=false]\begin{pdtheorem}}
    {\end{pdtheorem}\end{mdframed}\bigskip}

\usepackage[protrusion=true,expansion=true]{microtype} % Better typography
\usepackage{wrapfig} % Allows in-line images
\usepackage{mathpazo} % Use the Palatino font
\usepackage[T1]{fontenc} % Required for accented characters
\linespread{1.05} % Change line spacing here, Palatino benefits from a slight increase by default

\makeatletter
\renewcommand\@biblabel[1]{\textbf{#1.}} % Change the square brackets for each bibliography item from '[1]' to '1.'
\renewcommand{\@listI}{\itemsep=0pt} % Reduce the space between items in the itemize and enumerate environments and the bibliography

\renewcommand{\maketitle}{ % Customize the title - do not edit title
  % and author name here, see the TITLE block
  % below
  \renewcommand\refname{参考文献}
  \newcommand{\D}{\displaystyle}\newcommand{\ri}{\Rightarrow}
  \newcommand{\ds}{\displaystyle} \renewcommand{\ni}{\noindent}
  \newcommand{\pa}{\partial} \newcommand{\Om}{\Omega}
  \newcommand{\om}{\omega} \newcommand{\sik}{\sum_{i=1}^k}
  \newcommand{\vov}{\Vert\omega\Vert} \newcommand{\Umy}{U_{\mu_i,y^i}}
  \newcommand{\lamns}{\lambda_n^{^{\scriptstyle\sigma}}}
  \newcommand{\chiomn}{\chi_{_{\Omega_n}}}
  \newcommand{\ullim}{\underline{\lim}} \newcommand{\bsy}{\boldsymbol}
  \newcommand{\mvb}{\mathversion{bold}} \newcommand{\la}{\lambda}
  \newcommand{\La}{\Lambda} \newcommand{\va}{\varepsilon}
  \newcommand{\be}{\beta} \newcommand{\al}{\alpha}
  \newcommand{\dis}{\displaystyle} \newcommand{\R}{{\mathbb R}}
  \newcommand{\N}{{\mathbb N}} \newcommand{\cF}{{\mathcal F}}
  \newcommand{\gB}{{\mathfrak B}} \newcommand{\eps}{\epsilon}
  \begin{flushright} % Right align
    {\LARGE\@title} % Increase the font size of the title
    
    \vspace{50pt} % Some vertical space between the title and author name
    
    {\large\@author} % Author name
    \\\@date % Date
    
    \vspace{40pt} % Some vertical space between the author block and abstract
  \end{flushright}
}

% ----------------------------------------------------------------------------------------
%	TITLE
% ----------------------------------------------------------------------------------------
\begin{document}
\begin{CJK}{UTF8}{gkai}
  \title{\textbf{例20.2.3}}
  % \setlength\epigraphwidth{0.7\linewidth}
  \author{\small{叶卢庆}\\{\small{杭州师范大学理学院,学
        号:1002011005}}\\{\small{Email:h5411167@gmail.com}}} % Institution
  \renewcommand{\today}{\number\year. \number\month. \number\day}
  \date{\today} % Date
  
  % ----------------------------------------------------------------------------------------
  
  
  \maketitle % Print the title section
  
  % ----------------------------------------------------------------------------------------
  %	ABSTRACT AND KEYWORDS
  % ----------------------------------------------------------------------------------------
  
  % \renewcommand{\abstractname}{摘要} % Uncomment to change the name of the abstract to something else
  
  % \begin{abstract}
  
  % \end{abstract}
  
  % \hspace*{3,6mm}\textit{关键词:} % Keywords
  
  % \vspace{30pt} % Some vertical space between the abstract and first section
  
  % ----------------------------------------------------------------------------------------
  %	ESSAY BODY
  % ----------------------------------------------------------------------------------------
  \begin{example}[20.2.3]
计算三重积分
$$
I=\iiint_V(x^2+y^2+z^2)dxdydz,
$$
$V$ 是椭球面
$$
\frac{x^2}{a^2}+\frac{y^2}{b^2}+\frac{z^2}{c^2}=1
$$
的内部区域.
\end{example}
\begin{proof}[解]
不妨设 $a,b,c>0$.还有注意,为书写简便
期间,以下积分中,任意常数 $C$ 都不写.我们来求
$$
\int_{-c}^c\int_{-a \sqrt{1-\frac{z^2}{c^2}}}^{a
  \sqrt{1-\frac{z^2}{c^2}}}\int_{-b
  \sqrt{1-\frac{z^2}{c^2}-\frac{x^2}{a^2}}}^{b
  \sqrt{1-\frac{z^2}{c^2}-\frac{x^2}{a^2}}}(x^2+y^2+z^2)dxdydz.
$$
把这个求出来,我们的目的就达到了.令 $x=au$,$y=bv$,$z=cw$,则 Jacobi 行列
式为
$$
\begin{vmatrix}
  a&0&0\\
0&b&0\\
0&0&c\\
\end{vmatrix}=abc.
$$
我们只用求
$$
abc\int_{-1}^1\int_{-\sqrt{1-w^2}}^{\sqrt{1-w^2}}\int_{-\sqrt{1-w^2-v^2}}^{\sqrt{1+w^2+v^2}}(a^2u^2+b^2v^2+c^2w^2)dudvdw.
$$
进行球坐标变换,令$u=\rho\sin\phi\cos\theta$,$v=\rho\sin\phi\sin\theta$,$w=\rho\cos\phi$.则得到 Jacobi 行列式
$$
\begin{vmatrix}
 \frac{\pa u}{\pa\rho}&\frac{\pa u}{\pa \phi}&\frac{\pa u}{\pa \theta}
 \\
 \frac{\pa v}{\pa\rho}&\frac{\pa v}{\pa \phi}&\frac{\pa v}{\pa \theta}
 \\
 \frac{\pa w}{\pa\rho}&\frac{\pa w}{\pa \phi}&\frac{\pa w}{\pa \theta} \\
\end{vmatrix}=\rho^2\sin\phi.
$$
因此化为
\begin{align*}
abc\int_0^{2\pi}\int_0^{\pi}\int_0^1(a^2\rho^2\sin^2\phi\cos^2\theta+b^2\rho^2\sin^2\phi\sin^2\theta+c^2\rho^2\cos^2\phi)\rho^2\sin\phi
d\rho
d\phi d\theta.
\end{align*}

我们知道,
\begin{align*}
&  \int_0^1(a^2\rho^2\sin^2\phi\cos^2\theta+b^2\rho^2\sin^2\phi\sin^2\theta+c^2\rho^2\cos^2\phi)\rho^2\sin\phi
  d\rho\\&=\frac{1}{5}(a^2\sin^2\phi\cos^2\theta+b^2\sin^2\phi\sin^2\theta+c^2\cos^2\phi)\sin\phi
\end{align*}
我们知道,
$$
\sin^3\phi=\frac{1}{4} (3 \sin \phi-\sin 3\phi)
$$
$$
\cos^2\phi\sin\phi=\frac{1}{4} (\sin \phi+\sin 3\phi).
$$
因此,
\begin{align*}
&\frac{1}{5}(a^2\sin^2\phi\cos^2\theta+b^2\sin^2\phi\sin^2\theta+c^2\cos^2\phi)\sin\phi\\&=\frac{1}{20}a^2\cos^2\theta(3\sin
\phi-\sin 3\phi)+\frac{1}{20}b^2\sin^2\theta(3\sin \phi-\sin
3\phi)+\frac{1}{20}c^2(\sin \phi+\sin 3\phi).
\end{align*}
因此
\begin{align*}
&  \int \frac{1}{20}a^2\cos^2\theta(3\sin
\phi-\sin 3\phi)+\frac{1}{20}b^2\sin^2\theta(3\sin \phi-\sin
3\phi)+\frac{1}{20}c^2(\sin
\phi+\sin 3\phi)d\phi\\&=
\frac{1}{20}a^2\cos^2\theta
(-3\cos\phi+\frac{1}{3}\cos 3\phi)+\frac{1}{20}b^2\sin^2\theta(-3\cos\phi+\frac{1}{3}\cos3\phi)\\&+\frac{1}{20}(-\cos\phi-\frac{1}{3}\cos3\phi)c^{2}.
\end{align*}
因此
\begin{align*}
&  \int_0^{\pi}\frac{1}{20}a^2\cos^2\theta(3\sin
\phi-\sin 3\phi)+\frac{1}{20}b^2\sin^2\theta(3\sin \phi-\sin
3\phi)+\frac{1}{20}c^2(\sin
\phi+\sin 3\phi)d\phi\\&=\frac{4}{15}a^2\cos^2\theta+\frac{4}{15}b^2\sin^2\theta+\frac{2}{15}c^{2}.
\end{align*}
我们知道,
\begin{align*}
  \int \frac{4}{15}a^2\cos^2\theta+\frac{4}{15}b^2\sin^2\theta+\frac{2}{15}c^2d\theta=\frac{2}{15}a^2(\theta+\sin\theta\cos\theta)+\frac{2}{15}b^2(\theta-\sin\theta\cos\theta)+\frac{2}{15}c^2\theta.
\end{align*}
因此
$$
\int_0^{2\pi}\frac{4}{15}a^2\cos^2\theta+\frac{4}{15}b^2\sin^2\theta+\frac{2}{15}c^2d\theta=\frac{4\pi}{15}a^2+\frac{4\pi}{15}b^2+\frac{4\pi}{15}c^2.
$$
因此最后的答案为
$$
abc(\frac{4\pi}{15}a^2+\frac{4\pi}{15}b^2+\frac{4\pi}{15}c^2).
$$

\end{proof}
% ----------------------------------------------------------------------------------------
% BIBLIOGRAPHY
% ----------------------------------------------------------------------------------------
  
\bibliographystyle{unsrt}
  
\bibliography{sample}
  
% ----------------------------------------------------------------------------------------
\end{CJK}
\end{document}