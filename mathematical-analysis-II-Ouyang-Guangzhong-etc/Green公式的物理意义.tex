\documentclass[twoside,11pt]{article} 
\usepackage{amsmath,amsfonts,bm}
\usepackage{hyperref}
\usepackage{amsthm} 
\usepackage{pgf,tikz}
\usetikzlibrary{arrows}
\usepackage{pstricks-add}
\pagestyle{empty}
\usepackage{amssymb}
\usepackage{framed,mdframed}
\usepackage{graphicx,color} 
\usepackage{mathrsfs,xcolor} 
\usepackage[all]{xy}
\usepackage{fancybox} 
% \usepackage{CJKutf8}
\usepackage{xeCJK}\usepackage[]{caption2}\renewcommand{\captionlabeldelim}{}
\newtheorem{theorem}{定理}
\newtheorem{lemma}{引理}
\newtheorem{corollary}{推论}
\newtheorem{remark}{注}
\setCJKmainfont[BoldFont=Adobe Heiti Std R]{Adobe Song Std L}
% \usepackage{latexdef}
\def\ZZ{\mathbb{Z}} \topmargin -0.40in \oddsidemargin 0.08in
\evensidemargin 0.08in \marginparwidth 0.00in \marginparsep 0.00in
\textwidth 16cm \textheight 24cm \newcommand{\D}{\displaystyle}
\newcommand{\ds}{\displaystyle} \renewcommand{\ni}{\noindent}
\newcommand{\pa}{\partial} \newcommand{\Om}{\Omega}
\newcommand{\om}{\omega} \newcommand{\sik}{\sum_{i=1}^k}
\newcommand{\vov}{\Vert\omega\Vert} \newcommand{\Umy}{U_{\mu_i,y^i}}
\newcommand{\lamns}{\lambda_n^{^{\scriptstyle\sigma}}}
\newcommand{\chiomn}{\chi_{_{\Omega_n}}}
\newcommand{\ullim}{\underline{\lim}} \newcommand{\bsy}{\boldsymbol}
\newcommand{\mvb}{\mathversion{bold}} \newcommand{\la}{\lambda}
\newcommand{\La}{\Lambda} \newcommand{\va}{\varepsilon}
\newcommand{\be}{\beta} \newcommand{\al}{\alpha}
\newcommand{\dis}{\displaystyle} \newcommand{\R}{{\mathbb R}}
\newcommand{\N}{{\mathbb N}} \newcommand{\cF}{{\mathcal F}}
\renewcommand\figurename{图}
\newcommand{\gB}{{\mathfrak B}} \newcommand{\eps}{\epsilon}
\renewcommand\refname{参考文献} \def \qed {\hfill \vrule height6pt
  width 6pt depth 0pt} \topmargin -0.40in \oddsidemargin 0.08in
\evensidemargin 0.08in \marginparwidth0.00in \marginparsep 0.00in
\textwidth 15.5cm \textheight 24cm \pagestyle{myheadings}
\markboth{\rm \centerline{}} {\rm \centerline{}}
\begin{document}
\title{\huge{\textbf{Green公式的证明及其物理意义}}} \author{\small{叶卢
    庆\footnote{叶卢庆(1992---),男,杭州师范大学理学院数学与应用数学专业
      本科在读,E-mail:h5411167@gmail.com}}\\{\small{杭州师范大学理学院,浙
      江~杭州~310036}}} \date{}
\maketitle

% ----------------------------------------------------------------------------------------
% ABSTRACT AND KEYWORDS
% ----------------------------------------------------------------------------------------


\textbf{\small{摘要}:}给出了平面Green公式的物理解释. \smallskip

\textbf{\small{关键词}:}Green公式,功\smallskip


\vspace{30pt} % Some vertical space between the abstract and first section

% ----------------------------------------------------------------------------------------
% ESSAY BODY
% ----------------------------------------------------------------------------------------
平面Green 公式叙述如下:
\begin{theorem}[平面Green公式\cite{ouyangguangzhong}]
  设 $D$ 是以光滑曲线 $l$ 为边界的平面单连通区
  域,函数 $P(x,y)$ 和$Q(x,y)$ 在 $D$ 以及 $l$ 上连续并具有
  对 $x$ 和 $y$ 的连续偏导数,则有
$$
\iint_D \left(\frac{\pa Q}{\pa x}-\frac{\pa P}{\pa
    y}\right)dxdy=\oint_l Pdx+Qdy.
$$
这里右断为沿有向闭路的积分,积分路径的方向是和趋于正向联系的,即当一个人
沿着曲线 $l$ 行走时区域 $D$ 恒在其左边(亦即服从右手法则的联系).
\end{theorem}
\begin{remark}
  文献 \cite{ouyangguangzhong} 对于单连通区域的定义很粗糙,我们现在来叙
  述\href{http://en.wikipedia.org/wiki/Simply_connected_space}{维基百
    科}上的定义:一个拓扑空间 $X$ 称为是单连通的,如果它
  是道路连通的,而且任意连续映射 $f:S^1\to X$ 都能在如下意义上被压缩为一
  点:存在连续映射 $F:D^2\to X$,当 $F$ 限制在 $S^1$ 上时就
  是 $f$.其中 $S^1$ 是二维欧氏空间中的单位圆,$D^2$ 是二维欧氏空间中的单
  位圆盘.
\end{remark}
\begin{remark}
  初看之下,文献 \cite{ouyangguangzhong} 对于此定理的叙述似乎是漏了一个
  条件.我们还得指出 $l$ 是闭曲线.其实不然,$D$ 为单连通区域已经蕴含
  了 $l$是闭曲线.否则,如果 $D$ 是单连通的,$l$ 不是闭曲线,则可以构造一
  条 $D$ 内的闭曲线包围曲线 $l$,显然这条闭曲线不能连续收缩为一
  点,这与$D$ 是单连通区域矛盾.
\end{remark}

\bigskip\bigskip在证明之前,我们先引进一些概念,证明一些引理.  光滑曲线的
定义可以见文献\cite{kodaira}.对于光滑曲线 $S$,根据光滑曲线的定
义,设 $S$ 的参数方程为
$$
(s_1(t),s_2(t)),
$$
其中 $t\in [m,n]$,$m<n$ 为实数. $s_1(t),s_2(t)$ 都为连续可微函
数,且在 $[m,n]$ 上恒有 $s_1'(t)^2+s_2'(t)^2>0$.因此,根据隐函数定理,对于
曲线 $S$ 上的任意一点 $(s_1(t'),s_2(t'))$ 来说,都存在 $\va(t')>0$,使得
在$(t'-\va(t'),t'+\va(t'))$ 之内,$s_1(t')$ 和 $s_2(t')$ 有连续可微的函
数关系.\\

下面我们来看图 \ref{fig:1}.如图,是一条光滑曲线 $S$.除了在点 $P$ 附近外,该
光滑曲线在其余的点附近,$s_2(t)$ 都能表示为 $s_1(t)$ 的连续可微的函数.而
在点 $P$附近,$s_2(t)$ 不能表示为 $s_1(t)$ 的连续可微的函
数,但是 $s_1(t)$ 能表示为 $s_2(t)$ 的连续可微的函
数 $g(s_2(t))=s_1(t)$.易得$g'(s_2(t_{P}))=0$,其中$(s_1(t_p),s_2(t_p))$
是点 $P$ 的坐标.像 $P$这种点,我们将其叫做垂直点,把曲线在垂直点处的切线
叫做垂直线,图中 $A$ 就
是一条垂直线.将垂直线上的垂直点对应的横坐标叫做垂直线对应的点.\\
\begin{figure}\centering
  \scalebox{1} % Change this value to rescale the drawing.
  {
    \begin{pspicture}(0,-3.39)(6.4559374,3.39)
      \psbezier[linewidth=0.04](0.0,-0.13)(0.0,-0.93)(4.3839707,-1.7152581)(5.18,-1.11)(5.9760294,-0.50474185)(6.082291,1.447214)(5.46,2.23)(4.837709,3.012786)(3.502296,2.984781)(2.56,2.65)
      \psline[linewidth=0.04cm](5.88,3.37)(5.92,-3.37)
      \usefont{T1}{ptm}{m}{n} \rput(6.2865624,-3.225){A}
      \psdots[dotsize=0.12](5.88,0.85) \usefont{T1}{ptm}{m}{n}
      \rput(6.170781,0.915){P}
    \end{pspicture}
  }\caption{}\label{fig:1}
\end{figure}

我们再来看图 \ref{fig:2}.也是一条光滑曲线 $S$.在其它点附近,$s_1(t)$总能
表达成$s_2(t)$ 的连续可微的函数 $s_1(t)=f(s_2(t))$.但是在
点 $n$ 和 $m$附近,$s_1(t)$ 不能表达为 $s_2(t)$ 的函数,但是 $s_2(t)$能表
达成 $s_1(t)$的函
数 $s_2(t)=g(s_1(t))$,且 $g'(s_1(t_n))=0,g'(s_1(t_m))=0$.其中 $n$的坐标
为 $(s_1(t_n),s_2(t_n))$,$m$ 的坐标是 $(s_1(t_m),s_2(t_m))$.曲线
在 $n$和 $m$ 点的切线称为水平线.水平线上的点的纵坐标成为水平线所对应
的点.\\
\begin{figure}\centering
  \scalebox{1} % Change this value to rescale the drawing.
  {
    \begin{pspicture}(0,-4.383411)(16.6,4.403411)
      \psbezier[linewidth=0.04](0.0,0.6649734)(0.0,-0.13502665)(6.822719,-2.9058676)(7.78,-3.1950266)(8.737281,-3.4841857)(12.416475,-4.363411)(13.3,-3.8950267)(14.183525,-3.4266424)(16.313738,0.40962902)(15.96,1.3449733)(15.606261,2.2803178)(10.257148,4.0804515)(9.26,4.0049734)(8.262853,3.929495)(6.295142,2.079246)(5.36,1.7249733)
      \psline[linewidth=0.04cm](2.02,-4.0550265)(16.58,-4.0950265)
      \psline[linewidth=0.04cm](5.58,4.0049734)(15.4,4.0049734)
      \usefont{T1}{ptm}{m}{n} \rput(1.9332813,-4.110027){M}
      \usefont{T1}{ptm}{m}{n} \rput(4.8098435,4.2299733){N}
      \psdots[dotsize=0.12](9.38,3.9649734)
      \psdots[dotsize=0.12](12.52,-4.0550265) \usefont{T1}{ptm}{m}{n}
      \rput(9.468125,3.8299735){n} \usefont{T1}{ptm}{m}{n}
      \rput(12.597656,-4.150027){m}
    \end{pspicture}
  }\caption{}\label{fig:2}
\end{figure}

下面我们来证明一个结论:
\begin{lemma}
  一条光滑的曲线 $W$,其参数化的表达为 $(w_1(t),w_2(t))$,其中 $t\in
  [m,n]$,则在区间 $[m,n]$ 上只有有限个点是垂直线对应的点,也只有有限个点
  是水平线对应的点.
\end{lemma}
\begin{proof}[\bf{证明}]
  假若在区间 $[m,n]$ 上,有无限个垂直线对应的点,则根据聚点原
  理,存在$k\in [m,n]$,对于 $k$ 的任意邻域来说,该邻域内都有无限个垂直线
  对应的点,由于 $W$ 是光滑的,导致这是不可能的(为什么?).对于水平线的证明
  也类似.
\end{proof}

\begin{proof}[\bf{Green定理的证明}]
  我们结合物理意义进行论述.见图 \ref{fig:3},设曲线 $l$ 的参数方程
  为$(l_1(t),l_2(t))$,其中 $a\leq t\leq b$,$l_1,l_2$ 是 $[a,b]$ 上的连
  续可微函数且 $\forall t\in [a,b]$,$l_1^2(t)'+l_2^2(t)'>0$.且区
  域 $D$的最左边的点和最右边的点的坐标为
  $(l_1(t_l),l_2(t_l))$,$(l_1(t_r),l_2(t_r))$.\\

  质点开始时位于点 $A$ 处,然后在力场的作用下开始沿着曲线 $l$ 做逆时针运
  动.力 $\mathbf{F}(x,y)$ 关于位置的函数为
$$
\mathbf{F}(x,y)=(P(x,y),Q(x,y)).
$$
质点在力场的作用下沿着光滑闭曲线 $l$ 运动一周后又回到 $A$ 点.显然在这个
过程里,力对质点所做的功为
\begin{equation}
  \label{eq:1}
  \oint_lP(x,y)dx+Q(x,y)dy=\oint_lP(x,y)dx+\int_lQ(x,y)dy.
\end{equation}
等式 \eqref{eq:1} 的物理意义实际上是把质点在各点的运动都分解为两个方向
的运动,一个是沿着横坐标方向,一个是沿着纵坐标方向.同时把作用在质点上的力
也分解为两个方向的力,一个是沿着横坐标方向,一个是沿着纵坐标方向.于是等
式 \eqref{eq:1} 说明,质点绕一圈回到原地,力对质点所做的功,等于沿着横坐标
方向的力对质点的横向运动所做的功,加上沿着纵坐标方向的力对质点的纵向运动
所做的功.\\

\definecolor{xdxdff}{rgb}{0.49,0.49,1}
\definecolor{uququq}{rgb}{0.25,0.25,0.25}
\begin{figure}
  \begin{tikzpicture}[line cap=round,line join=round,>=triangle
    45,x=1.0cm,y=1.0cm]
    \draw[->,color=black] (-4.77,0) -- (25.29,0); \foreach \x in
    {-4,-2,2,4,6,8,10,12,14,16,18,20,22,24}
    \draw[shift={(\x,0)},color=black] (0pt,2pt) -- (0pt,-2pt)
    node[below] {\footnotesize $\x$}; \draw[->,color=black] (0,-6.66)
    -- (0,6.73); \foreach \y in {-6,-4,-2,2,4,6}
    \draw[shift={(0,\y)},color=black] (2pt,0pt) -- (-2pt,0pt)
    node[left] {\footnotesize $\y$}; \draw[color=black] (0pt,-10pt)
    node[right] {\footnotesize $0$}; \clip(-4.77,-6.66) rectangle
    (25.29,6.73); \draw [rotate around={-174.76:(6.96,3.52)},line
    width=2.8pt] (6.96,3.52) ellipse (5cm and 2.46cm); \draw [line
    width=2.8pt] (2,-6.66) -- (2,6.73); \draw [line
    width=2.8pt,domain=-4.77:25.29] plot(\x,{(--1-0*\x)/1}); \draw
    [line width=2.8pt] (12,-6.66) -- (12,6.73); \draw [line
    width=2.8pt,domain=-4.77:25.29] plot(\x,{(--6-0*\x)/1});
    \begin{scriptsize}
      \fill [color=uququq] (12,0) circle (1.5pt); \draw[color=uququq]
      (12.17,0.29) node {$b$}; \fill [color=uququq] (0,1) circle
      (1.5pt); \draw[color=uququq] (0.16,1.28) node {$c$}; \fill
      [color=uququq] (0,6) circle (1.5pt); \draw[color=uququq]
      (0.18,6.29) node {$d$}; \fill [color=uququq] (2,0) circle
      (1.5pt); \draw[color=uququq] (2.19,0.29) node {$a$}; \fill
      [color=xdxdff] (2.94,1.78) circle (1.5pt); \draw[color=xdxdff]
      (3.11,2.07) node {$A$};
    \end{scriptsize}
  \end{tikzpicture}\caption{}\label{fig:3}
\end{figure}

然而曲线 $l$ 的情形比较复杂.用图像来表达,曲线 $l$ 最简单的情形是图
\ref{fig:4} 和图 \ref{fig:5}.\\
\begin{figure}\centering
  \scalebox{1} % Change this value to rescale the drawing.
  {
    \begin{pspicture}(0,-1.54)(11.26,1.54)
      \psellipse[linewidth=0.04,dimen=outer](5.63,0.0)(5.63,1.54)
    \end{pspicture}
  }\caption{}\label{fig:4}
\end{figure}
\begin{figure}\centering
  \scalebox{1} % Change this value to rescale the drawing.
  {
    \begin{pspicture}(0,-3.2264042)(11.84,3.226404)
      \psbezier[linewidth=0.04](0.0,0.686404)(0.0,-0.11359599)(1.0838287,-2.620788)(2.04,-2.913596)(2.9961712,-3.206404)(4.163497,-2.0309374)(4.98,-1.453596)(5.796503,-0.8762546)(5.353545,-0.49643216)(6.3,-0.173596)(7.2464547,0.14924017)(8.47926,-1.4262803)(8.6,-0.433596)
      \psline[linewidth=0.04cm](8.6,-0.433596)(8.6,1.126404)
      \psarc[linewidth=0.04](9.41,1.056404){0.83}{0.97102195}{180.0}
      \psbezier[linewidth=0.04](10.24,1.066404)(10.24,0.266404)(11.82,1.046404)(11.82,1.846404)
      \pscustom[linewidth=0.04] { \newpath \moveto(11.82,1.846404)
        \lineto(11.82,1.926404)
        \curveto(11.82,1.966404)(11.815,2.041404)(11.81,2.076404)
        \curveto(11.805,2.111404)(11.76,2.1714041)(11.72,2.196404)
        \curveto(11.68,2.221404)(11.625,2.271404)(11.61,2.2964041)
        \curveto(11.595,2.321404)(11.56,2.381404)(11.54,2.416404)
        \curveto(11.52,2.451404)(11.47,2.506404)(11.44,2.526404)
        \curveto(11.41,2.5464041)(11.34,2.576404)(11.3,2.586404)
        \curveto(11.26,2.596404)(11.18,2.626404)(11.14,2.646404)
        \curveto(11.1,2.666404)(11.02,2.701404)(10.98,2.716404)
        \curveto(10.94,2.731404)(10.86,2.761404)(10.82,2.776404)
        \curveto(10.78,2.791404)(10.7,2.816404)(10.66,2.826404)
        \curveto(10.62,2.836404)(10.54,2.856404)(10.5,2.866404)
        \curveto(10.46,2.876404)(10.38,2.891404)(10.34,2.896404)
        \curveto(10.3,2.901404)(10.22,2.916404)(10.18,2.926404)
        \curveto(10.14,2.936404)(10.06,2.951404)(10.02,2.956404)
        \curveto(9.98,2.961404)(9.9,2.976404)(9.86,2.986404)
        \curveto(9.82,2.996404)(9.74,3.011404)(9.7,3.016404)
        \curveto(9.66,3.021404)(9.58,3.036404)(9.54,3.0464041)
        \curveto(9.5,3.056404)(9.42,3.071404)(9.38,3.076404)
        \curveto(9.34,3.081404)(9.265,3.086404)(9.23,3.086404)
        \curveto(9.195,3.086404)(9.125,3.091404)(9.09,3.096404)
        \curveto(9.055,3.101404)(8.98,3.106404)(8.94,3.106404)
        \curveto(8.9,3.106404)(8.815,3.106404)(8.77,3.106404)
        \curveto(8.725,3.106404)(8.64,3.106404)(8.6,3.106404)
        \curveto(8.56,3.106404)(8.485,3.106404)(8.45,3.106404)
        \curveto(8.415,3.106404)(8.34,3.106404)(8.3,3.106404)
        \curveto(8.26,3.106404)(8.18,3.106404)(8.14,3.106404)
        \curveto(8.1,3.106404)(8.015,3.111404)(7.97,3.116404)
        \curveto(7.925,3.121404)(7.84,3.131404)(7.8,3.136404)
        \curveto(7.76,3.141404)(7.68,3.146404)(7.64,3.146404)
        \curveto(7.6,3.146404)(7.52,3.151404)(7.48,3.156404)
        \curveto(7.44,3.161404)(7.36,3.1714041)(7.32,3.176404)
        \curveto(7.28,3.181404)(7.2,3.186404)(7.16,3.186404)
        \curveto(7.12,3.186404)(7.045,3.186404)(7.01,3.186404)
        \curveto(6.975,3.186404)(6.9,3.186404)(6.86,3.186404)
        \curveto(6.82,3.186404)(6.74,3.186404)(6.7,3.186404)
        \curveto(6.66,3.186404)(6.58,3.186404)(6.54,3.186404)
        \curveto(6.5,3.186404)(6.415,3.186404)(6.37,3.186404)
        \curveto(6.325,3.186404)(6.24,3.186404)(6.2,3.186404)
        \curveto(6.16,3.186404)(6.08,3.186404)(6.04,3.186404)
        \curveto(6.0,3.186404)(5.92,3.191404)(5.88,3.196404)
        \curveto(5.84,3.201404)(5.755,3.206404)(5.71,3.206404)
        \curveto(5.665,3.206404)(5.595,3.201404)(5.57,3.196404)
        \curveto(5.545,3.191404)(5.475,3.186404)(5.43,3.186404)
        \curveto(5.385,3.186404)(5.305,3.186404)(5.27,3.186404)
        \curveto(5.235,3.186404)(5.16,3.186404)(5.12,3.186404)
        \curveto(5.08,3.186404)(5.0,3.186404)(4.96,3.186404)
        \curveto(4.92,3.186404)(4.84,3.186404)(4.8,3.186404)
        \curveto(4.76,3.186404)(4.68,3.186404)(4.64,3.186404)
        \curveto(4.6,3.186404)(4.52,3.186404)(4.48,3.186404)
        \curveto(4.44,3.186404)(4.36,3.186404)(4.32,3.186404)
        \curveto(4.28,3.186404)(4.2,3.186404)(4.16,3.186404)
        \curveto(4.12,3.186404)(4.035,3.186404)(3.99,3.186404)
        \curveto(3.945,3.186404)(3.86,3.186404)(3.82,3.186404)
        \curveto(3.78,3.186404)(3.7,3.186404)(3.66,3.186404)
        \curveto(3.62,3.186404)(3.53,3.186404)(3.48,3.186404)
        \curveto(3.43,3.186404)(3.34,3.186404)(3.3,3.186404)
        \curveto(3.26,3.186404)(3.175,3.181404)(3.13,3.176404)
        \curveto(3.085,3.1714041)(3.0,3.156404)(2.96,3.146404)
        \curveto(2.92,3.136404)(2.835,3.111404)(2.79,3.096404)
        \curveto(2.745,3.081404)(2.66,3.056404)(2.62,3.0464041)
        \curveto(2.58,3.036404)(2.505,3.011404)(2.47,2.996404)
        \curveto(2.435,2.981404)(2.36,2.946404)(2.32,2.926404)
        \curveto(2.28,2.906404)(2.195,2.871404)(2.15,2.856404)
        \curveto(2.105,2.841404)(2.025,2.816404)(1.99,2.806404)
        \curveto(1.955,2.7964041)(1.88,2.771404)(1.84,2.756404)
        \curveto(1.8,2.741404)(1.72,2.706404)(1.68,2.686404)
        \curveto(1.64,2.666404)(1.56,2.626404)(1.52,2.606404)
        \curveto(1.48,2.586404)(1.4,2.556404)(1.36,2.5464041)
        \curveto(1.32,2.536404)(1.245,2.506404)(1.21,2.486404)
        \curveto(1.175,2.466404)(1.135,2.411404)(1.13,2.376404)
        \curveto(1.125,2.341404)(1.12,2.271404)(1.12,2.236404)
        \curveto(1.12,2.201404)(1.12,2.136404)(1.12,2.106404)
        \curveto(1.12,2.076404)(1.12,2.011404)(1.12,1.976404)
        \curveto(1.12,1.941404)(1.12,1.871404)(1.12,1.836404)
        \curveto(1.12,1.801404)(1.12,1.7314041)(1.12,1.696404)
        \curveto(1.12,1.661404)(1.12,1.591404)(1.12,1.556404)
        \curveto(1.12,1.521404)(1.12,1.451404)(1.12,1.416404)
        \curveto(1.12,1.381404)(1.12,1.311404)(1.12,1.276404)
        \curveto(1.12,1.241404)(1.125,1.176404)(1.13,1.146404)
        \curveto(1.135,1.116404)(1.14,1.046404)(1.14,1.006404)
        \curveto(1.14,0.966404)(1.135,0.89140403)(1.13,0.856404)
        \curveto(1.125,0.821404)(1.12,0.756404)(1.12,0.726404)
        \curveto(1.12,0.696404)(1.115,0.631404)(1.11,0.596404)
        \curveto(1.105,0.561404)(1.065,0.506404)(1.03,0.486404)
        \curveto(0.995,0.46640402)(0.92,0.441404)(0.88,0.43640402)
        \curveto(0.84,0.431404)(0.775,0.431404)(0.75,0.43640402)
        \curveto(0.725,0.441404)(0.695,0.476404)(0.69,0.506404)
        \curveto(0.685,0.536404)(0.675,0.601404)(0.67,0.636404)
        \curveto(0.665,0.671404)(0.66,0.741404)(0.66,0.776404)
        \curveto(0.66,0.811404)(0.645,0.881404)(0.63,0.916404)
        \curveto(0.615,0.95140404)(0.585,1.021404)(0.57,1.056404)
        \curveto(0.555,1.091404)(0.52,1.156404)(0.5,1.186404)
        \curveto(0.48,1.216404)(0.43,1.276404)(0.4,1.306404)
        \curveto(0.37,1.336404)(0.295,1.366404)(0.25,1.366404)
        \curveto(0.205,1.366404)(0.14,1.3564041)(0.12,1.346404)
        \curveto(0.1,1.336404)(0.08,1.291404)(0.08,1.256404)
        \curveto(0.08,1.221404)(0.075,1.151404)(0.07,1.116404)
        \curveto(0.065,1.081404)(0.05,1.016404)(0.04,0.986404)
        \curveto(0.03,0.95640403)(0.02,0.89140403)(0.02,0.856404)
        \curveto(0.02,0.821404)(0.02,0.751404)(0.02,0.716404)
        \curveto(0.02,0.681404)(0.02,0.621404)(0.02,0.546404) }
    \end{pspicture}
  }\caption{}\label{fig:5}
\end{figure}
\\
这种情形也就是,$l$ 是不自交的曲线,且对于区
间 $[a,b]$ 来说,点 $l_1(t_l)$ 和 $l_1(t_r)$ 把该区间分成三段,不妨设分为
$$
[a,t_l],[t_l,t_r],[t_r,b]
$$
或者是
$$
[a,t_r],[t_r,t_l],[t_l,b].
$$
在每一段上,$l_1(t)$ 都是单调函数(未必严格单调).\\


在这种情形下,我们来看沿着横坐标方向的力对质点的横向运动所做的功
$$
\oint_lP(x,y)dx.
$$
不妨让点 $A$ 是区域 $D$ 的最右一点 $(l_1(t_l),l_2(t_l)$.质点从 $A$ 出
发,然后达到区域 $D$ 的最右端,此时,质点的横坐标是
从$l_1(t_l)$变到 $l_1(t_r)$.在质点从 $A$ 达到区域 $D$最右端的过程里,经
过了曲线 $l$的一部分.我们将区间 $[l_1(t_l),l_1(t_r)]$ 分割
成 $n$ 等分,形成 $n$ 个小区间
\begin{equation}\label{eq:1} [l_1(t_l),l_1(t_l)+\Delta
  x],[l_1(t_l)+\Delta x,l_1(t_l)+2\Delta
  x],\cdots,[l_1(t_l)+(n-1)\Delta x,l_1(t_l)+n\Delta x].
\end{equation}
其中 $\Delta x=\frac{l_1(t_r)-l_1(t_{l})}{n}$.当 $n$ 越来越大时,每个区
间都越来越小,对应于有限条垂直线的点将被有限个越来越小的区间所覆盖,因此
可将有限条垂直线对应的点忽略.因此,我们不妨认为
在 $[l_1(t_l),l_1(t_r)]$上,质点在曲线$ l$ 上从$A$ 到最右端按逆时针方向
走过的部分是 曲线 $l_{lower}$,不妨看作一个逐段连续函数 $f$ 的图像,其中
间断点只有有限个.然后我们在区间 $[l_1(t_l)+(i-1)\Delta x,l_1(t_l)+i
\Delta x]$ 上任取一点 $P(x_{i},f(x_{i}))$,其中 $x_i\in
[l_1(t_l)+(i-1)\Delta x,l_1(t_l)+i\Delta x]$,且 $(x_{i},f(x_{i}))$ 位于
曲线 $l_{lower}$ 上.然后得到和式
$$
\sum_{i=1}^nP(x_{i},f(x_{i}))\Delta x.
$$
当 $n\to \infty$ 时,上述和式变成积分
$$
\int_{l_1(t_l)}^{l_1(t_r)}P(x,f(x))dx
$$
由于 $P(x,y)$ 的连续性,结合 $l$ 在区间 $[l_1(t_l),l_1(t_r)]$ 只有有限个
间断点,因此易得上述积分是存在的.质点在到达最右端后,开始返回,当然,这里返
回的意思是在横坐标意义上的返回.为什么质点在达到横向最右端后会返回呢?这
是当然的,因为达到最右了,不能再往右了,只能返回了.质点达到最右后,开始沿着
不同的路径返回,直到回到 $A$ 为止.质点沿着不同的路径返回,经过的曲线
是 $l_{upper}$.易得 $l_{upper}$ 和$l_{lower}$ 只相交于区域 $D$ 最左边的
一点和最右边的一点,且 $l_{upper}$和 $l_{lower}$ 的并形成曲线 $l$.对于曲
线 $l_{upper}$ 来说,我们也可以将
其看作逐段连续函数 $g$ 的图像,其中 $g$ 只有有限个间断点.\\

质点返回的过程就是质点在曲线 $l_{upper}$ 上运动,且质点运动的横坐标
从 $l_1(t_r)$ 变回到 $l_1(t_l)$ 的过程.在质点回来的过程里,我们考虑区
间 $[l_1(t_A),l_1(t_r)]$ 的同一个分割\eqref{eq:1}.我们会得到和式
$$
\sum_{i=1}^nP(x_{i},g(x_{i}))\Delta x,(x_{i},g(x_{i}))\in l_{upper}.
$$
令 $n\to\infty$,上述和式变成积分
$$
\int_{l_1(t_r)}^{l_1(t_l)}P(x,g(x))dx,(x,g(x))\in l_{upper},
$$
因此,可得
\begin{align*}
  \oint_lP(x,y)dx&=\int_{l_1(t_l)}^{l_1(t_r)}P(x,f(x))dx+\int_{l_1(t_r)}^{l_1(t_l)}P(x,g(x))dx\\&=\int_{l_1(t_l)}^{l_1(t_r)}P(x,f(x))dx-\int_{l_1(t_l)}^{l_1(t_r)}P(x,g(x))dx\\&=\int_{l_1(t_l)}^{l_1(t_r)}(P(x,f(x))-P(x,g(x)))dx
\end{align*}
根据微积分基本定理,易得
$$
P(x,f(x))-P(x,g(x))= \int_{g(x)}^{f(x)}\frac{\pa P(x,y)}{\pa
  y}dy=-\int_{f(x)}^{g(x)}\frac{\pa P(x,y)}{\pa y}dy,
$$
因此,
$$
\oint_lP(x,y)dx=-\int_{l_1(t_l)}^{l_1(t_r)}\int_{f(x)}^{g(x)}\frac{\pa
  P(x,y)}{\pa y}dydx=-\iint_D \frac{\pa P(x,y)}{\pa y}dydx.
$$
在曲线 $l$ 最简单的情形下,我们再来看沿着纵坐标方向的力对质点的纵向运动
所做的功
$$
\oint_lQ(x,y)dy.
$$
这个时候,我们要考虑分割.如图 \ref{fig:6},我们把图 \ref{fig:5}在纵方向上
分割,有 do,re,mi,fa,so,la,si,doo 这有限条分割线,这有限条分割线全是曲
线 $l$的水平线,但不是所有水平线都要形成一条分割线.水平线肯定与曲线在某
些点相切,所有这些切点中,若存在着某个切点,设这个点的坐标
为$(l_1(t'),l_2(t'))$,当在 $t'$ 附近$l_2(t)$ 不再单调,则我们把水平线作
为一条分割线,而若在 $t'$ 附近 $l_2(t)$ 单调,则我们不把水平线作为一条分
割线.易得在相邻\footnote{这里的相邻不是水平线距离的相邻,而是随着$t$ 的
  变化,而依次经过的相邻水平线.}的分割线之间,$l_2(t)$ 都是随着$t$ 单调变
化
的(为什么?).\\

\begin{figure}\centering
  \scalebox{1} % Change this value to rescale the drawing.
  {
    \begin{pspicture}(0,-3.3614042)(13.735313,3.3814042)
      \psbezier[linewidth=0.04](0.28,0.55140406)(0.28,-0.24859594)(1.3638287,-2.7557878)(2.32,-3.048596)(3.2761712,-3.3414042)(4.443497,-2.1659372)(5.26,-1.588596)(6.076503,-1.0112547)(5.633545,-0.6314321)(6.58,-0.30859593)(7.526455,0.01424026)(8.75926,-1.5612803)(8.88,-0.56859595)
      \psline[linewidth=0.04cm](8.88,-0.56859595)(8.88,0.99140406)
      \psarc[linewidth=0.04](9.69,0.92140406){0.83}{0.97102195}{180.0}
      \psbezier[linewidth=0.04](10.52,0.93140405)(10.52,0.13140404)(12.1,0.9114041)(12.1,1.7114041)
      \pscustom[linewidth=0.04] { \newpath \moveto(12.1,1.711404)
        \lineto(12.1,1.7914041)
        \curveto(12.1,1.8314041)(12.095,1.9064041)(12.09,1.9414041)
        \curveto(12.085,1.9764041)(12.04,2.0364041)(12.0,2.061404)
        \curveto(11.96,2.086404)(11.905,2.136404)(11.89,2.1614041)
        \curveto(11.875,2.186404)(11.84,2.2464042)(11.82,2.281404)
        \curveto(11.8,2.316404)(11.75,2.3714042)(11.72,2.3914042)
        \curveto(11.69,2.4114041)(11.62,2.441404)(11.58,2.451404)
        \curveto(11.54,2.461404)(11.46,2.491404)(11.42,2.511404)
        \curveto(11.38,2.531404)(11.3,2.5664039)(11.26,2.5814042)
        \curveto(11.22,2.5964043)(11.14,2.6264043)(11.1,2.6414042)
        \curveto(11.06,2.656404)(10.98,2.6814039)(10.94,2.6914039)
        \curveto(10.9,2.701404)(10.82,2.721404)(10.78,2.731404)
        \curveto(10.74,2.7414038)(10.66,2.7564037)(10.62,2.761404)
        \curveto(10.58,2.766404)(10.5,2.781404)(10.46,2.791404)
        \curveto(10.42,2.801404)(10.34,2.816404)(10.3,2.8214042)
        \curveto(10.26,2.826404)(10.18,2.8414042)(10.14,2.8514042)
        \curveto(10.1,2.8614042)(10.02,2.876404)(9.98,2.8814042)
        \curveto(9.94,2.886404)(9.86,2.9014041)(9.82,2.9114044)
        \curveto(9.78,2.9214044)(9.7,2.9364042)(9.66,2.941404)
        \curveto(9.62,2.946404)(9.545,2.9514039)(9.51,2.951404)
        \curveto(9.475,2.9514043)(9.405,2.9564042)(9.37,2.961404)
        \curveto(9.335,2.966404)(9.26,2.9714038)(9.22,2.971404)
        \curveto(9.18,2.9714043)(9.095,2.9714043)(9.05,2.971404)
        \curveto(9.005,2.9714038)(8.92,2.9714038)(8.88,2.971404)
        \curveto(8.84,2.9714043)(8.765,2.9714043)(8.73,2.971404)
        \curveto(8.695,2.9714038)(8.62,2.9714038)(8.58,2.971404)
        \curveto(8.54,2.9714043)(8.46,2.9714043)(8.42,2.971404)
        \curveto(8.38,2.9714038)(8.295,2.976404)(8.25,2.981404)
        \curveto(8.205,2.9864042)(8.12,2.9964042)(8.08,3.001404)
        \curveto(8.04,3.006404)(7.96,3.011404)(7.92,3.011404)
        \curveto(7.88,3.0114043)(7.8,3.0164042)(7.76,3.021404)
        \curveto(7.72,3.026404)(7.64,3.036404)(7.6,3.041404)
        \curveto(7.56,3.0464044)(7.48,3.0514042)(7.44,3.051404)
        \curveto(7.4,3.051404)(7.325,3.051404)(7.29,3.051404)
        \curveto(7.255,3.0514042)(7.18,3.0514042)(7.14,3.051404)
        \curveto(7.1,3.051404)(7.02,3.051404)(6.98,3.051404)
        \curveto(6.94,3.0514042)(6.86,3.0514042)(6.82,3.051404)
        \curveto(6.78,3.051404)(6.695,3.051404)(6.65,3.051404)
        \curveto(6.605,3.0514042)(6.52,3.0514042)(6.48,3.051404)
        \curveto(6.44,3.051404)(6.36,3.051404)(6.32,3.051404)
        \curveto(6.28,3.0514042)(6.2,3.0564044)(6.16,3.061404)
        \curveto(6.12,3.0664039)(6.035,3.071404)(5.99,3.0714042)
        \curveto(5.945,3.0714042)(5.875,3.0664043)(5.85,3.061404)
        \curveto(5.825,3.0564039)(5.755,3.051404)(5.71,3.051404)
        \curveto(5.665,3.0514042)(5.585,3.0514042)(5.55,3.051404)
        \curveto(5.515,3.051404)(5.44,3.051404)(5.4,3.051404)
        \curveto(5.36,3.0514042)(5.28,3.0514042)(5.24,3.051404)
        \curveto(5.2,3.051404)(5.12,3.051404)(5.08,3.051404)
        \curveto(5.04,3.0514042)(4.96,3.0514042)(4.92,3.051404)
        \curveto(4.88,3.051404)(4.8,3.051404)(4.76,3.051404)
        \curveto(4.72,3.0514042)(4.64,3.0514042)(4.6,3.051404)
        \curveto(4.56,3.051404)(4.48,3.051404)(4.44,3.051404)
        \curveto(4.4,3.0514042)(4.315,3.0514042)(4.27,3.051404)
        \curveto(4.225,3.051404)(4.14,3.051404)(4.1,3.051404)
        \curveto(4.06,3.0514042)(3.98,3.0514042)(3.94,3.051404)
        \curveto(3.9,3.051404)(3.81,3.051404)(3.76,3.051404)
        \curveto(3.71,3.0514042)(3.62,3.0514042)(3.58,3.051404)
        \curveto(3.54,3.051404)(3.455,3.046404)(3.41,3.041404)
        \curveto(3.365,3.0364044)(3.28,3.0214043)(3.24,3.011404)
        \curveto(3.2,3.001404)(3.115,2.976404)(3.07,2.961404)
        \curveto(3.025,2.9464042)(2.94,2.9214044)(2.9,2.9114041)
        \curveto(2.86,2.9014041)(2.785,2.876404)(2.75,2.8614042)
        \curveto(2.715,2.8464043)(2.64,2.8114042)(2.6,2.791404)
        \curveto(2.56,2.771404)(2.475,2.736404)(2.43,2.7214038)
        \curveto(2.385,2.706404)(2.305,2.6814039)(2.27,2.6714041)
        \curveto(2.235,2.6614044)(2.16,2.6364043)(2.12,2.6214042)
        \curveto(2.08,2.606404)(2.0,2.571404)(1.96,2.551404)
        \curveto(1.92,2.5314043)(1.84,2.491404)(1.8,2.4714038)
        \curveto(1.76,2.4514039)(1.68,2.421404)(1.64,2.4114041)
        \curveto(1.6,2.4014044)(1.525,2.3714044)(1.49,2.3514042)
        \curveto(1.455,2.3314037)(1.415,2.2764037)(1.41,2.241404)
        \curveto(1.405,2.2064042)(1.4,2.1364045)(1.4,2.1014042)
        \curveto(1.4,2.0664039)(1.4,2.0014038)(1.4,1.9714041)
        \curveto(1.4,1.9414045)(1.4,1.8764044)(1.4,1.8414041)
        \curveto(1.4,1.8064038)(1.4,1.7364038)(1.4,1.701404)
        \curveto(1.4,1.6664041)(1.4,1.5964041)(1.4,1.561404)
        \curveto(1.4,1.5264038)(1.4,1.4564039)(1.4,1.421404)
        \curveto(1.4,1.3864042)(1.4,1.3164041)(1.4,1.2814041)
        \curveto(1.4,1.2464039)(1.4,1.1764041)(1.4,1.1414042)
        \curveto(1.4,1.1064041)(1.405,1.0414041)(1.41,1.0114042)
        \curveto(1.415,0.9814041)(1.42,0.9114041)(1.42,0.8714041)
        \curveto(1.42,0.8314044)(1.415,0.7564044)(1.41,0.7214041)
        \curveto(1.405,0.6864041)(1.4,0.6214041)(1.4,0.5914041)
        \curveto(1.4,0.5614044)(1.395,0.4964044)(1.39,0.4614041)
        \curveto(1.385,0.4264041)(1.345,0.3714041)(1.31,0.3514041)
        \curveto(1.275,0.33140442)(1.2,0.3064044)(1.16,0.3014041)
        \curveto(1.12,0.2964041)(1.055,0.2964041)(1.03,0.3014041)
        \curveto(1.005,0.3064044)(0.975,0.3414044)(0.97,0.3714041)
        \curveto(0.965,0.4014041)(0.955,0.4664041)(0.95,0.5014041)
        \curveto(0.945,0.53640443)(0.94,0.6064044)(0.94,0.6414041)
        \curveto(0.94,0.6764038)(0.925,0.7464038)(0.91,0.7814041)
        \curveto(0.895,0.8164044)(0.865,0.8864044)(0.85,0.9214041)
        \curveto(0.835,0.9564038)(0.8,1.0214038)(0.78,1.0514041)
        \curveto(0.76,1.0814041)(0.71,1.1414042)(0.68,1.1714041)
        \curveto(0.65,1.2014037)(0.575,1.2314038)(0.53,1.2314041)
        \curveto(0.485,1.2314044)(0.42,1.2214044)(0.4,1.2114041)
        \curveto(0.38,1.2014035)(0.36,1.1564035)(0.36,1.121404)
        \curveto(0.36,1.0864044)(0.355,1.0164044)(0.35,0.9814041)
        \curveto(0.345,0.9464038)(0.33,0.8814038)(0.32,0.8514041)
        \curveto(0.31,0.8214047)(0.3,0.7564047)(0.3,0.7214041)
        \curveto(0.3,0.6864038)(0.3,0.6164038)(0.3,0.5814041)
        \curveto(0.3,0.5464044)(0.3,0.4614044)(0.3,0.4114041) }
      \psline[linewidth=0.04cm](0.5,-3.1085956)(13.06,-3.048596)
      \psline[linewidth=0.04cm](5.26,-0.30859587)(8.54,-0.28859583)
      \psline[linewidth=0.04cm](7.94,-0.8485958)(9.2,-0.86859584)
      \psline[linewidth=0.04cm](8.84,1.751404)(10.6,1.751404)
      \psline[linewidth=0.04cm](9.96,0.5914042)(11.94,0.5914042)
      \psline[linewidth=0.04cm](0.5,3.051404)(12.82,3.0914042)
      \psline[linewidth=0.04cm](0.56,0.23140416)(2.68,0.23140416)
      \psline[linewidth=0.04cm](0.0,1.2114042)(1.08,1.2114042)
      \usefont{T1}{ptm}{m}{n} \rput(13.515626,-3.103596){do}
      \usefont{T1}{ptm}{m}{n} \rput(9.7675,-0.8435958){re}
      \usefont{T1}{ptm}{m}{n} \rput(4.9681253,-0.22359584){mi}
      \usefont{T1}{ptm}{m}{n} \rput(3.0853124,0.27640414){fa}
      \usefont{T1}{ptm}{m}{n} \rput(12.55,0.6564042){so}
      \usefont{T1}{ptm}{m}{n} \rput(0.2084375,1.5964042){la}
      \usefont{T1}{ptm}{m}{n} \rput(8.04375,1.996404){si}
      \usefont{T1}{ptm}{m}{n} \rput(13.2856245,3.1964042){doo}
      \psline[linewidth=0.04cm](0.32,0.2214041)(8.9,0.2614041)
      \psline[linewidth=0.04cm](0.5,-0.2985959)(8.88,-0.2985959)
      \psline[linewidth=0.04cm](0.68,-0.8785959)(10.16,-0.8585959)
      \psline[linewidth=0.04cm](1.08,1.2014041)(11.92,1.2214041)
      \psline[linewidth=0.04cm](1.42,1.741404)(12.12,1.741404)
      \psline[linewidth=0.04cm](10.0,0.6014041)(0.3,0.6014041)
    \end{pspicture}
  }
  \caption{}\label{fig:6}
\end{figure}

容易证明,这些分割线把区域 $D$ 分割成了有限个区域,这些区域都是如
图 \ref{fig:7} 的形状.这个区域 $GXFK$的边界共有四条曲线组
成,其中 $F,G$是水平的线段,而 $K,X$ 是作为曲线 $l$ 的被截取一部分,在这些
部分上,$l_1(t),l_2(t)$都是单调变化的(未必严格单调).当然,水平线段 $F,G$
是有可能退化成点的.而
且,曲线 $X$ 和 $K$ 分别是闭区间上的函数 $x=h(y)$ 和 $x=p(y)$ 的图像.\\

下面,针对 图 \ref{fig:7} 中的区域,我们来讨论沿着纵坐标方向的力对质点的
纵向运动所做的功.我们不妨设质点从 $G$ 与 $K$ 的交点开始出发,沿着逆时针
方向运动,当质点还在 $G$ 上时,纵坐标方向的力对质点不作功,然后质点运行
到$G$ 和 $X$ 的交点,不妨设交点坐标为 $(l_1(t_{GX}),l_2(t_{GX}))$,从此质
点开始在 $X$ 上运动,直到质点运动到 $X$ 与 $F$ 的交
点$(l_1(t_{XF}),l_2(t_{XF}))$.在这个过程里,沿着纵方向的力对质点做的功为
$$
\int_{l_2(t_{GX})}^{l_2(t_{XF})}Q(h(y),y) dy.
$$
质点在经过曲线 $X$ 后,进入曲线 $F$,在 $F$ 上运动的时候,沿着纵方向的力对
质点不作功.然后质点达到曲线 $F$ 和曲线 $K$ 的交点,不妨设交点坐标
为$(l_1(t_{FK}),l_2(t_{FK}))$.再走完曲线 $K$,到达 $K$ 和 $G$ 的交
点 $(l_1(t_{KG}),l_2(t_{KG}))$.质点走完曲线 $K$ 的过程里,沿着纵方向的力
对质点做的功为
$$
\int_{l_2(t_{FK})}^{l_2(t_{KG})}Q(p(y),y)dy.
$$
因此,质点沿着逆时针方向,绕着图 \ref{fig:7} 中的曲线走一圈,纵方向的力对
质点所作的功为
\begin{align*}
  \int_{l_2(t_{GX})}^{l_2(t_{XF})}Q(h(y),y)
  dy+\int_{l_2(t_{FK})}^{l_2(t_{KG})}Q(p(y),y)dy&=\int_{l_2(t_{GX})}^{l_2(t_{XF})}(Q(h(y),y)-Q(p(y),y))dy\\&=\int_{l_2(t_{GX})}^{l_2(t_{XF})}\int_{p(y)}^{h(y)}\frac{\pa
    Q(x,y)}{\pa x}dxdy\\&=\iint_{GXFK}\frac{\pa Q(x,y)}{\pa x}dxdy.
\end{align*}

由于质点在水平线段 $F,G$ 上运动时,纵方向的力不作功,因此我们可以忽略水平
线.事实上,只有质点在曲线 $X$ 和 $K$ 上运动时,纵方向的力才可能作功.因此,当
我们整体地来看,质点绕曲线 $l$ 一周时,纵坐标方向的力做的功是在各个被分割
的小区域上做的功的累加,结果为
$$\iint_D \frac{\pa Q(x,y)}{\pa x}dxdy.$$
因此,对于曲线 $l$ 的简单情形,我们得知,横坐标和纵坐标作的功的总和为
$$
\iint_D \left(\frac{\pa Q(x,y)}{\pa x}-\frac{\pa P(x,y)}{\pa
    y}\right)dxdy=\oint_l P(x,y)dx+Q(x,y)dy.
$$



\begin{figure}\centering
  \scalebox{1} % Change this value to rescale the drawing.
  {
    \begin{pspicture}(0,-1.841875)(9.729688,1.841875)
      \pscustom[linewidth=0.04] { \newpath
        \moveto(0.1740625,1.1634375) \lineto(0.2440625,1.0434375)
        \curveto(0.2790625,0.9834375)(0.3290625,0.8884375)(0.3440625,0.8534375)
        \curveto(0.3590625,0.8184375)(0.3890625,0.7384375)(0.4040625,0.6934375)
        \curveto(0.4190625,0.6484375)(0.4390625,0.5634375)(0.4440625,0.5234375)
        \curveto(0.4490625,0.4834375)(0.4590625,0.4084375)(0.4640625,0.3734375)
        \curveto(0.4690625,0.3384375)(0.4840625,0.2484375)(0.4940625,0.1934375)
        \curveto(0.5040625,0.1384375)(0.5340625,0.0434375)(0.5540625,0.0034375)
        \curveto(0.5740625,-0.0365625)(0.6140625,-0.1165625)(0.6340625,-0.1565625)
        \curveto(0.6540625,-0.1965625)(0.7090625,-0.2665625)(0.7440625,-0.2965625)
        \curveto(0.7790625,-0.3265625)(0.8340625,-0.3915625)(0.8540625,-0.4265625)
        \curveto(0.8740625,-0.4615625)(0.9190625,-0.5165625)(0.9440625,-0.5365625)
        \curveto(0.9690625,-0.5565625)(1.0140625,-0.6015625)(1.0340625,-0.6265625)
        \curveto(1.0540625,-0.6515625)(1.0940624,-0.7215625)(1.1140625,-0.7665625)
        \curveto(1.1340625,-0.8115625)(1.1790625,-0.8915625)(1.2040625,-0.9265625)
        \curveto(1.2290626,-0.9615625)(1.2740625,-1.0315624)(1.2940625,-1.0665625)
        \curveto(1.3140625,-1.1015625)(1.3690625,-1.1565624)(1.4040625,-1.1765625)
        \curveto(1.4390625,-1.1965625)(1.4990625,-1.2465625)(1.5240625,-1.2765625)
        \curveto(1.5490625,-1.3065625)(1.6090626,-1.3465625)(1.6440625,-1.3565625)
        \curveto(1.6790625,-1.3665625)(1.7290626,-1.3815625)(1.7740625,-1.3965625)
      } \pscustom[linewidth=0.04] { \newpath
        \moveto(7.9540625,-1.4165626) \lineto(8.034062,-1.3965625)
        \curveto(8.074062,-1.3865625)(8.149062,-1.3415625)(8.184063,-1.3065625)
        \curveto(8.219063,-1.2715625)(8.279062,-1.2015625)(8.304063,-1.1665626)
        \curveto(8.329062,-1.1315625)(8.379063,-1.0615625)(8.404062,-1.0265625)
        \curveto(8.429063,-0.9915625)(8.484062,-0.9165625)(8.514063,-0.8765625)
        \curveto(8.544063,-0.8365625)(8.594063,-0.7615625)(8.614062,-0.7265625)
        \curveto(8.634063,-0.6915625)(8.6640625,-0.6215625)(8.674063,-0.5865625)
        \curveto(8.684063,-0.5515625)(8.719063,-0.4815625)(8.744062,-0.4465625)
        \curveto(8.769062,-0.4115625)(8.819062,-0.3365625)(8.844063,-0.2965625)
        \curveto(8.869062,-0.2565625)(8.924063,-0.1765625)(8.954062,-0.1365625)
        \curveto(8.984062,-0.0965625)(9.0390625,-0.0165625)(9.064062,0.0234375)
        \curveto(9.089063,0.0634375)(9.134063,0.1384375)(9.154062,0.1734375)
        \curveto(9.174063,0.2084375)(9.209063,0.2784375)(9.224063,0.3134375)
        \curveto(9.239062,0.3484375)(9.264063,0.4184375)(9.274062,0.4534375)
        \curveto(9.284062,0.4884375)(9.299063,0.5584375)(9.304063,0.5934375)
        \curveto(9.309063,0.6284375)(9.314062,0.7084375)(9.314062,0.7534375)
        \curveto(9.314062,0.7984375)(9.309063,0.8834375)(9.304063,0.9234375)
        \curveto(9.299063,0.9634375)(9.2890625,1.0384375)(9.284062,1.0734375)
        \curveto(9.279062,1.1084375)(9.264063,1.1534375)(9.254063,1.1634375)
        \curveto(9.244062,1.1734375)(9.234062,1.1634375)(9.234062,1.1034375)
      }
      \psline[linewidth=0.04cm](1.7740625,-1.3965625)(7.9740624,-1.4165626)
      \psline[linewidth=0.04cm](0.1940625,1.1634375)(9.154062,1.1434375)
      \usefont{T1}{ptm}{m}{n} \rput(4.54375,1.6684375){F}
      \usefont{T1}{ptm}{m}{n} \rput(4.7951565,-1.6915625){G}
      \usefont{T1}{ptm}{m}{n} \rput(0.12265625,-0.0315625){K}
      \usefont{T1}{ptm}{m}{n} \rput(9.560625,0.2484375){X}
    \end{pspicture}
  }\caption{}\label{fig:7}
\end{figure}
\bigskip而对于曲线 $l$ 的更复杂的情形,也就是,$l$ 仅仅是不自交的曲线(两
头的端点除外,也就是除去$(l_1(a),l_2(a))=(l_1(b),l_2(b))$ 这个特例) 这种
情形,我们该怎么办呢?这个时候,我们要考虑分割.如图 \ref{fig:8},我们把图在
纵方向上分割,形成有限条分割线,这有限条分割线全是曲线 $l$的水平线,但不是
所有水平线都要形成一条分割线.水平线肯定与曲线在某些点相切,所有这些切点
中,若存在着某个切点,设这个点的坐标
为$(l_1(t'),l_2(t'))$,当在 $t'$ 附近$l_2(t)$ 不再单调,则我们把水平线作
为一条分割线,而若在 $t'$ 附近 $l_2(t)$ 单调,则我们不把水平线作为一条分
割线.我们把图在水平方向上也进行分割,形成有限条分割线,这有限条分割线全是
曲线 $l$的垂直线,但不是所有垂直线都要形成一条分割线.垂直线肯定与曲线在
某些点相切,所有这些切点中,若存在着某个切点,设这个点的坐标
为$(l_1(t''),l_2(t''))$,当在 $t''$ 附近$l_1(t)$ 不再单调,则我们把水平线
作为一条分割线,而若在 $t''$ 附近 $l_1(t)$ 单调,则我们不把水平线作为一条
分割线.易得在相邻\footnote{这里的相邻不是水平线和垂直线距离的相邻,而是
  随着$t$ 的变化,而依次经过的相邻水平线和垂直线.}的分割线之
间,$l_2(t)$和 $l_1(t)$
都是随着$t$ 单调变化的(为什么?).这样子,就成为了已解决情形.\\

\begin{figure}
  \centering \scalebox{1} % Change this value to rescale the drawing.
  {
    \begin{pspicture}(0,-4.12)(11.78,4.12)
      \psbezier[linewidth=0.04](0.7,1.02)(0.7,0.22)(7.464309,-2.5919437)(8.1,-1.82)(8.735691,-1.0480564)(6.4457154,-0.6293288)(6.76,0.32)(7.0742846,1.2693288)(10.3988085,0.7400007)(10.4,1.74)(10.401191,2.7399993)(7.4340777,3.5795586)(6.48,3.28)(5.5259223,2.9804413)(6.357999,1.6632311)(5.36,1.6)(4.362001,1.5367689)(4.8928704,3.0286498)(3.92,3.26)(2.9471295,3.4913502)(1.3993076,2.7805188)(0.8,1.98)
      \pscustom[linewidth=0.04] { \newpath \moveto(0.86,1.98)
        \lineto(0.8,1.94) \curveto(0.77,1.92)(0.725,1.885)(0.71,1.87)
        \curveto(0.695,1.855)(0.655,1.815)(0.63,1.79)
        \curveto(0.605,1.765)(0.58,1.705)(0.58,1.67)
        \curveto(0.58,1.635)(0.575,1.575)(0.57,1.55)
        \curveto(0.565,1.525)(0.56,1.46)(0.56,1.42)
        \curveto(0.56,1.38)(0.56,1.31)(0.56,1.28)
        \curveto(0.56,1.25)(0.58,1.19)(0.6,1.16)
        \curveto(0.62,1.13)(0.66,1.065)(0.68,1.03)
        \curveto(0.7,0.995)(0.74,0.935)(0.76,0.91)
        \curveto(0.78,0.885)(0.805,0.85)(0.82,0.82) }
      \pscustom[linewidth=0.04] { \newpath \moveto(0.88,2.06)
        \lineto(0.83,2.01)
        \curveto(0.805,1.985)(0.755,1.925)(0.73,1.89)
        \curveto(0.705,1.855)(0.65,1.8)(0.62,1.78)
        \curveto(0.59,1.76)(0.56,1.7)(0.56,1.66)
        \curveto(0.56,1.62)(0.56,1.545)(0.56,1.51)
        \curveto(0.56,1.475)(0.57,1.415)(0.58,1.39)
        \curveto(0.59,1.365)(0.61,1.305)(0.62,1.27)
        \curveto(0.63,1.235)(0.65,1.17)(0.66,1.14)
        \curveto(0.67,1.11)(0.69,1.04)(0.7,1.0)
        \curveto(0.71,0.96)(0.74,0.895)(0.76,0.87)
        \curveto(0.78,0.845)(0.82,0.805)(0.88,0.76) }
      \pscustom[linewidth=0.04] { \newpath \moveto(0.62,1.2)
        \lineto(0.61,1.26) \curveto(0.605,1.29)(0.59,1.295)(0.58,1.27)
        \curveto(0.57,1.245)(0.575,1.21)(0.62,1.18) }
      \psline[linewidth=0.04cm](0.0,1.56)(11.7,1.58)
      \psline[linewidth=0.04cm](0.12,3.36)(11.76,3.36)
      \psline[linewidth=0.04cm](6.7,4.1)(6.7,-4.1)
    \end{pspicture}
  }
  \caption{}\label{fig:8}
\end{figure}

还有一种更复杂的情况,如果曲线 $l$ 自交了,那该怎么办?此时,情形如
图 \ref{fig:9}:
\begin{figure}
  \centering \scalebox{1} % Change this value to rescale the drawing.
  {
    \begin{pspicture}(0,-4.1031947)(16.110449,4.1031947)
      \psbezier[linewidth=0.04](0.06,0.34663287)(0.06,-0.45336717)(8.746049,-4.0831947)(9.74,-3.9733672)(10.733951,-3.8635402)(13.276668,-2.353228)(13.26,-1.3533671)(13.243332,-0.3535061)(10.672713,1.6569383)(10.06,0.8666328)(9.447287,0.07632738)(14.309552,-4.028453)(15.2,-3.573367)(16.090448,-3.1182816)(15.34014,1.9135506)(14.66,2.646633)(13.97986,3.3797152)(7.6706314,4.0831947)(6.68,3.9466329)(5.6893687,3.810071)(0.5976637,1.7097923)(0.06,0.8666328)
      \pscustom[linewidth=0.04] { \newpath \moveto(0.06,0.84663296)
        \lineto(0.08,0.84663296)
        \curveto(0.09,0.84663296)(0.11,0.84663296)(0.12,0.84663296) }
      \pscustom[linewidth=0.04] { \newpath \moveto(0.14,0.96663296)
        \lineto(0.1,0.91663295)
        \curveto(0.08,0.8916329)(0.05,0.8366329)(0.04,0.80663294)
        \curveto(0.03,0.7766329)(0.02,0.7116329)(0.02,0.67663294)
        \curveto(0.02,0.6416329)(0.015,0.5716329)(0.01,0.53663296)
        \curveto(0.005,0.5016329)(0.0,0.43163294)(0.0,0.39663294)
        \curveto(0.0,0.36163294)(0.02,0.30163294)(0.04,0.27663293)
        \curveto(0.06,0.25163293)(0.13,0.18663293)(0.18,0.14663294) }
      \usefont{T1}{ptm}{m}{n} \rput(12.7265625,-2.6318054){A}
      \usefont{T1}{ptm}{m}{n} \rput(10.196719,1.2081945){O}
      \usefont{T1}{ptm}{m}{n} \rput(12.8671875,-0.05180548){B}
    \end{pspicture}
  }\caption{}\label{fig:9}
\end{figure}
首先,由引理易得,当光滑曲线存在自交的时候,只存在有限个自交点.因此,我们不
妨只看一个自交点.如图\ref{fig:10},图中画斜线的阴影部分是区域 $D$,且区
域 $D$ 是开区域,则易得区域 $D$ 的边界是自交的曲线.此时,质点沿着曲线逆时
针运动,只可能是沿着路径 $A-O-B-A$ 运动,而不可能是沿着路径 $A-B-O-A$ 运
动.这个时候,易得
Green 定理仍然成立.\\
\begin{figure}
  \centering \scalebox{1} {
    \begin{pspicture}(0,-4.1031947)(16.110449,4.1031947)
      \psbezier[linewidth=0.04](0.06,0.34663287)(0.06,-0.45336717)(8.746049,-4.0831947)(9.74,-3.9733672)(10.733951,-3.8635402)(13.276668,-2.353228)(13.26,-1.3533671)(13.243332,-0.3535061)(10.672713,1.6569383)(10.06,0.8666328)(9.447287,0.07632738)(14.309552,-4.028453)(15.2,-3.573367)(16.090448,-3.1182816)(15.34014,1.9135506)(14.66,2.646633)(13.97986,3.3797152)(7.6706314,4.0831947)(6.68,3.9466329)(5.6893687,3.810071)(0.5976637,1.7097923)(0.06,0.8666328)
      \pscustom[linewidth=0.04] { \newpath \moveto(0.06,0.84663296)
        \lineto(0.08,0.84663296)
        \curveto(0.09,0.84663296)(0.11,0.84663296)(0.12,0.84663296) }
      \pscustom[linewidth=0.04] { \newpath \moveto(0.14,0.96663296)
        \lineto(0.1,0.91663295)
        \curveto(0.08,0.8916329)(0.05,0.8366329)(0.04,0.80663294)
        \curveto(0.03,0.7766329)(0.02,0.7116329)(0.02,0.67663294)
        \curveto(0.02,0.6416329)(0.015,0.5716329)(0.01,0.53663296)
        \curveto(0.005,0.5016329)(0.0,0.43163294)(0.0,0.39663294)
        \curveto(0.0,0.36163294)(0.02,0.30163294)(0.04,0.27663293)
        \curveto(0.06,0.25163293)(0.13,0.18663293)(0.18,0.14663294) }
      \psline[linewidth=0.04cm](1.52,1.6631945)(0.9,0.30319452)
      \psline[linewidth=0.04cm](2.74,2.1831946)(1.74,-0.31680548)
      \psline[linewidth=0.04cm](4.48,2.9031944)(2.92,-0.8768055)
      \psline[linewidth=0.04cm](6.3,3.6231945)(4.12,-1.5368055)
      \psline[linewidth=0.04cm](7.44,3.7631946)(5.08,-1.8968055)
      \psline[linewidth=0.04cm](8.72,3.5431945)(6.22,-2.5168054)
      \psline[linewidth=0.04cm](10.12,3.5031946)(7.58,-2.9168055)
      \psline[linewidth=0.04cm](11.7,3.0631945)(11.04,1.4031945)
      \psline[linewidth=0.04cm](10.22,-0.7368055)(9.12,-3.1968055)
      \psline[linewidth=0.04cm](13.18,2.8231945)(12.38,1.0231946)
      \psline[linewidth=0.04cm](11.28,-1.8568054)(10.62,-3.3568056)
      \psline[linewidth=0.04cm](14.94,1.4431945)(13.92,-1.0768055)
      \psline[linewidth=0.04cm](13.76,-1.4568055)(13.36,-2.3968055)
      \psline[linewidth=0.04cm](15.28,-0.89680547)(14.46,-3.0168054)
    \end{pspicture}
  }
  \caption{}
  \label{fig:10}
\end{figure}
\end{proof}


\begin{thebibliography}{}
\bibitem[1]{ouyangguangzhong}欧阳光中,朱学炎,金福临,陈传璋.数学分
  析[M].第三版.北京:高等教育出版社,2008
\bibitem[2]{kodaira}小平邦彦.微积分入门[M].北京:人民邮电出版社,2008:410
\end{thebibliography}
% ----------------------------------------------------------------------------------------
\end{document}


v





