\documentclass[twoside,11pt]{article} 
\usepackage{amsmath,amsfonts,bm}
\usepackage{hyperref}
\usepackage{amsthm} 
\usepackage{amssymb}
\usepackage{framed,mdframed}
\usepackage{graphicx,color} 
\usepackage{mathrsfs,xcolor} 
\usepackage[all]{xy}
\usepackage{fancybox} 
% \usepackage{CJKutf8}
\usepackage{xeCJK,yhmath}
\newtheorem{theorem}{定理}
\newtheorem{lemma}{引理}
\newtheorem{corollary}{推论}
\newtheorem*{exercise}{习题}
\usepackage{pgf,tikz}
\usetikzlibrary{arrows}
\pagestyle{empty}
\usepackage{pstricks-add}
\setCJKmainfont[BoldFont=Adobe Heiti Std R]{Adobe Song Std L}
% \usepackage{latexdef}
\def\ZZ{\mathbb{Z}} \topmargin -0.40in \oddsidemargin 0.08in
\evensidemargin 0.08in \marginparwidth 0.00in \marginparsep 0.00in
\textwidth 16cm \textheight 24cm \newcommand{\D}{\displaystyle}
\newcommand{\ds}{\displaystyle} \renewcommand{\ni}{\noindent}
\newcommand{\pa}{\partial} \newcommand{\Om}{\Omega}
\newcommand{\om}{\omega} \newcommand{\sik}{\sum_{i=1}^k}
\newcommand{\vov}{\Vert\omega\Vert} \newcommand{\Umy}{U_{\mu_i,y^i}}
\newcommand{\lamns}{\lambda_n^{^{\scriptstyle\sigma}}}
\newcommand{\chiomn}{\chi_{_{\Omega_n}}}
\newcommand{\ullim}{\underline{\lim}} \newcommand{\bsy}{\boldsymbol}
\newcommand{\mvb}{\mathversion{bold}} \newcommand{\la}{\lambda}
\newcommand{\La}{\Lambda} \newcommand{\va}{\varepsilon}
\newcommand{\be}{\beta} \newcommand{\al}{\alpha}
\newcommand{\dis}{\displaystyle} \newcommand{\R}{{\mathbb R}}
\newcommand{\N}{{\mathbb N}} \newcommand{\cF}{{\mathcal F}}
\newcommand{\gB}{{\mathfrak B}} \newcommand{\eps}{\epsilon}
\renewcommand\refname{参考文献} \def \qed {\hfill \vrule height6pt
  width 6pt depth 0pt} \topmargin -0.40in \oddsidemargin 0.08in
\evensidemargin 0.08in \marginparwidth0.00in \marginparsep 0.00in
\textwidth 15.5cm \textheight 24cm \pagestyle{myheadings}
\markboth{\rm \centerline{}} {\rm \centerline{}}
\begin{document}
\title{\huge{\bf{习题22.1.1.4}}} \author{\small{叶卢
    庆\footnote{叶卢庆(1992---),男,杭州师范大学理学院数学与应用数学专业
      本科在读,E-mail:h5411167@gmail.com}}\\{\small{杭州师范大学理学院,浙
      江~杭州~310036}}} \date{}
\maketitle

% ----------------------------------------------------------------------------------------
% ABSTRACT AND KEYWORDS
% ----------------------------------------------------------------------------------------



\vspace{30pt} % Some vertical space between the abstract and first section

% ----------------------------------------------------------------------------------------
% ESSAY BODY
% ----------------------------------------------------------------------------------------
\begin{exercise}
利用Green公式计算曲线积分:
$$
\int_{\wideparen{AMO}}(e^x\sin y-my)dx+(e^x\cos y-m)dy,
$$
其中 $\wideparen{AMO}$ 为由点 $A(a,0)$ 至点 $O(0,0)$ 经过上半圆
周$x^2+y^2=ax$ 的道路.
\end{exercise}
\begin{proof}[解]
我们先画出上半圆周 $x^2+y^2-ax=0$.图像如下:
\begin{center}
\newrgbcolor{xdxdff}{0.49 0.49 1}
\psset{xunit=1.0cm,yunit=1.0cm,algebraic=true,dotstyle=o,dotsize=3pt 0,linewidth=0.8pt,arrowsize=3pt 2,arrowinset=0.25}
\begin{pspicture*}(-4.22,-1.04)(9.87,5.25)
\pscircle(0.97,0){1}
\psline{->}(-0.03,0)(5.5,0)
\psline{->}(-0.03,0)(0,5.17)
\rput[tl](5.23,-0.16){$$ X $$}
\rput[tl](-0.55,5.23){$$ Y $$}
\rput[tl](2.17,0.01){$$ A $$}
\rput[tl](-0.26,0.1){$$ O $$}
\parametricplot[linewidth=5.2pt,fillcolor=black,fillstyle=solid,opacity=0.4]{0.0}{3.141592653589793}{1*1*cos(t)+0*1*sin(t)+0.97|0*1*cos(t)+1*1*sin(t)+0}
\begin{scriptsize}
\psdots[dotstyle=*,linecolor=xdxdff](1.97,0)
\rput[bl](2.01,0.06){\xdxdff{$a$}}
\end{scriptsize}
\end{pspicture*}
\end{center}
令 $P(x,y)=e^x\sin y-my$,$Q(x,y)=e^x\cos y-m$.则易得 $P(x,0)=0$.因此当
质点在 $OA$ 上运动时,力对质点不作功.因此我们可以连接 $OA$,使得 $AMO$
成为一个封闭的区域.这样之后,就满足 Green定理的条件了.易得
$$
\iint_{D}\left(\frac{\pa Q(x,y)}{\pa x}-\frac{\pa
    P(x,y)}{\pa y}\right)dxdy=\oint_{\wideparen{AMOA}}P(x,y)dx+Q(x,y)dy.
$$
其中 $D$ 是上半圆区域.因此,
$$
\oint_{\wideparen{AMOA}}P(x,y)dx+Q(x,y)dy=\iint_D
\left(m\right)dxdy=\frac{m}{8}\pi a^2.
$$

\end{proof}
% BIBLIOGRAPHY
% ----------------------------------------------------------------------------------------
% 

% ----------------------------------------------------------------------------------------
\end{document}








