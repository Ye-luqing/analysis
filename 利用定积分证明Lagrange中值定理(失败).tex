\documentclass[a4paper]{article}
\usepackage{amsmath,amsfonts,amsthm,amssymb}
\usepackage{bm}
\usepackage{hyperref}
\usepackage{geometry}
\usepackage{yhmath}
\usepackage{pstricks-add}
\usepackage{framed,mdframed}
\usepackage{graphicx,color} 
\usepackage{mathrsfs,xcolor} 
\usepackage[all]{xy}
\usepackage{fancybox} 
\usepackage{xeCJK}
\newtheorem*{theo}{定理}
\newtheorem*{exe}{题目}
\newtheorem*{rem}{评论}
\newmdtheoremenv{lemma}{引理}
\newmdtheoremenv{corollary}{推论}
\newtheorem*{exa}{例}
\newenvironment{theorem}
{\bigskip\begin{mdframed}\begin{theo}}
    {\end{theo}\end{mdframed}\bigskip}
\newenvironment{exercise}
{\bigskip\begin{mdframed}\begin{exe}}
    {\end{exe}\end{mdframed}\bigskip}
\newenvironment{remark}
{\begin{mdframed}\begin{rem}}
    {\end{rem}\end{mdframed}\bigskip}
\geometry{left=2.5cm,right=2.5cm,top=2.5cm,bottom=2.5cm}
\setCJKmainfont[BoldFont=SimHei]{SimSun}
\renewcommand{\today}{\number\year 年 \number\month 月 \number\day 日}
\newcommand{\D}{\displaystyle}\newcommand{\ri}{\Rightarrow}
\newcommand{\ds}{\displaystyle} \renewcommand{\ni}{\noindent}
\newcommand{\ov}{\overrightarrow}
\newcommand{\pa}{\partial} \newcommand{\Om}{\Omega}
\newcommand{\om}{\omega} \newcommand{\sik}{\sum_{i=1}^k}
\newcommand{\vov}{\Vert\omega\Vert} \newcommand{\Umy}{U_{\mu_i,y^i}}
\newcommand{\lamns}{\lambda_n^{^{\scriptstyle\sigma}}}
\newcommand{\chiomn}{\chi_{_{\Omega_n}}}
\newcommand{\ullim}{\underline{\lim}} \newcommand{\bsy}{\boldsymbol}
\newcommand{\mvb}{\mathversion{bold}} \newcommand{\la}{\lambda}
\newcommand{\La}{\Lambda} \newcommand{\va}{\varepsilon}
\newcommand{\be}{\beta} \newcommand{\al}{\alpha}
\newcommand{\dis}{\displaystyle} \newcommand{\R}{{\mathbb R}}
\newcommand{\N}{{\mathbb N}} \newcommand{\cF}{{\mathcal F}}
\newcommand{\gB}{{\mathfrak B}} \newcommand{\eps}{\epsilon}
\renewcommand\refname{参考文献}\renewcommand\figurename{图}
\usepackage[]{caption2} 
\renewcommand{\captionlabeldelim}{}
\setlength\parindent{0pt}
\begin{document}
\title{\huge{\bf{利用定积分证明Lagrange中值定理}}} \author{\small{叶卢庆\footnote{叶卢庆(1992---),男,杭州师范大学理学院数学与应用数学专业本科在读,E-mail:yeluqingmathematics@gmail.com}}}
\maketitle
众所周知Lagrange中值定理叙述如下:
\begin{theorem}[Lagrange中值定理]
  若函数$f(x)$在区间$[a,b]$上连续,在$(a,b)$上可微,则存在$\xi\in
  (a,b)$,使得
\begin{equation}\label{eq:1}
f'(\xi)=\frac{f(b)-f(a)}{b-a}.
\end{equation}
\end{theorem}
很多书上都是利用构造辅助函数,再借助于Rolle定理将其证明.现在我们偏偏不
这么做.我们使用定积分.首先,我们证明微积分基本定理,即Newton-Leibniz公式.之
所以证明这个公式,是为了保证我们没有使用Lagrange中值定理来证明它,从而确
保没有逻辑循环.
\begin{theorem}[Newton-Leibniz]
若$f(x)$在区间$[a,b]$连续,在$(a,b)$可微,则
$$
\int_a^bf'(x)dx=f(b)-f(a).
$$
\end{theorem}
\begin{remark}
当然,$f'(x)$在$a,b$两点很可能不可微,$f$在点$a,b$处很可能只有单侧导数,从而导
致$f'(x)$在$a,b$处无定义,但是这并不影响该定理,因为黎曼积分并不会被
函数在某一点处的值所影响,因此即使$f'(x)$在$a,b$处不可微,我们也可以假设
其在$a,b$处可微,而不会对Newton-Leibniz定理造成任何影响.
\end{remark}
\begin{proof}[\textbf{证明}]
$f$在$(a,b)$可微,说明$\forall x\in (a,b)$,
\begin{equation}\label{eq:2}
f(x+\frac{1}{n})-f(x)=f'(x)\frac{1}{n}+o(\frac{1}{n}).
\end{equation}
其中$n\in \mathbf{N}^+$,且$\frac{1}{n}$小到能使得$x+\frac{1}{n}\in (a,b)$.且$\lim_{\Delta \frac{1}{n}\to
  0}o(\frac{1}{n})n=0$.将式\eqref{eq:2}累加,可得
\begin{equation}\label{eq:3}
f(b)-f(a)=\frac{1}{n}\sum_{i=0}^{n-1} f'(a+i \frac{b-a}{n})+\sum_{i=0}^{n-1} o(\frac{1}{n})=\frac{1}{n}\sum_{i=0}^{n-1} f'(a+i \frac{b-a}{n})+o(\frac{1}{n})n.
\end{equation}
令$n\to\infty$,方程\eqref{eq:3}即可变为
\begin{equation}
  \label{eq:4}
  f(b)-f(a)=\int_a^bf'(x)dx.
\end{equation}
这样就证明了Newton-Leibniz定理.
\end{proof}
现在我们去证明Lagrange中值定理.根据Newton-Leibniz定理,方程\eqref{eq:1}
等价于
\begin{equation}
  \label{eq:5}
f'(\xi)=\frac{1}{b-a}\int_a^bf'(x)dx.
\end{equation}
做到这里我们就继续不下去了.虽然如此,如果假设Lagrange中值定理是成立的,
那么通过式\eqref{eq:5},我们发现了更加广泛的积分中值定理.
\end{document}