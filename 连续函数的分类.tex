\documentclass[twoside,11pt]{article} 
\usepackage{amsmath,amsfonts,bm}
\usepackage{hyperref}
\usepackage{amsthm} 
\usepackage{amssymb}
\usepackage{framed,mdframed}
\usepackage{graphicx,color} 
\usepackage{mathrsfs,xcolor} 
\usepackage[all]{xy}
\usepackage{fancybox} 
%\usepackage{CJKutf8}
\usepackage{xeCJK}
\newtheorem{theorem}{定理}
\newtheorem{lemma}{引理}
\newtheorem{corollary}{推论}
\newtheorem{definition}{定义}
\setCJKmainfont[BoldFont=Adobe Heiti Std R]{Adobe Song Std L}
% \usepackage{latexdef}
\def\ZZ{\mathbb{Z}} \topmargin -0.40in \oddsidemargin 0.08in
\evensidemargin 0.08in \marginparwidth 0.00in \marginparsep 0.00in
\textwidth 16cm \textheight 24cm \newcommand{\D}{\displaystyle}
\newcommand{\ds}{\displaystyle} \renewcommand{\ni}{\noindent}
\newcommand{\pa}{\partial} \newcommand{\Om}{\Omega}
\newcommand{\om}{\omega} \newcommand{\sik}{\sum_{i=1}^k}
\newcommand{\vov}{\Vert\omega\Vert} \newcommand{\Umy}{U_{\mu_i,y^i}}
\newcommand{\lamns}{\lambda_n^{^{\scriptstyle\sigma}}}
\newcommand{\chiomn}{\chi_{_{\Omega_n}}}
\newcommand{\ullim}{\underline{\lim}} \newcommand{\bsy}{\boldsymbol}
\newcommand{\mvb}{\mathversion{bold}} \newcommand{\la}{\lambda}
\newcommand{\La}{\Lambda} \newcommand{\va}{\varepsilon}
\newcommand{\be}{\beta} \newcommand{\al}{\alpha}
\newcommand{\dis}{\displaystyle} \newcommand{\R}{{\mathbb R}}
\newcommand{\N}{{\mathbb N}} \newcommand{\cF}{{\mathcal F}}
\newcommand{\gB}{{\mathfrak B}} \newcommand{\eps}{\epsilon}
\renewcommand\refname{参考文献} \def \qed {\hfill \vrule height6pt
  width 6pt depth 0pt} \topmargin -0.40in \oddsidemargin 0.08in
\evensidemargin 0.08in \marginparwidth0.00in \marginparsep 0.00in
\textwidth 15.5cm \textheight 24cm \pagestyle{myheadings}
\markboth{\rm \centerline{}} {\rm \centerline{}}
\begin{document}
\title{\huge{\textbf{连续函数的分类}}} \author{\small{叶卢
    庆\footnote{叶卢庆(1992---),男,杭州师范大学理学院数学与应用数学专业
      本科在读,E-mail:h5411167@gmail.com}}\\{\small{杭州师范大学理学院,浙
      江~杭州~310036}}} \date{}
\maketitle
  
% ----------------------------------------------------------------------------------------
% ABSTRACT AND KEYWORDS
% ----------------------------------------------------------------------------------------


\textbf{\small{摘要}:}   \smallskip

\textbf{\small{关键词}:} \smallskip

\textbf{\small{中图分类号}:}
  
\vspace{30pt} % Some vertical space between the abstract and first section
  
% ----------------------------------------------------------------------------------------
% ESSAY BODY
% ----------------------------------------------------------------------------------------
众所周知,$\mathbf{R}\to \mathbf{R}$ 的函数 $f$ 在点 $x_0$ 处连续,意味着
对于任意给定的 $\va>0$,都存在相应的 $\delta>0$,使得 $\forall x\in
(x_0-\delta,x_0+\delta)$,都有 $|f(x)-f(x_0)|<\va$.然而这个定义并没有
对$f$ 在 $x_0$ 处的收敛速度进行刻画.粗糙地说,对于任意给定的 $\va>0$,如
果$\delta$ 要比 $\va$ 小得多,才能使得 $|f(x)-f(x_0)|<\va$,我们就说函
数$f$ 在 $x_0$ 处收敛地比较慢.
% BIBLIOGRAPHY
% ----------------------------------------------------------------------------------------
%
\begin{thebibliography}{}


  % \small{
%    
  % \bibitem[2]{rudin}Walter Rudin.数学分析原理[M].赵慈庚,蒋铎,译.原书
  %   第3版.北京:机械工业出版社,2004:202

  % \bibitem[3]{apostol}Tom M.Apostol .数学分析[M].邢富冲,邢辰,李松
  %   洁,贾婉丽,译.原书第2版.北京:机械工业出版社,2006:302-303 }
\end{thebibliography}
% ----------------------------------------------------------------------------------------
\centering\title{{\textbf{{A proof of the inverse function theorem}}}}
\bigskip\\\author{\small{Luqing Ye}\\{\small{College of
      Science,Hangzhou Normal University,Hangzhou~310036,China}}}
\maketitle
\begin{abstract}

\end{abstract}
\textbf{\small{Keywords}:}
\end{document}