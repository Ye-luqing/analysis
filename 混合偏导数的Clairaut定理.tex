\documentclass[a4paper]{article}
\usepackage{amsmath,amsfonts,amsthm,amssymb}
\usepackage{bm}
\usepackage{hyperref}
\usepackage{geometry,euler}
\usepackage{yhmath}
\usepackage{pstricks-add}
\usepackage{framed,mdframed}
\usepackage{graphicx,color} 
\usepackage{mathrsfs,xcolor} 
\usepackage[all]{xy}
\usepackage{fancybox} 
\usepackage{xeCJK}
\newtheorem*{theo}{定理}
\newtheorem*{exe}{题目}
\newtheorem*{rem}{评论}
\newmdtheoremenv{lemma}{引理}
\newmdtheoremenv{corollary}{推论}
\newtheorem*{exa}{例}
\newenvironment{theorem}
{\bigskip\begin{mdframed}\begin{theo}}
    {\end{theo}\end{mdframed}\bigskip}
\newenvironment{exercise}
{\bigskip\begin{mdframed}\begin{exe}}
    {\end{exe}\end{mdframed}\bigskip}
\newenvironment{example}
{\bigskip\begin{mdframed}\begin{exa}}
    {\end{exa}\end{mdframed}\bigskip}
\geometry{left=2.5cm,right=2.5cm,top=2.5cm,bottom=2.5cm}
\setCJKmainfont[BoldFont=SimHei]{SimSun}
\renewcommand{\today}{\number\year 年 \number\month 月 \number\day 日}
\newcommand{\D}{\displaystyle}\newcommand{\ri}{\Rightarrow}
\newcommand{\ds}{\displaystyle} \renewcommand{\ni}{\noindent}
\newcommand{\ov}{\overrightarrow}
\newcommand{\pa}{\partial} \newcommand{\Om}{\Omega}
\newcommand{\om}{\omega} \newcommand{\sik}{\sum_{i=1}^k}
\newcommand{\vov}{\Vert\omega\Vert} \newcommand{\Umy}{U_{\mu_i,y^i}}
\newcommand{\lamns}{\lambda_n^{^{\scriptstyle\sigma}}}
\newcommand{\chiomn}{\chi_{_{\Omega_n}}}
\newcommand{\ullim}{\underline{\lim}} \newcommand{\bsy}{\boldsymbol}
\newcommand{\mvb}{\mathversion{bold}} \newcommand{\la}{\lambda}
\newcommand{\La}{\Lambda} \newcommand{\va}{\varepsilon}
\newcommand{\be}{\beta} \newcommand{\al}{\alpha}
\newcommand{\dis}{\displaystyle} \newcommand{\R}{{\mathbb R}}
\newcommand{\N}{{\mathbb N}} \newcommand{\cF}{{\mathcal F}}
\newcommand{\gB}{{\mathfrak B}} \newcommand{\eps}{\epsilon}
\renewcommand\refname{参考文献}\renewcommand\figurename{图}
\usepackage[]{caption2} 
\renewcommand{\captionlabeldelim}{}
\setlength\parindent{0pt}
\begin{document}
\title{\huge{\bf{混合偏导数的Clairaut定理}}}
\author{\small{叶卢庆\footnote{叶卢庆(1992---),男,杭州师范大学理学院数
      学与应用数学专业本科在读,E-mail:yeluqingmathematics@gmail.com}}}
\maketitle
混合偏导数的Clairaut定理叙述如下:
\begin{theorem}
  设$E$是$\mathbf{R}^n$的开子集
  合,并设$f:\mathbf{E}\to\mathbf{R}^{m}$是$E$上的二次连续可微函数.那么
  对于一切$x_0\in E$和$1\leq i,j\leq n$,
$$
\frac{\partial }{\partial x_j}\frac{\partial f}{\partial x_i}(x_0)=
\frac{\partial }{\partial x_i}\frac{\partial f}{\partial x_j}(x_0).
$$
\end{theorem}
\begin{proof}[\textbf{证明}]
不妨设 $j<i$.设 $x_0$ 在 $\mathbf{R}^n$ 中的坐标为
$(a_1,a_2,\cdots,a_n)$.我们主要来看表达式{\tiny
\begin{equation}\label{eq:1}
\frac{\frac{f(a_1,\cdots,a_j+\Delta x_j,\cdots,a_i+\Delta
          x_i,\cdots,a_n)-f(a_1,\cdots,a_j+\Delta
          x_j,\cdots,a_i,\cdots,a_n)}{\Delta
          x_i}-\frac{f(a_1,\cdots,a_j,\cdots,a_i+\Delta
          x_i,\cdots,a_n)-f(a_1,\cdots,a_i,\cdots,a_n)}{\Delta
          x_i}}{\Delta x_j}.
\end{equation}}
如果先让$\Delta x_i$趋于$0$,再让$\Delta x_j$趋于$0$,得到的结果是
$\frac{\pa}{\pa x_j}\frac{\pa f}{\pa x_i}$.如果先让$\Delta x_j$趋于
$0$,再让$\Delta x_i$趋于$0$,得到的结果会是$\frac{\pa}{\pa
  x_i}\frac{\pa f}{\pa x_j}$.我们来证明后者等于前者.这是因为,当$\Delta
x_i$固定,而让$\Delta x_j$趋于$0$时,根据Lagrange中值定理,存在位于
$a_i+\Delta x_i$和$a_i$之间的数$a_i'$,使得式\eqref{eq:1}等于
\begin{equation}
  \label{eq:2}
  \frac{\frac{\pa f}{\pa x_i}(a_1,\cdots,a_j+\Delta
    x_j,\cdots,a_i',\cdots,a_n)-\frac{\pa f}{\pa
      x_i}(a_1,\cdots,a_j,\cdots,a_i',\cdots,a_n)}{\Delta x_j}.
\end{equation}
再次使用Lagrange中值定理,可得存在位于$a_{j}$和$a_{j}+\Delta_j$之间的数,使
得式\eqref{eq:2}等于
\begin{equation}
  \label{eq:3}
  \frac{\pa f}{\pa x_j}\frac{\pa f}{\pa x_i}(a_1,\cdots,a_j',\cdots,a_i',\cdots,a_n).
\end{equation}
由于二阶偏导数连续,因此当$\Delta x_i,\Delta x_{j}$趋于$0$,即$a_j'$趋于
$a_j$,$a_i'$趋于$a_i$时,式\eqref{eq:3}确实会趋于$\frac{\pa }{\pa
  x_j}\frac{\pa f}{\pa x_i}$.这样就完成了证明.
\end{proof}
\end{document}
