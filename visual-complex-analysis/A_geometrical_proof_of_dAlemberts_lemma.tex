\documentclass{amsart}

\usepackage{amsmath}
\usepackage{amsthm}
\usepackage{amsfonts}
\usepackage{amssymb}


% \newcommand\sign{\hbox{sign}}

%\subjclass{15A52}

\def\x{{\bf X}}
\def\y{{\bf Y}}
% \def\I#1{{\bf 1}_{#1}}
\def\mb{\mbox}

% \swapnumbers
% \pagestyle{headings}
\parindent = 5 pt
\parskip = 12 pt

\theoremstyle{plain}
\newtheorem{theorem}{Theorem}
\newtheorem{conjecture}[theorem]{Conjecture}
\newtheorem{problem}[theorem]{Problem}
\newtheorem{assumption}[theorem]{Assumption}
\newtheorem{heuristic}[theorem]{Heuristic}
\newtheorem{proposition}[theorem]{Proposition}
\newtheorem{fact}[theorem]{Fact}
\newtheorem{lemma}[theorem]{Lemma}
\newtheorem{corollary}[theorem]{Corollary}
\newtheorem{claim}[theorem]{Claim}
\newtheorem{question}[theorem]{Question}

\theoremstyle{definition}
\newtheorem{definition}[theorem]{Definition}
\newtheorem{example}[theorem]{Example}
\newtheorem{remark}[theorem]{Remark}

\include{psfig}

\begin{document}

\title{A geometrical proof of d'Alembert's lemma}

\author{Luqing Ye}
\address{College of Science, Hangzhou Normal University,Hangzhou City,Zhejiang Province,China}
\email{yeluqingmathematics@gmail.com}
% \thanks{}

\maketitle

\setcounter{tocdepth}{2}
% \tableofcontents


d'Alembert's lemma is stated as follows:
\begin{lemma}[d'Alembert's lemma\cite{JH}]
  If $p(z)$ is a nonconstant polynomial function and $p(z_0)\neq
  0$,then any neighborhood of $z_0$ contains a point $z_1$ such that $|p(z_1)|<|p(z_0)|$.
\end{lemma}

A proof of this lemma can be found in \cite{JH}.



\begin{thebibliography}{1}
\bibitem{JH}John Stillwell.Mathematics and Its History[M]. third edition. New York:Springer, 2010:287-294.
\end{thebibliography}
\end{document}