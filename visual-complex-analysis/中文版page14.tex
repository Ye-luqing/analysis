\documentclass[a4paper]{article} 
\usepackage{amsmath,amsfonts,bm}
\usepackage{hyperref}
\usepackage{amsthm} 
\usepackage{geometry}
\usepackage{amssymb}
\usepackage{pstricks-add}
\usepackage{framed,mdframed}
\usepackage{graphicx,color} 
\usepackage{mathrsfs,xcolor} 
\usepackage[all]{xy}
\usepackage{fancybox} 
% \usepackage{CJKutf8}
\usepackage{xeCJK}
\newtheorem{theorem}{定理}
\newtheorem{lemma}{引理}
\newtheorem{corollary}{推论}
\newtheorem*{exercise}{习题}
\newtheorem{example}{例}
\geometry{left=2.5cm,right=2.5cm,top=2.5cm,bottom=2.5cm}
\setCJKmainfont[BoldFont=Adobe Heiti Std R]{Adobe Song Std L}
\renewcommand{\today}{\number\year 年 \number\month 月 \number\day 日}
\newcommand{\D}{\displaystyle}
\newcommand{\ds}{\displaystyle} \renewcommand{\ni}{\noindent}
\newcommand{\pa}{\partial} \newcommand{\Om}{\Omega}
\newcommand{\om}{\omega} \newcommand{\sik}{\sum_{i=1}^k}
\newcommand{\vov}{\Vert\omega\Vert} \newcommand{\Umy}{U_{\mu_i,y^i}}
\newcommand{\lamns}{\lambda_n^{^{\scriptstyle\sigma}}}
\newcommand{\chiomn}{\chi_{_{\Omega_n}}}
\newcommand{\ullim}{\underline{\lim}} \newcommand{\bsy}{\boldsymbol}
\newcommand{\mvb}{\mathversion{bold}} \newcommand{\la}{\lambda}
\newcommand{\La}{\Lambda} \newcommand{\va}{\varepsilon}
\newcommand{\be}{\beta} \newcommand{\al}{\alpha}
\newcommand{\dis}{\displaystyle} \newcommand{\R}{{\mathbb R}}
\newcommand{\N}{{\mathbb N}} \newcommand{\cF}{{\mathcal F}}
\newcommand{\gB}{{\mathfrak B}} \newcommand{\eps}{\epsilon}
\renewcommand\refname{参考文献}
\begin{document}
\title{\huge{\bf{用复数来证明一个几何题}}} \author{\small{叶卢
    庆\footnote{叶卢庆(1992---),男,杭州师范大学理学院数学与应用数学专业
      本科在读,E-mail:h5411167@gmail.com}}\\{\small{杭州师范大学理学院,浙
      江~杭州~310036}}}
\maketitle
如图,$ABCD$ 是任意四边形.在四边形的四条边上各作一正方形.证明连接相对正
方形中心的线段互相垂直且等长.\\
\newrgbcolor{zzttqq}{0.6 0.2 0}
\psset{xunit=1.0cm,yunit=1.0cm,algebraic=true,dotstyle=o,dotsize=3pt 0,linewidth=0.8pt,arrowsize=3pt 2,arrowinset=0.25}
\begin{pspicture*}(-3.55,-9.13)(26.62,8.33)
\pspolygon[linecolor=zzttqq,fillcolor=zzttqq,fillstyle=solid,opacity=0.1](3.86,1.94)(7.92,1.38)(8.24,-2.36)(3.16,-3.3)
\pspolygon[linecolor=zzttqq,fillcolor=zzttqq,fillstyle=solid,opacity=0.1](3.86,1.94)(7.92,1.38)(8.48,5.44)(4.42,6)
\pspolygon[linecolor=zzttqq,fillcolor=zzttqq,fillstyle=solid,opacity=0.1](7.92,1.38)(8.24,-2.36)(11.98,-2.04)(11.66,1.7)
\pspolygon[linecolor=zzttqq,fillcolor=zzttqq,fillstyle=solid,opacity=0.1](8.24,-2.36)(3.16,-3.3)(4.1,-8.38)(9.18,-7.44)
\pspolygon[linecolor=zzttqq,fillcolor=zzttqq,fillstyle=solid,opacity=0.1](3.16,-3.3)(3.86,1.94)(-1.38,2.64)(-2.08,-2.6)
\psline[linecolor=zzttqq](3.86,1.94)(7.92,1.38)
\psline[linecolor=zzttqq](7.92,1.38)(8.24,-2.36)
\psline[linecolor=zzttqq](8.24,-2.36)(3.16,-3.3)
\psline[linecolor=zzttqq](3.16,-3.3)(3.86,1.94)
\psline[linecolor=zzttqq](3.86,1.94)(7.92,1.38)
\psline[linecolor=zzttqq](7.92,1.38)(8.48,5.44)
\psline[linecolor=zzttqq](8.48,5.44)(4.42,6)
\psline[linecolor=zzttqq](4.42,6)(3.86,1.94)
\psline[linecolor=zzttqq](7.92,1.38)(8.24,-2.36)
\psline[linecolor=zzttqq](8.24,-2.36)(11.98,-2.04)
\psline[linecolor=zzttqq](11.98,-2.04)(11.66,1.7)
\psline[linecolor=zzttqq](11.66,1.7)(7.92,1.38)
\psline[linecolor=zzttqq](8.24,-2.36)(3.16,-3.3)
\psline[linecolor=zzttqq](3.16,-3.3)(4.1,-8.38)
\psline[linecolor=zzttqq](4.1,-8.38)(9.18,-7.44)
\psline[linecolor=zzttqq](9.18,-7.44)(8.24,-2.36)
\psline[linecolor=zzttqq](3.16,-3.3)(3.86,1.94)
\psline[linecolor=zzttqq](3.86,1.94)(-1.38,2.64)
\psline[linecolor=zzttqq](-1.38,2.64)(-2.08,-2.6)
\psline[linecolor=zzttqq](-2.08,-2.6)(3.16,-3.3)
\psline(6.17,3.69)(6.17,-5.37)
\psline(9.95,-0.33)(0.89,-0.33)
\begin{scriptsize}
\psdots[dotstyle=*,linecolor=blue](3.86,1.94)
\rput[bl](3.97,2.11){\blue{$A$}}
\psdots[dotstyle=*,linecolor=blue](7.92,1.38)
\rput[bl](8.04,1.57){\blue{$B$}}
\psdots[dotstyle=*,linecolor=blue](8.24,-2.36)
\rput[bl](8.36,-2.19){\blue{$C$}}
\psdots[dotstyle=*,linecolor=blue](3.16,-3.3)
\rput[bl](3.28,-3.14){\blue{$D$}}
\psdots[dotstyle=*,linecolor=darkgray](8.48,5.44)
\rput[bl](8.59,5.61){\darkgray{$E$}}
\psdots[dotstyle=*,linecolor=darkgray](4.42,6)
\rput[bl](4.54,6.18){\darkgray{$F$}}
\psdots[dotstyle=*,linecolor=darkgray](11.98,-2.04)
\rput[bl](12.09,-1.88){\darkgray{$G$}}
\psdots[dotstyle=*,linecolor=darkgray](11.66,1.7)
\rput[bl](11.77,1.88){\darkgray{$H$}}
\psdots[dotstyle=*,linecolor=darkgray](4.1,-8.38)
\rput[bl](4.23,-8.21){\darkgray{$I$}}
\psdots[dotstyle=*,linecolor=darkgray](9.18,-7.44)
\rput[bl](9.3,-7.27){\darkgray{$J$}}
\psdots[dotstyle=*,linecolor=darkgray](-1.38,2.64)
\rput[bl](-1.28,2.8){\darkgray{$K$}}
\psdots[dotstyle=*,linecolor=darkgray](-2.08,-2.6)
\rput[bl](-1.97,-2.42){\darkgray{$L$}}
\psdots[dotstyle=*,linecolor=darkgray](6.17,3.69)
\rput[bl](6.29,3.86){\darkgray{$M$}}
\psdots[dotstyle=*,linecolor=darkgray](9.95,-0.33)
\rput[bl](10.08,-0.16){\darkgray{$N$}}
\psdots[dotstyle=*,linecolor=darkgray](6.17,-5.37)
\rput[bl](6.29,-5.2){\darkgray{$O$}}
\psdots[dotstyle=*,linecolor=darkgray](0.89,-0.33)
\rput[bl](1.02,-0.16){\darkgray{$P$}}
\end{scriptsize}
\end{pspicture*}
\begin{proof}[\textbf{证明}]
设点 $A,B,C,D$ 对应的复数分别为 $a,b,c,d$.设点 $M,N,O,P$ 对应的复数分
别为 $m,n,o,p$,设点 $F$ 对应的复数为$f$,点 $E$ 对应的复数为 $e$,则有
$$
f-a=i(b-a)\Rightarrow f=ib+(1-i)a.
$$
$$
e-b=i(b-a)\Rightarrow e=-ia+(1+i)b.
$$
于是
$$
m=\frac{1}{2}(a+b-i(a-b)).
$$
同理,
$$
n=\frac{1}{2}(b+c-i(b-c)).
$$
$$
o=\frac{1}{2}(c+d-i(c-d)).
$$
$$
p=\frac{1}{2}(d+a-i(d-a)).
$$
我们只用证明
$$
(m-o)i=p-n,
$$
即证明
$$
\frac{1}{2}((a+b-c-d)i-(c-d-a+b))=p-n,
$$
这是显然的.
\end{proof}
\end{document}








$$
\begin{cases}
\sqrt{2}(  m-a)=(b-a)\omega_{8},\\
\sqrt{2}(n-b)=(c-b)\omega_{8},\\
\sqrt{2}(o-c)=(d-c)\omega_{8},\\
\sqrt{2}(p-d)=(a-b)\omega_{8}.
\end{cases}
$$
为了证明 $MO$ 与 $PN$ 垂直且等长,只用证明
$$
(m-o)\omega_8^{2}=p-n.
$$
也即证明
$$
(a+\frac{1}{\sqrt{2}}(b-a)\sqrt{i}-c-\frac{1}{\sqrt{2}}(d-c)\sqrt{i})i=d+\frac{1}{\sqrt{2}}(a-b)\sqrt{i}-b-\frac{1}{\sqrt{2}}(c-b)\sqrt{i}.
$$
这是容易验证的(为什么?).