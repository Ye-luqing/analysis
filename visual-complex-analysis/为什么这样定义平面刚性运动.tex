\documentclass[a4paper]{article} 
\usepackage{amsmath,amsfonts,bm}
\usepackage{hyperref}
\usepackage{amsthm} 
\usepackage{geometry}
\usepackage{amssymb}
\usepackage{pstricks-add}
\usepackage{framed,mdframed}
\usepackage{graphicx,color} 
\usepackage{mathrsfs,xcolor} 
\usepackage[all]{xy}
\usepackage{fancybox} 
\usepackage{xeCJK}
\newtheorem{theorem}{定理}
\newtheorem{lemma}{引理}
\newtheorem{corollary}{推论}
\newtheorem{definition}{定义}
\newtheorem*{exercise}{习题}
\newtheorem{example}{例}
\geometry{left=2.5cm,right=2.5cm,top=2.5cm,bottom=2.5cm}
\setCJKmainfont[BoldFont=Adobe Heiti Std R]{Adobe Song Std L}
\renewcommand{\today}{\number\year 年 \number\month 月 \number\day 日}
\newcommand{\D}{\displaystyle}
\newcommand{\ds}{\displaystyle} \renewcommand{\ni}{\noindent}
\newcommand{\pa}{\partial} \newcommand{\Om}{\Omega}
\newcommand{\om}{\omega} \newcommand{\sik}{\sum_{i=1}^k}
\newcommand{\vov}{\Vert\omega\Vert} \newcommand{\Umy}{U_{\mu_i,y^i}}
\newcommand{\lamns}{\lambda_n^{^{\scriptstyle\sigma}}}
\newcommand{\chiomn}{\chi_{_{\Omega_n}}}
\newcommand{\ullim}{\underline{\lim}} \newcommand{\bsy}{\boldsymbol}
\newcommand{\mvb}{\mathversion{bold}} \newcommand{\la}{\lambda}
\newcommand{\La}{\Lambda} \newcommand{\va}{\varepsilon}
\newcommand{\be}{\beta} \newcommand{\al}{\alpha}
\newcommand{\dis}{\displaystyle} \newcommand{\R}{{\mathbb R}}
\newcommand{\N}{{\mathbb N}} \newcommand{\cF}{{\mathcal F}}
\newcommand{\gB}{{\mathfrak B}} \newcommand{\eps}{\epsilon}
\renewcommand\refname{参考文献}
\begin{document}
\title{\huge{\bf{为什么要这样定义平面刚性运动}}} \author{\small{叶卢
    庆\footnote{叶卢庆(1992---),男,杭州师范大学理学院数学与应用数学专业
      本科在读,E-mail:h5411167@gmail.com}}\\{\small{杭州师范大学理学院,浙
      江~杭州~310036}}}
\maketitle
平面刚性运动定义为
\begin{definition}
如果存在一个平面上的运动 $\mathcal{M}$,使得 $F'=\mathcal{M}(F)$,就说
$F$ 全等于 $F'$.其中 $\mathcal{M}$ 是平面到自身的映射,使得平面上任意两
点 $A$ 与 $B$ 的距离和 $\mathcal{M}(A)$ 与 $\mathcal{M}(B)$ 的距离相等.
\end{definition}

下面我们论述平面刚性运动和我们对于全等的直观印象符合,对于平面上的两个
点 $A_1,A_2$ 来说,若 $|A_1A_2|=|\mathcal{M}(A_1)\mathcal{M}(A_2)|$,则
根据我们的直观,线段 $\mathcal{M}(A_1)\mathcal{M}(A_2)$ 经过平移和绕着
某个点旋转后,会和线段 $A_1A_2$ 重合.\\

对于平面上的三个点 $A_1A_2A_3$来说,若
$|A_1A_2|=|\mathcal{M}(A_1)\mathcal{M}(A_2)|$,$|A_2A_3|=|\mathcal{M}(A_2)\mathcal{M}(A_3)|$,$|A_3A_1|=|\mathcal{M}(A_3)\mathcal{M}(A_1)|$,
则根据初中的全等三角形的知识,结合直观,可得三角形
$\mathcal{M}(A_1)\mathcal{M}(A_2)\mathcal{M}(A_3)$ 经过平移,翻转,和绕着某个
点旋转后会与三角形 $A_1A_2A_3$ 重合.\\

下面我们证明,如果平面上的四个点,两两之间的距离已经固定,那么这四个点的
相对位置也已经固定.这是因为,如图,\\
\psset{xunit=1.0cm,yunit=1.0cm,algebraic=true,dotstyle=o,dotsize=3pt 0,linewidth=0.8pt,arrowsize=3pt 2,arrowinset=0.25}
\begin{pspicture*}(-4.62,-5.88)(20.7,6.3)
\psline(3.04,3.84)(1.76,0.04)
\psline(1.76,0.04)(9.46,-0.04)
\psline(9.46,-0.04)(3.04,3.84)
\pscircle(1.76,0.04){5.57}
\pscircle(9.46,-0.04){3.96}
\psline(3.04,3.84)(6.63,2.73)
\psline(6.63,2.73)(6.58,-2.75)
\psline(3.04,3.84)(6.61,-0.01)
\psline(3.04,3.84)(6.58,-2.75)
\begin{scriptsize}
\psdots[dotstyle=*,linecolor=blue](1.76,0.04)
\rput[bl](1.84,0.16){\blue{$A$}}
\psdots[dotstyle=*,linecolor=blue](9.46,-0.04)
\rput[bl](9.54,0.08){\blue{$B$}}
\psdots[dotstyle=*,linecolor=blue](3.04,3.84)
\rput[bl](3.12,3.96){\blue{$C$}}
\psdots[dotstyle=*,linecolor=darkgray](6.63,2.73)
\rput[bl](6.72,2.84){\darkgray{$F$}}
\psdots[dotstyle=*,linecolor=darkgray](6.58,-2.75)
\rput[bl](6.66,-2.62){\darkgray{$G$}}
\psdots[dotstyle=*,linecolor=darkgray](6.61,-0.01)
\rput[bl](6.68,0.1){\darkgray{$H$}}
\end{scriptsize}
\end{pspicture*}
首先,我们已经证明,$A,B,C$ 这三个点的相对位置已经固定.对于剩下的一个点
来说,只可能是 $F$ 或 $G$ 这两种选择.易得 $|CG|\neq |CF|$,可见两种选择
不能共存.证明完毕.
\end{document}








