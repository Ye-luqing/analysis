\documentclass[a4paper]{article} 
\usepackage{amsmath,amsfonts,bm}
\usepackage{hyperref}
\usepackage{amsthm} 
\usepackage{geometry}
\usepackage{amssymb}
\usepackage{pstricks-add}
\usepackage{framed,mdframed}
\usepackage{graphicx,color} 
\usepackage{mathrsfs,xcolor} 
\usepackage[all]{xy}
\usepackage{fancybox} 
\usepackage{xeCJK}
\newtheorem*{theorem}{定理}
\newtheorem*{lemma}{引理}
\newtheorem*{corollary}{推论}
\newtheorem*{exercise}{习题}
\newtheorem*{example}{例}
\geometry{left=2.5cm,right=2.5cm,top=2.5cm,bottom=2.5cm}
\setCJKmainfont[BoldFont=Adobe Heiti Std R]{Adobe Song Std L}
\renewcommand{\today}{\number\year 年 \number\month 月 \number\day 日}
\newcommand{\D}{\displaystyle}\newcommand{\ri}{\Rightarrow}
\newcommand{\ds}{\displaystyle} \renewcommand{\ni}{\noindent}
\newcommand{\pa}{\partial} \newcommand{\Om}{\Omega}
\newcommand{\om}{\omega} \newcommand{\sik}{\sum_{i=1}^k}
\newcommand{\vov}{\Vert\omega\Vert} \newcommand{\Umy}{U_{\mu_i,y^i}}
\newcommand{\lamns}{\lambda_n^{^{\scriptstyle\sigma}}}
\newcommand{\chiomn}{\chi_{_{\Omega_n}}}
\newcommand{\ullim}{\underline{\lim}} \newcommand{\bsy}{\boldsymbol}
\newcommand{\mvb}{\mathversion{bold}} \newcommand{\la}{\lambda}
\newcommand{\La}{\Lambda} \newcommand{\va}{\varepsilon}
\newcommand{\be}{\beta} \newcommand{\al}{\alpha}
\newcommand{\dis}{\displaystyle} \newcommand{\R}{{\mathbb R}}
\newcommand{\N}{{\mathbb N}} \newcommand{\cF}{{\mathcal F}}
\newcommand{\gB}{{\mathfrak B}} \newcommand{\eps}{\epsilon}
\renewcommand\refname{参考文献}
\begin{document}
\title{\huge{\bf{习题2.9.1}}} \author{\small{叶卢
    庆\footnote{叶卢庆(1992---),男,杭州师范大学理学院数学与应用数学专业
      本科在读,E-mail:h5411167@gmail.com}}\\{\small{杭州师范大学理学院,浙
      江~杭州~310036}}}
\maketitle
\begin{exercise}[2.9.1]
画出圆周 $|z-1|=1$,用几何方法求出此圆周在映射 $z\to z^2$ 下象的极坐标
方程.画出象曲线,它称为心脏线.  
\end{exercise}
\begin{proof}[\textbf{解}]
如图.易得圆周的极坐标方程为
$$
\rho=2\cos\theta.
$$
在映射 $z\to z^2$ 的作用下,圆周上的点 $(\rho,\theta)$ 变为
$(\rho',\theta')=(\rho^2,2\theta)$.因此可得
$$
4(\cos\frac{\theta'}{2})^{2}=\rho' \iff \rho'=2\cos\theta'+2.
$$
\newrgbcolor{xdxdff}{0.49 0.49 1}
\psset{xunit=5.0cm,yunit=5.0cm,algebraic=true,dotstyle=o,dotsize=3pt 0,linewidth=0.8pt,arrowsize=3pt 2,arrowinset=0.25}
\begin{pspicture*}(-1,-1.44)(3.68,1.47)
\psaxes[labelFontSize=\scriptstyle,xAxis=true,yAxis=true,Dx=0.2,Dy=0.2,ticksize=-2pt 0,subticks=2]{->}(0,0)(-2.84,-1.44)(3.68,1.47)
\pscircle(1,0){5}
\psline(1.41,0.91)(0,0)
\psline(1.41,0.91)(1,0)
\begin{scriptsize}
\psdots[dotstyle=*,linecolor=xdxdff](1,0)
\rput[bl](1.02,0.03){\xdxdff{$A$}}
\psdots[dotstyle=*,linecolor=darkgray](0,0)
\rput[bl](0.02,0.03){\darkgray{$B$}}
\psdots[dotstyle=*,linecolor=xdxdff](1.41,0.91)
\rput[bl](1.42,0.94){\xdxdff{$C$}}
\end{scriptsize}
\end{pspicture*}
\end{proof}
\end{document}








