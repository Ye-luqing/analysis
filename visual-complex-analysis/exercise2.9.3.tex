\documentclass[a4paper]{article} 
\usepackage{amsmath,amsfonts,bm}
\usepackage{hyperref}
\usepackage{amsthm} 
\usepackage{geometry}
\usepackage{amssymb}
\usepackage{pstricks-add}
\usepackage{framed,mdframed}
\usepackage{graphicx,color} 
\usepackage{mathrsfs,xcolor} 
\usepackage[all]{xy}
\usepackage{fancybox} 
\usepackage{xeCJK}
\newtheorem*{theorem}{定理}
\newtheorem*{lemma}{引理}
\newtheorem*{corollary}{推论}
\newtheorem*{exercise}{习题}
\newtheorem*{example}{例}
\geometry{left=2.5cm,right=2.5cm,top=2.5cm,bottom=2.5cm}
\setCJKmainfont[BoldFont=Adobe Heiti Std R]{Adobe Song Std L}
\renewcommand{\today}{\number\year 年 \number\month 月 \number\day 日}
\newcommand{\D}{\displaystyle}\newcommand{\ri}{\Rightarrow}
\newcommand{\ds}{\displaystyle} \renewcommand{\ni}{\noindent}
\newcommand{\pa}{\partial} \newcommand{\Om}{\Omega}
\newcommand{\om}{\omega} \newcommand{\sik}{\sum_{i=1}^k}
\newcommand{\vov}{\Vert\omega\Vert} \newcommand{\Umy}{U_{\mu_i,y^i}}
\newcommand{\lamns}{\lambda_n^{^{\scriptstyle\sigma}}}
\newcommand{\chiomn}{\chi_{_{\Omega_n}}}
\newcommand{\ullim}{\underline{\lim}} \newcommand{\bsy}{\boldsymbol}
\newcommand{\mvb}{\mathversion{bold}} \newcommand{\la}{\lambda}
\newcommand{\La}{\Lambda} \newcommand{\va}{\varepsilon}
\newcommand{\be}{\beta} \newcommand{\al}{\alpha}
\newcommand{\dis}{\displaystyle} \newcommand{\R}{{\mathbb R}}
\newcommand{\N}{{\mathbb N}} \newcommand{\cF}{{\mathcal F}}
\newcommand{\gB}{{\mathfrak B}} \newcommand{\eps}{\epsilon}
\renewcommand\refname{参考文献}
\begin{document}
\title{\huge{\bf{习题2.9.3}}} \author{\small{叶卢
    庆\footnote{叶卢庆(1992---),男,杭州师范大学理学院数学与应用数学专业
      本科在读,E-mail:h5411167@gmail.com}}\\{\small{杭州师范大学理学院,浙
      江~杭州~310036}}}
\maketitle
\begin{exercise}[2.9.3]
考虑一族映射
$$
z\to M_a(z)=\frac{z-a}{\overline{a}z-1},
$$
其中 $a$ 是常数.
\begin{enumerate}
\item 证明 $M_a[M_a(z)]=z$.换言之,$M_a$ 是自逆的.

\begin{proof}[\textbf{证明}]
  \begin{align*}
    M_a[M_a(z)]&=\frac{\frac{z-a}{\overline{a}z-1}-a}{\overline{a}\frac{z-a}{\overline{a}z-1}-1}\\&=\frac{z-a-a(\overline{a}z-1)}{\overline{a}(z-a)-(\overline{a}z-1)}\\&=\frac{z-|a|^2z}{1-|a|^2}\\&=z.
  \end{align*}
\end{proof}
\item 证明 $M_a(z)$ 映单位圆周为其自身.
  \begin{proof}[\textbf{证明}]
我们来计算
\begin{align*}
  \left| \frac{z-a}{\overline{a}z-1}\right|^{2}&=\frac{|z-a|^{2}}{|\overline{a}z-1|^{2}}\\&=\frac{(z-a)(\overline{z}-\overline{a})}{(\overline{a}z-1)(a\overline{z}-1)}\\&=\frac{|z|^2-z\overline{a}-a\overline{z}+|a|^2}{|a|^2|z|^2-\overline{a}z-a\overline{z}+1}.
\end{align*}
当 $|z|=1$ 时,上式变为
$$
\frac{1-z\overline{a}-a\overline{z}+|a|^2}{|a|^2-\overline{a}z-a\overline{z}+1}=1.
$$
命题得证.
  \end{proof}
  \item 证明若 $a$ 位于单位圆盘内,则 $M_a(z)$映单位圆盘为其自身.
  \begin{proof}[\textbf{证明}]
  我们来看
  \begin{align*}
    \left|\frac{z-a}{\overline{a}z-1}\right|^2=\frac{|z|^2-z\overline{a}-a\overline{z}+|a|^2}{|a|^2|z|^2-\overline{a}z-a\overline{z}+1},
  \end{align*}
当 $|a|,|z|\leq 1$ 时,我们来证明
$$
|z|^2-z\overline{a}-a\overline{z}+|a|^2\leq |a|^2|z|^2-\overline{a}z-a\overline{z}+1.
$$
也就是证明
$$
|z|^2(1-|a|^2)\leq 1-|a|^2.
$$
这是显然的.
  \end{proof}
\end{enumerate}
\end{exercise}
\end{document}








