\documentclass[a4paper]{article} 
\usepackage{amsmath,amsfonts,bm}
\usepackage{hyperref}
\usepackage{amsthm} 
\usepackage{geometry}
\usepackage{amssymb}
\usepackage{pstricks-add}
\usepackage{framed,mdframed}
\usepackage{graphicx,color} 
\usepackage{mathrsfs,xcolor} 
\usepackage[all]{xy}
\usepackage{fancybox} 
\usepackage{xeCJK}
\newtheorem*{theorem}{定理}
\newtheorem*{lemma}{引理}
\newtheorem*{corollary}{推论}
\newtheorem*{exercise}{习题}
\newtheorem*{example}{例}
\geometry{left=2.5cm,right=2.5cm,top=2.5cm,bottom=2.5cm}
\setCJKmainfont[BoldFont=Adobe Heiti Std R]{Adobe Song Std L}
\renewcommand{\today}{\number\year 年 \number\month 月 \number\day 日}
\newcommand{\D}{\displaystyle}\newcommand{\ri}{\Rightarrow}
\newcommand{\ds}{\displaystyle} \renewcommand{\ni}{\noindent}
\newcommand{\pa}{\partial} \newcommand{\Om}{\Omega}
\newcommand{\om}{\omega} \newcommand{\sik}{\sum_{i=1}^k}
\newcommand{\vov}{\Vert\omega\Vert} \newcommand{\Umy}{U_{\mu_i,y^i}}
\newcommand{\lamns}{\lambda_n^{^{\scriptstyle\sigma}}}
\newcommand{\chiomn}{\chi_{_{\Omega_n}}}
\newcommand{\ullim}{\underline{\lim}} \newcommand{\bsy}{\boldsymbol}
\newcommand{\mvb}{\mathversion{bold}} \newcommand{\la}{\lambda}
\newcommand{\La}{\Lambda} \newcommand{\va}{\varepsilon}
\newcommand{\be}{\beta} \newcommand{\al}{\alpha}
\newcommand{\dis}{\displaystyle} \newcommand{\R}{{\mathbb R}}
\newcommand{\N}{{\mathbb N}} \newcommand{\cF}{{\mathcal F}}
\newcommand{\gB}{{\mathfrak B}} \newcommand{\eps}{\epsilon}
\renewcommand\refname{参考文献}
\begin{document}
\title{\huge{\bf{封闭不自交的四边形的面积公式}}} \author{\small{叶卢
    庆\footnote{叶卢庆(1992---),男,杭州师范大学理学院数学与应用数学专业
      本科在读,E-mail:h5411167@gmail.com}}\\{\small{杭州师范大学理学院,浙
      江~杭州~310036}}}
\maketitle
设 $A_1A_2A_3A_4A_1$ 是个一封闭的四边形,且该四边形的边除顶点之外互
不相交.下面我们来求其面积表达式.如图,当原点 $O$ 位于四边形的内部或边界
时,显然四边形的面积为
\begin{align*}
  \frac{1}{2}|OB\times OA+OA\times OD+OD\times OC+OC\times OB|
\end{align*}
易得如果原点$O$ 位于四边形外面,上面公式依然成立.\\
\newrgbcolor{zzttqq}{0.6 0.2 0}
\psset{xunit=1.0cm,yunit=1.0cm,algebraic=true,dotstyle=o,dotsize=3pt 0,linewidth=0.8pt,arrowsize=3pt 2,arrowinset=0.25}
\begin{pspicture*}(-1.84,-7.66)(29.74,10.88)
\pspolygon[linecolor=zzttqq,fillcolor=zzttqq,fillstyle=solid,opacity=0.1](2.06,0.92)(5.7,4.98)(10.75,0.8)(5.54,-0.35)
\psline[linecolor=zzttqq](2.06,0.92)(5.7,4.98)
\psline[linecolor=zzttqq](5.7,4.98)(10.75,0.8)
\psline[linecolor=zzttqq](10.75,0.8)(5.54,-0.35)
\psline[linecolor=zzttqq](5.54,-0.35)(2.06,0.92)
\psline{->}(6.67,1.38)(5.7,4.98)
\psline{->}(6.67,1.38)(2.06,0.92)
\psline{->}(6.67,1.38)(10.75,0.8)
\psline{->}(6.67,1.38)(5.54,-0.36)
\psplot{-11.84}{29.74}{(--5.82-1.27*x)/3.48}
\psplot{-11.84}{29.74}{(-8.24--1.16*x)/5.21}
\psplot{-11.84}{29.74}{(-48.92--4.18*x)/-5.05}
\psplot{-11.84}{29.74}{(--5.01-4.06*x)/-3.64}
\begin{scriptsize}
\psdots[dotstyle=*,linecolor=blue](2.06,0.92)
\rput[bl](2.19,1.11){\blue{$A$}}
\psdots[dotstyle=*,linecolor=blue](5.7,4.98)
\rput[bl](5.81,5.16){\blue{$B$}}
\psdots[dotstyle=*,linecolor=blue](10.75,0.8)
\rput[bl](10.87,0.99){\blue{$C$}}
\psdots[dotstyle=*,linecolor=blue](5.54,-0.35)
\rput[bl](5.66,-0.17){\blue{$D$}}
\psdots[dotstyle=*,linecolor=blue](6.67,1.38)
\rput[bl](6.79,1.57){\blue{$O$}}
\end{scriptsize}
\end{pspicture*}
\end{document}








