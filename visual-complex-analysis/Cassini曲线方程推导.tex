\documentclass[a4paper]{article} 
\usepackage{amsmath,amsfonts,bm}
\usepackage{hyperref}
\usepackage{amsthm} 
\usepackage{geometry}
\usepackage{amssymb}
\usepackage{pstricks-add}
\usepackage{framed,mdframed}
\usepackage{graphicx,color} 
\usepackage{mathrsfs,xcolor} 
\usepackage[all]{xy}
\usepackage{fancybox} 
\usepackage{xeCJK}
\newtheorem*{theorem}{定理}
\newtheorem*{lemma}{引理}
\newtheorem*{corollary}{推论}
\newtheorem*{exercise}{习题}
\newtheorem*{example}{例}
\geometry{left=2.5cm,right=2.5cm,top=2.5cm,bottom=2.5cm}
\setCJKmainfont[BoldFont=Adobe Heiti Std R]{Adobe Song Std L}
\renewcommand{\today}{\number\year 年 \number\month 月 \number\day 日}
\newcommand{\D}{\displaystyle}\newcommand{\ri}{\Rightarrow}
\newcommand{\ds}{\displaystyle} \renewcommand{\ni}{\noindent}
\newcommand{\pa}{\partial} \newcommand{\Om}{\Omega}
\newcommand{\om}{\omega} \newcommand{\sik}{\sum_{i=1}^k}
\newcommand{\vov}{\Vert\omega\Vert} \newcommand{\Umy}{U_{\mu_i,y^i}}
\newcommand{\lamns}{\lambda_n^{^{\scriptstyle\sigma}}}
\newcommand{\chiomn}{\chi_{_{\Omega_n}}}
\newcommand{\ullim}{\underline{\lim}} \newcommand{\bsy}{\boldsymbol}
\newcommand{\mvb}{\mathversion{bold}} \newcommand{\la}{\lambda}
\newcommand{\La}{\Lambda} \newcommand{\va}{\varepsilon}
\newcommand{\be}{\beta} \newcommand{\al}{\alpha}
\newcommand{\dis}{\displaystyle} \newcommand{\R}{{\mathbb R}}
\newcommand{\N}{{\mathbb N}} \newcommand{\cF}{{\mathcal F}}
\newcommand{\gB}{{\mathfrak B}} \newcommand{\eps}{\epsilon}
\renewcommand\refname{参考文献}
\begin{document}
\title{\huge{\bf{Cassini曲线方程推导}}} \author{\small{叶卢
    庆\footnote{叶卢庆(1992---),男,杭州师范大学理学院数学与应用数学专业
      本科在读,E-mail:h5411167@gmail.com}}\\{\small{杭州师范大学理学院,浙
      江~杭州~310036}}}
\maketitle
我们设平面直角坐标系上有点 $(a,0)$,$(-a,0)$.下面我们来求到这两个点的距
离乘积为 $k(k>0)$ 的所有点形成的曲线的方程.可得
$$
\sqrt{(x-a)^2+y^2}\sqrt{(x+a)^2+y^2}=k.
$$
于是
$$
[(x-a)^2+y^2][(x+a)^2+y^2]=k^2.
$$
于是
$$
(x^2-a^2)^2+y^2((x-a)^2+(x+a)^2)+y^4=k^2.
$$
于是
$$
(x^2-a^2)^2+2y^2(x^2+a^2)+y^4=k^2.
$$
特别的,当 $k=a^2$ 时,Cassini曲线可以化为 Bernoulli双纽线
$$
x^4-2x^2a^2+2y^2(x^2+a^2)+y^4=0. \iff (x^2+y^2)^2+2a^2(y^2-x^2)=0.
$$
对于Bernoulli双纽线来说,特别的,当 $a=\frac{\sqrt{2}}{2}$ 时,可得
$$
(x^2+y^2)^2=x^2-y^2.
$$
即
$$
x^2+y^2=\sqrt{x^2-y^2}.
$$
化为极坐标方程,令 $x=\rho\cos\theta,y=\rho\sin\theta$,即为
$$
\rho^2=\rho \sqrt{|\cos2\theta|}.$
$
当 $\rho\neq 0$ 时,化为
$$
\rho^{2}=\cos2\theta.
$$

\end{document}








