\documentclass{amsart}

\usepackage{amsmath,color}
\usepackage{amsthm}
\usepackage{amsfonts}
\usepackage{amssymb}



\newcommand\pa{{\partial}}
% \swapnumbers
% \pagestyle{headings}
\parindent = 5 pt
\parskip = 12 pt

\theoremstyle{plain}
\newtheorem{theorem}{Theorem}
\newtheorem{conjecture}[theorem]{Conjecture}
\newtheorem{problem}[theorem]{Problem}
\newtheorem{assumption}[theorem]{Assumption}
\newtheorem{heuristic}[theorem]{Heuristic}
\newtheorem{proposition}[theorem]{Proposition}
\newtheorem{fact}[theorem]{Fact}
\newtheorem{lemma}[theorem]{Lemma}
\newtheorem{corollary}[theorem]{Corollary}
\newtheorem{claim}[theorem]{Claim}
\newtheorem{question}[theorem]{Question}

\theoremstyle{definition}
\newtheorem{definition}[theorem]{Definition}
\newtheorem{example}[theorem]{Example}
\newtheorem{remark}[theorem]{Remark}

\include{psfig}

\begin{document}

\title{foundations for a general theory of functions of a variable
  complex magnitude}

\author{Bernhard Riemann}
%\address{College of Science, Hangzhou Normal University,Hangzhou City,Zhejiang Province,China}
%\email{yeluqingmathematics@gmail.com}
 \thanks{This \LaTeX is typeset and commented in red by
   Luqing Ye,whose email is yeluqingmathematics@gmail.com}


\maketitle

\setcounter{tocdepth}{2}
% \tableofcontents

\section*{1}
\label{sec:1}


If we consider $z$ to be a variable magnitude which can gradually
assume all possible real values,then we call $w$ a function of
$z$,when each of its real values corresponds to a single value of
undetermined magnitude such as $w_0$.If $w$ also constantly change
while $z$ continuously goes through all the value lying between two
fixed values,then we call this function within these intervals a
constant or a continuous function.


Obviously,this definition does not set up any absolute law between the
individual values of the function,because when we assign a determinate
value to this function,the way in which it continues outside of this
interval remains totally arbitrary.

We can express the slope function (dependence) of magnitude $w(z)$ by
a mathematical law so that we can find the corresponding value of $w$
for every value of $z$ through determinate numerical
operations(Groessen operationen).{\color{blue}Previously,people have
  only considered a certain kind of function (functiones continuae
  according to Euler's usage) as having the ability of being able to
  determine all the value of $z$ lying between a given interval by
  using that same slope function law;}\footnote{{\color{red}
    Sentences which I do not understand are typeset in
    blue.}}however,in the meantime,new research has shown that there
are analytic expressions that can represent each and every constant
function for a given interval\footnote{{\color{red} Fourier
    theorem.}}.This holds,regardless of whether the slope function of
magnitude $w$ (magnitude $z$) is conditionally defined as an arbitrary
given numerical operation,or as an determinate numerical operation.As
a result of the theorems mentioned above,both concepts are congruent.

But the situation is different when we do not limit the variability of
magnitude $z$ to real values,but instead allow complex values of the
form $x+yi$(where $i=\sqrt{-1}$).

Assume that $x+yi$ and $x+yi+dx+dyi$ are two infinitesimally slightly
different values for magnitude $z$,which correspond to the values
$u+vi$ and $u+vi+du+dvi$ for magnitude {\color{green}$w$}.So then,if
the slope function of magnitude $w(z)$ is an arbitrarily given
one,then generally speaking,the ratio $\frac{du+dvi}{dx+dyi}$ changes
for the values for $dx,dy$,because when we have $dx+dyi=\varepsilon
e^{\varphi i}$,then
\begin{eqnarray*}
  \frac{du + dv \, i}{dx + dy \, i}
  &=&   \frac{1}{2}
  \left(
    \frac{\partial u}{\partial x}
    + \frac{\partial v}{\partial y}
  \right)
  + \frac{1}{2}
  \left(
    \frac{\partial v}{\partial x}
    - \frac{\partial u}{\partial y}
  \right) i \\
  & & + \frac{1}{2}
  \left[
    \frac{\partial u}{\partial x}
    - \frac{\partial v}{\partial y}
    + \left(
      \frac{\partial v}{\partial x}
      + \frac{\partial u}{\partial y}
    \right) i
  \right]
  \frac{dx - dy \, i}{dx + dy \, i} \\
  &=&   \frac{1}{2}
  \left(
    \frac{\partial u}{\partial x}
    + \frac{\partial v}{\partial y}
  \right)
  + \frac{1}{2}
  \left(
    \frac{\partial v}{\partial x}
    - \frac{\partial u}{\partial y}
  \right) i \\
  & & + \frac{1}{2}
  \left[
    \frac{\partial u}{\partial x}
    - \frac{\partial v}{\partial y}
    + \left(
      \frac{\partial v}{\partial x}
      + \frac{\partial u}{\partial y}
    \right) i
  \right]
  e^{-2 \varphi i}
\end{eqnarray*}
\footnote{{\color{red}
Here we give a modern derivation of the Cauchy-Riemann equations.Let $f(x,y)=(u,v)$,where $f:\mathbf{R}^2\to \mathbf{R}^2$ is
differentiable.So there is a linear map which maps $(dx,dy)$ to $(du,dv)$,i.e,
$$
\begin{pmatrix}
  du\\
dv
\end{pmatrix}=\begin{pmatrix}
  \frac{\partial u}{\partial x}&\frac{\partial u}{\partial y}\\
\frac{\partial v}{\partial x}&\frac{\partial v}{\partial y}
\end{pmatrix}\begin{pmatrix}
  dx\\
dy
\end{pmatrix}.
$$
We also want that 
$$
du+dv i=(c+di) (dx+dy i)+o(dx+dy i),
$$
where  $c+di$ is a complex number,and 
$$
\lim_{dx+dy i\to 0} \frac{o(dx+dy i)}{dx+dy i}=0.
$$
So
$$
\begin{pmatrix}
  du\\
dv
\end{pmatrix}=\begin{pmatrix}
  c&-d\\
d&c\\
\end{pmatrix}\begin{pmatrix}
  dx\\
dy
\end{pmatrix}.
$$
So
$$
\frac{\pa u}{\pa x}=\frac{\pa v}{\pa y}=c,\frac{\pa v}{\pa x}=-\frac{\pa
u}{\pa y}=d.
$$

Riemann's derivation gives us more information than the modern method
we give above.If the Cauchy-Riemann
equations are not satisfied by $\frac{\pa u}{\pa x},\frac{\pa u}{\pa
  y},\frac{\pa v}{\pa x},\frac{\pa v}{\pa y}$,then the linear map from $\mathbf{R}^2$ to $\mathbf{R}^2$ maps a circle
centering at the $(0,0)$ to an ellipse,Riemann's derivation provide us
error term $\frac{1}{2}
  \left[
    \frac{\partial u}{\partial x}
    - \frac{\partial v}{\partial y}
    + \left(
      \frac{\partial v}{\partial x}
      + \frac{\partial u}{\partial y}
    \right) i
  \right]
  e^{-2 \varphi i}$.
  }}
However,regardless of the manner in which we define $w$ as a function
of $z$ through these simple numerical operation,the value of the
differential quotient $\frac{dw}{dz}$ is always independent of the
special values of differential $dz$.Obviously,not every arbitrary
slope function of complex magnitude $w$ (complex magnitude $z$) can be
expressed in this manner.
\end{document}