\documentclass[a4paper]{article} 
\usepackage{amsmath,amsfonts,bm}
\usepackage{hyperref}
\usepackage{amsthm} 
\usepackage{geometry}
\usepackage{amssymb}
\usepackage{pstricks-add}
\usepackage{framed,mdframed}
\usepackage{graphicx,color} 
\usepackage{mathrsfs,xcolor} 
\usepackage[all]{xy}
\usepackage{fancybox} 
% \usepackage{CJKutf8}
\usepackage{xeCJK}
\newtheorem{theorem}{定理}
\newtheorem{lemma}{引理}
\newtheorem{corollary}{推论}
\newtheorem*{exercise}{习题}
\newtheorem{example}{例}
\geometry{left=2.5cm,right=2.5cm,top=2.5cm,bottom=2.5cm}
\setCJKmainfont[BoldFont=Adobe Heiti Std R]{Adobe Song Std L}
\renewcommand{\today}{\number\year 年 \number\month 月 \number\day 日}
\newcommand{\D}{\displaystyle}
\newcommand{\ds}{\displaystyle} \renewcommand{\ni}{\noindent}
\newcommand{\pa}{\partial} \newcommand{\Om}{\Omega}
\newcommand{\om}{\omega} \newcommand{\sik}{\sum_{i=1}^k}
\newcommand{\vov}{\Vert\omega\Vert} \newcommand{\Umy}{U_{\mu_i,y^i}}
\newcommand{\lamns}{\lambda_n^{^{\scriptstyle\sigma}}}
\newcommand{\chiomn}{\chi_{_{\Omega_n}}}
\newcommand{\ullim}{\underline{\lim}} \newcommand{\bsy}{\boldsymbol}
\newcommand{\mvb}{\mathversion{bold}} \newcommand{\la}{\lambda}
\newcommand{\La}{\Lambda} \newcommand{\va}{\varepsilon}
\newcommand{\be}{\beta} \newcommand{\al}{\alpha}
\newcommand{\dis}{\displaystyle} \newcommand{\R}{{\mathbb R}}
\newcommand{\N}{{\mathbb N}} \newcommand{\cF}{{\mathcal F}}
\newcommand{\gB}{{\mathfrak B}} \newcommand{\eps}{\epsilon}
\renewcommand\refname{参考文献}
\begin{document}
\title{\huge{\bf{习题1.5.19}}} \author{\small{叶卢
    庆\footnote{叶卢庆(1992---),男,杭州师范大学理学院数学与应用数学专业
      本科在读,E-mail:h5411167@gmail.com}}\\{\small{杭州师范大学理学院,浙
      江~杭州~310036}}}
\maketitle
\begin{exercise}
三角形 $T$ 的重心 $G$ 是其中线的交点.若顶点 $a,b,c$ 为复数,如图所示.可
以证明 $G=\frac{1}{3}(a+b+c)$.在三角形的各边上,作三个任意形状的相似三
角形,可得顶点为 $JKI$ 的三角形,证明新三角形的重心与老三角形的重心重合.\\
\newrgbcolor{zzttqq}{0.6 0.2 0}
\psset{xunit=1.0cm,yunit=1.0cm,algebraic=true,dotstyle=o,dotsize=3pt 0,linewidth=0.8pt,arrowsize=3pt 2,arrowinset=0.25}
\begin{pspicture*}(-4.98,-6.06)(19.85,5.01)
\pspolygon[linecolor=zzttqq,fillcolor=zzttqq,fillstyle=solid,opacity=0.1](5.22,2.84)(2.34,-0.48)(11.3,-0.86)
\pspolygon[linecolor=zzttqq,fillcolor=zzttqq,fillstyle=solid,opacity=0.1](2.34,-0.48)(5.22,2.84)(3.7,2.94)
\pspolygon[linecolor=zzttqq,fillcolor=zzttqq,fillstyle=solid,opacity=0.1](5.22,2.84)(11.08,0.92)(11.3,-0.86)
\pspolygon[linecolor=zzttqq,fillcolor=zzttqq,fillstyle=solid,opacity=0.1](11.3,-0.86)(4.34,-2.22)(2.34,-0.48)
\psline[linecolor=zzttqq](5.22,2.84)(2.34,-0.48)
\psline[linecolor=zzttqq](2.34,-0.48)(11.3,-0.86)
\psline[linecolor=zzttqq](11.3,-0.86)(5.22,2.84)
\psline[linecolor=zzttqq](2.34,-0.48)(5.22,2.84)
\psline[linecolor=zzttqq](5.22,2.84)(3.7,2.94)
\psline[linecolor=zzttqq](3.7,2.94)(2.34,-0.48)
\psline[linecolor=zzttqq](5.22,2.84)(11.08,0.92)
\psline[linecolor=zzttqq](11.08,0.92)(11.3,-0.86)
\psline[linecolor=zzttqq](11.3,-0.86)(5.22,2.84)
\psline[linecolor=zzttqq](11.3,-0.86)(4.34,-2.22)
\psline[linecolor=zzttqq](4.34,-2.22)(2.34,-0.48)
\psline[linecolor=zzttqq](2.34,-0.48)(11.3,-0.86)
\psline(3.7,2.94)(11.08,0.92)
\psline(11.08,0.92)(4.34,-2.22)
\psline(3.7,2.94)(4.34,-2.22)
\begin{scriptsize}
\psdots[dotstyle=*,linecolor=blue](5.22,2.84)
\rput[bl](5.29,2.96){\blue{$A$}}
\psdots[dotstyle=*,linecolor=blue](2.34,-0.48)
\rput[bl](2.42,-0.37){\blue{$B$}}
\psdots[dotstyle=*,linecolor=blue](11.3,-0.86)
\rput[bl](11.38,-0.75){\blue{$C$}}
\psdots[dotstyle=*,linecolor=darkgray](6.29,0.5)
\rput[bl](6.36,0.61){\darkgray{$G$}}
\psdots[dotstyle=*,linecolor=blue](3.7,2.94)
\rput[bl](3.78,3.05){\blue{$J$}}
\psdots[dotstyle=*,linecolor=blue](11.08,0.92)
\rput[bl](11.15,1.03){\blue{$K$}}
\psdots[dotstyle=*,linecolor=blue](4.34,-2.22)
\rput[bl](4.42,-2.12){\blue{$L$}}
\end{scriptsize}
\end{pspicture*}
\end{exercise}
\begin{proof}[\textbf{证明}]
易得
$$
\frac{J-B}{J-A}=\frac{K-A}{K-C}=\frac{I-C}{I-B}=p.
$$
于是
$$
J=\frac{B-pA}{1-p},
$$
$$
K=\frac{A-pC}{1-p},
$$
$$
I=\frac{C-pB}{1-p}.
$$
于是,
$$
J+K+I=A+B+C,
$$
得证.
\end{proof}
\end{document}








