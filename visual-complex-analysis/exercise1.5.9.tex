\documentclass[a4paper]{article} 
\usepackage{amsmath,amsfonts,bm}
\usepackage{hyperref}
\usepackage{amsthm} 
\usepackage{geometry}
\usepackage{amssymb}
\usepackage{pstricks-add}
\usepackage{framed,mdframed}
\usepackage{graphicx,color} 
\usepackage{mathrsfs,xcolor} 
\usepackage[all]{xy}
\usepackage{fancybox} 
% \usepackage{CJKutf8}
\usepackage{xeCJK}
\newtheorem{theorem}{定理}
\newtheorem{lemma}{引理}
\newtheorem{corollary}{推论}
\newtheorem*{exercise}{习题}
\newtheorem{example}{例}
\geometry{left=2.5cm,right=2.5cm,top=2.5cm,bottom=2.5cm}
\setCJKmainfont[BoldFont=Adobe Heiti Std R]{Adobe Song Std L}
\renewcommand{\today}{\number\year 年 \number\month 月 \number\day 日}
\newcommand{\D}{\displaystyle}
\newcommand{\ds}{\displaystyle} \renewcommand{\ni}{\noindent}
\newcommand{\pa}{\partial} \newcommand{\Om}{\Omega}
\newcommand{\om}{\omega} \newcommand{\sik}{\sum_{i=1}^k}
\newcommand{\vov}{\Vert\omega\Vert} \newcommand{\Umy}{U_{\mu_i,y^i}}
\newcommand{\lamns}{\lambda_n^{^{\scriptstyle\sigma}}}
\newcommand{\chiomn}{\chi_{_{\Omega_n}}}
\newcommand{\ullim}{\underline{\lim}} \newcommand{\bsy}{\boldsymbol}
\newcommand{\mvb}{\mathversion{bold}} \newcommand{\la}{\lambda}
\newcommand{\La}{\Lambda} \newcommand{\va}{\varepsilon}
\newcommand{\be}{\beta} \newcommand{\al}{\alpha}
\newcommand{\dis}{\displaystyle} \newcommand{\R}{{\mathbb R}}
\newcommand{\N}{{\mathbb N}} \newcommand{\cF}{{\mathcal F}}
\newcommand{\gB}{{\mathfrak B}} \newcommand{\eps}{\epsilon}
\renewcommand\refname{参考文献}
\begin{document}
\title{\huge{\bf{习题1.5.9}}} \author{\small{叶卢
    庆\footnote{叶卢庆(1992---),男,杭州师范大学理学院数学与应用数学专业
      本科在读,E-mail:h5411167@gmail.com}}\\{\small{杭州师范大学理学院,浙
      江~杭州~310036}}}
\maketitle
\begin{exercise}
  令 $A,B,C,D$ 为单位圆上四点,且 $A+B+C+D=0$,证明这四点必成一矩形.
\end{exercise}
\begin{proof}[\textbf{证明}]
由于 $A,B,C,D$ 为单位圆上的四点, $A=\cos\theta_A+i\sin\theta_A$,
其余各点类似.则我们有
$$
\begin{cases}
  \cos\theta_A+\cos\theta_B+\cos\theta_C+\cos\theta_D=0,\\
  \sin\theta_A+\sin\theta_B+\sin\theta_C+\sin\theta_D=0.
\end{cases}.
$$
进行和差化积,上面两条等式可以化为
$$
\begin{cases}
  \cos \frac{\theta_A+\theta_C}{2}\cos \frac{\theta_A-\theta_C}{2}+\cos
  \frac{\theta_B+\theta_D}{2}\cos\frac{\theta_B-\theta_D}{2}=0,\\
\sin \frac{\theta_A+\theta_C}{2}\cos \frac{\theta_A-\theta_C}{2}+\sin \frac{\theta_B+\theta_D}{2}\cos \frac{\theta_B-\theta_D}{2}=0.
\end{cases}
$$
将第一条式子乘以 $\sin \frac{\theta_A+\theta_C}{2}$,可得
$$
\cos \frac{\theta_A+\theta_C}{2}\sin \frac{\theta_A+\theta_C}{2}\cos \frac{\theta_A-\theta_C}{2}+\sin
\frac{\theta_A+\theta_C}{2}\cos \frac{\theta_B+\theta_D}{2}\cos \frac{\theta_B-\theta_D}{2}=0,
$$
将第二条式子代进上式,可得
$$
-\sin \frac{\theta_B+\theta_D}{2}\cos \frac{\theta_B-\theta_D}{2}\cos \frac{\theta_A+\theta_C}{2}+\sin
\frac{\theta_A+\theta_C}{2}\cos \frac{\theta_B+\theta_D}{2}\cos \frac{\theta_B-\theta_D}{2}=0.
$$
也即,
$$
\cos \frac{\theta_B-\theta_D}{2}\sin \frac{\theta_A+\theta_C-\theta_B-\theta_D}{2}=0.
$$
不妨设 $\theta_A,\theta_B,\theta_C,\theta_D\in [0,2\pi]$,且 $\theta_A<\theta_B<\theta_C<\theta_D$.则
$\theta_B-\theta_D=-\pi$,或者
$\theta_A+\theta_C-\theta_B-\theta_D=- 2\pi$.如果是
$\theta_B-\theta_D=-\pi$,则根据对称性,必定也有
$\theta_A-\theta_C=-\pi$,此时是矩形.下面我们来证明,
$\theta_A+\theta_C-\theta_B-\theta_D=-2\pi$ 是不可能的,因为$A,B,C,D$
在四个象限都有分布(可以包括坐标轴),$\theta_{A}$ 最小为0,$\theta_C$ 最
小为 $\pi$,$\theta_B$ 最大为 $\pi$($\theta_C,\theta_B$ 由于不能重合,因
此必有一个无法达到 $\pi$),$\theta_D$ 最大为$2\pi$($\theta_A,\theta_D$
不可能一者为 $0$ 一者为 $2\pi$),因此
$\theta_A+\theta_C-\theta_B-\theta_D$ 最小也不可能达到 $-2\pi$,因此不
可能.综上我们已经证明了四个点形成矩形.
\end{proof}
\end{document}












