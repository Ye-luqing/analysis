\documentclass[a4paper]{article} 
\usepackage{amsmath,amsfonts,bm}
\usepackage{hyperref}
\usepackage{amsthm} 
\usepackage{geometry}
\usepackage{amssymb}
\usepackage{pstricks-add}
\usepackage{framed,mdframed}
\usepackage{graphicx,color} 
\usepackage{mathrsfs,xcolor} 
\usepackage[all]{xy}
\usepackage{fancybox} 
\usepackage{xeCJK}
\newtheorem*{theorem}{定理}
\newtheorem*{lemma}{引理}
\newtheorem*{corollary}{推论}
\newtheorem*{exercise}{习题}
\newtheorem*{example}{例}
\geometry{left=2.5cm,right=2.5cm,top=2.5cm,bottom=2.5cm}
\setCJKmainfont[BoldFont=Adobe Heiti Std R]{Adobe Song Std L}
\renewcommand{\today}{\number\year 年 \number\month 月 \number\day 日}
\newcommand{\D}{\displaystyle}\newcommand{\ri}{\Rightarrow}
\newcommand{\ds}{\displaystyle} \renewcommand{\ni}{\noindent}
\newcommand{\pa}{\partial} \newcommand{\Om}{\Omega}
\newcommand{\om}{\omega} \newcommand{\sik}{\sum_{i=1}^k}
\newcommand{\vov}{\Vert\omega\Vert} \newcommand{\Umy}{U_{\mu_i,y^i}}
\newcommand{\lamns}{\lambda_n^{^{\scriptstyle\sigma}}}
\newcommand{\chiomn}{\chi_{_{\Omega_n}}}
\newcommand{\ullim}{\underline{\lim}} \newcommand{\bsy}{\boldsymbol}
\newcommand{\mvb}{\mathversion{bold}} \newcommand{\la}{\lambda}
\newcommand{\La}{\Lambda} \newcommand{\va}{\varepsilon}
\newcommand{\be}{\beta} \newcommand{\al}{\alpha}
\newcommand{\dis}{\displaystyle} \newcommand{\R}{{\mathbb R}}
\newcommand{\N}{{\mathbb N}} \newcommand{\cF}{{\mathcal F}}
\newcommand{\gB}{{\mathfrak B}} \newcommand{\eps}{\epsilon}
\renewcommand\refname{参考文献}
\begin{document}
\title{\huge{\bf{习题2.9.2}}} \author{\small{叶卢
    庆\footnote{叶卢庆(1992---),男,杭州师范大学理学院数学与应用数学专业
      本科在读,E-mail:h5411167@gmail.com}}\\{\small{杭州师范大学理学院,浙
      江~杭州~310036}}}
\maketitle
\begin{exercise}[2.9.2]
考虑复映射 $z\to \omega=\frac{z-a}{z-b}$.用几何方法证明,若将此映射施于
连接 $a,b$ 两点的线段之垂直平分线,则象是单位圆周.较详细地描述当 $z$ 以
匀速沿着此直线运动时 $\omega$ 的运动.  
\end{exercise}
\begin{proof}[\textbf{解}]
只看下图便可知.其中 $A=1,B=-1$.
$$
Z'=\frac{Z-A}{Z-B}.
$$
\newrgbcolor{xdxdff}{0.49 0.49 1}
\psset{xunit=1.0cm,yunit=1.0cm,algebraic=true,dotstyle=o,dotsize=3pt 0,linewidth=0.8pt,arrowsize=3pt 2,arrowinset=0.25}
\begin{pspicture*}(-12.51,-4.12)(13.51,7.48)
\psline{->}(0,0)(9,0)
\psline{->}(0,0)(0,9)
\psline{->}(0,0)(-9,0)
\psline{->}(0,0)(0,-9)
\pscircle(0,0){4}
\psline{->}(-4,0)(0,6.07)
\psline{->}(4,0)(0,6.07)
\psline{->}(0,0)(1.56,3.7)
\begin{scriptsize}
\psdots[dotstyle=*,linecolor=xdxdff](0,0)
\rput[bl](0.08,0.11){\xdxdff{$O$}}
\psdots[dotstyle=*,linecolor=xdxdff](0,6.07)
\rput[bl](0.08,6.18){\xdxdff{$Z$}}
\psdots[dotstyle=*,linecolor=xdxdff](4,0)
\rput[bl](4.08,0.11){\xdxdff{$A$}}
\psdots[dotstyle=*,linecolor=xdxdff](-4,0)
\rput[bl](-3.92,0.11){\xdxdff{$B$}}
\psdots[dotstyle=*,linecolor=xdxdff](1.54,3.67)
\rput[bl](1.62,3.78){\xdxdff{$Z'$}}
\end{scriptsize}
\end{pspicture*}
\end{proof}
\end{document}








