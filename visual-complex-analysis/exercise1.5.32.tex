\documentclass[a4paper]{article} 
\usepackage{amsmath,amsfonts,bm}
\usepackage{hyperref}
\usepackage{amsthm} 
\usepackage{geometry}
\usepackage{amssymb}
\usepackage{pstricks-add}
\usepackage{framed,mdframed}
\usepackage{graphicx,color} 
\usepackage{mathrsfs,xcolor} 
\usepackage[all]{xy}
\usepackage{fancybox} 
\usepackage{xeCJK}
\newtheorem*{theorem}{定理}
\newtheorem*{lemma}{引理}
\newtheorem*{corollary}{推论}
\newtheorem*{exercise}{习题}
\newtheorem*{example}{例}
\geometry{left=2.5cm,right=2.5cm,top=2.5cm,bottom=2.5cm}
\setCJKmainfont[BoldFont=Adobe Heiti Std R]{Adobe Song Std L}
\renewcommand{\today}{\number\year 年 \number\month 月 \number\day 日}
\newcommand{\D}{\displaystyle}\newcommand{\ri}{\Rightarrow}
\newcommand{\ds}{\displaystyle} \renewcommand{\ni}{\noindent}
\newcommand{\pa}{\partial} \newcommand{\Om}{\Omega}
\newcommand{\om}{\omega} \newcommand{\sik}{\sum_{i=1}^k}
\newcommand{\vov}{\Vert\omega\Vert} \newcommand{\Umy}{U_{\mu_i,y^i}}
\newcommand{\lamns}{\lambda_n^{^{\scriptstyle\sigma}}}
\newcommand{\chiomn}{\chi_{_{\Omega_n}}}
\newcommand{\ullim}{\underline{\lim}} \newcommand{\bsy}{\boldsymbol}
\newcommand{\mvb}{\mathversion{bold}} \newcommand{\la}{\lambda}
\newcommand{\La}{\Lambda} \newcommand{\va}{\varepsilon}
\newcommand{\be}{\beta} \newcommand{\al}{\alpha}
\newcommand{\dis}{\displaystyle} \newcommand{\R}{{\mathbb R}}
\newcommand{\N}{{\mathbb N}} \newcommand{\cF}{{\mathcal F}}
\newcommand{\gB}{{\mathfrak B}} \newcommand{\eps}{\epsilon}
\renewcommand\refname{参考文献}
\begin{document}
\title{\huge{\bf{习题1.5.32}}} \author{\small{叶卢
    庆\footnote{叶卢庆(1992---),男,杭州师范大学理学院数学与应用数学专业
      本科在读,E-mail:h5411167@gmail.com}}\\{\small{杭州师范大学理学院,浙
      江~杭州~310036}}}
\maketitle
\begin{exercise}
  证明当 $n=2m$ 为偶数时,
$$
{2m\choose 1}-{2m\choose 3}+{2m\choose 5}-\cdots+(-1)^{m+1}{2m\choose
  2m-1}=2^m\sin (\frac{m\pi}{2}).
$$
\end{exercise}
\begin{proof}[\textbf{证明}]
我们发现,$2^m\sin (\frac{m\pi}{2})$ 是复数 $2^me^{\frac{m\pi i}{2}}$ 的
虚部.而
  \begin{align*}
    2^me^{\frac{m\pi i}{2}}&=(1+i)^{2m}\\&=\sum_{k=1}^m {m\choose k}i^k.
  \end{align*}
然后把实部虚部分离就可以得出题目中的结论.
\end{proof}
\end{document}








我们发现,$2^m\sin (\frac{m\pi}{2})$ 是复数 $2^me^{\frac{m\pi i}{2}}$ 的
虚部.下面我们来研究复数序列 $(2^me^{\frac{m\pi
    i}{2}})_{m=1}^{\infty}$,将序列中的各点标在复平面上,如下图所示.\\
\psset{xunit=1.0cm,yunit=1.0cm}
\begin{pspicture*}(-10.08,-7.5)(14.02,3.24)
\psgrid[subgriddiv=0,gridlabels=0,gridcolor=lightgray](0,0)(-10.08,-7.5)(14.02,3.24)
\psset{xunit=0.5cm,yunit=0.5cm,algebraic=true,dotstyle=o,dotsize=3pt 0,linewidth=0.8pt,arrowsize=3pt 2,arrowinset=0.25}
\psaxes[labelFontSize=\scriptstyle,xAxis=true,yAxis=true,Dx=2,Dy=2,ticksize=-2pt 0,subticks=2]{->}(0,0)(-20.16,-15.01)(28.05,6.49)
\begin{scriptsize}
\psdots[dotstyle=*,linecolor=blue](1,0)
\rput[bl](1.16,0.21){\blue{$a_0$}}
\psdots[dotstyle=*,linecolor=blue](0,2)
\rput[bl](0.13,2.22){\blue{$a_1$}}
\psdots[dotstyle=*,linecolor=blue](-4,0)
\rput[bl](-3.86,0.21){\blue{$a_2$}}
\psdots[dotstyle=*,linecolor=blue](0,-8)
\rput[bl](0.13,-7.81){\blue{$a_3$}}
\psdots[dotstyle=*,linecolor=blue](16,0)
\rput[bl](16.15,0.21){\blue{$a_4$}}
\end{scriptsize}
\end{pspicture*}
我们发现,这些复数的虚部的变化规律是
$$
0\to 2\to 0\to -8\to 0\to 32\to\cdots
$$
