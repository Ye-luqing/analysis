\documentclass[a4paper]{article} 
\usepackage{amsmath,amsfonts,bm}
\usepackage{hyperref}
\usepackage{amsthm} 
\usepackage{geometry}
\usepackage{amssymb}
\usepackage{pstricks-add}
\usepackage{framed,mdframed}
\usepackage{graphicx,color} 
\usepackage{mathrsfs,xcolor} 
\usepackage[all]{xy}
\usepackage{fancybox} 
% \usepackage{CJKutf8}
\usepackage{xeCJK}
\newtheorem{theorem}{定理}
\newtheorem{lemma}{引理}
\newtheorem{corollary}{推论}
\newtheorem*{exercise}{习题}
\newtheorem{example}{例}
\geometry{left=2.5cm,right=2.5cm,top=2.5cm,bottom=2.5cm}
\setCJKmainfont[BoldFont=Adobe Heiti Std R]{Adobe Song Std L}
\renewcommand{\today}{\number\year 年 \number\month 月 \number\day 日}
\newcommand{\D}{\displaystyle}
\newcommand{\ds}{\displaystyle} \renewcommand{\ni}{\noindent}
\newcommand{\pa}{\partial} \newcommand{\Om}{\Omega}
\newcommand{\om}{\omega} \newcommand{\sik}{\sum_{i=1}^k}
\newcommand{\vov}{\Vert\omega\Vert} \newcommand{\Umy}{U_{\mu_i,y^i}}
\newcommand{\lamns}{\lambda_n^{^{\scriptstyle\sigma}}}
\newcommand{\chiomn}{\chi_{_{\Omega_n}}}
\newcommand{\ullim}{\underline{\lim}} \newcommand{\bsy}{\boldsymbol}
\newcommand{\mvb}{\mathversion{bold}} \newcommand{\la}{\lambda}
\newcommand{\La}{\Lambda} \newcommand{\va}{\varepsilon}
\newcommand{\be}{\beta} \newcommand{\al}{\alpha}
\newcommand{\dis}{\displaystyle} \newcommand{\R}{{\mathbb R}}
\newcommand{\N}{{\mathbb N}} \newcommand{\cF}{{\mathcal F}}
\newcommand{\gB}{{\mathfrak B}} \newcommand{\eps}{\epsilon}
\renewcommand\refname{参考文献}
\begin{document}
\title{\huge{\bf{习题1.5.11}}} \author{\small{叶卢
    庆\footnote{叶卢庆(1992---),男,杭州师范大学理学院数学与应用数学专业
      本科在读,E-mail:h5411167@gmail.com}}\\{\small{杭州师范大学理学院,浙
      江~杭州~310036}}}
\maketitle
\begin{exercise}
用几何方法解释,何以适合
$$
\arg \left(\frac{z-a}{z-b}\right)=const.
$$
的 $z$ 的轨迹是过定点 $a$ 和 $b$ 的一段圆弧.
\end{exercise}
\begin{proof}[\textbf{证明}]
由图易得 $\angle aAb=2\alpha$.\\
\newrgbcolor{xdxdff}{0.49 0.49 1}
\newrgbcolor{qqwuqq}{0 0.39 0}
\psset{xunit=1.0cm,yunit=1.0cm,algebraic=true,dotstyle=o,dotsize=3pt 0,linewidth=0.8pt,arrowsize=3pt 2,arrowinset=0.25}
\begin{pspicture*}(-6.5,-6.18)(17.34,7.57)
\pscircle(0.16,-0.06){5.13}
\psline(-2.08,4.55)(4.48,2.71)
\psline(-2.08,4.55)(-3.89,3.09)
\psplot[linestyle=dashed,dash=5pt 5pt]{-3.89}{17.34}{(--11.29--1.47*x)/1.81}
\pscustom[linecolor=qqwuqq,fillcolor=qqwuqq,fillstyle=solid,opacity=0.1]{\parametricplot{-0.27456762732897516}{0.6811556330533943}{0.68*cos(t)+-2.08|0.68*sin(t)+4.55}\lineto(-2.08,4.55)\closepath}
\psline(-3.89,3.09)(0.16,-0.06)
\psline(4.48,2.71)(0.16,-0.06)
\begin{scriptsize}
\psdots[dotstyle=*,linecolor=blue](0.16,-0.06)
\rput[bl](0.25,0.07){\blue{$A$}}
\psdots[dotstyle=*,linecolor=xdxdff](-3.89,3.09)
\rput[bl](-3.79,3.21){\xdxdff{$a$}}
\psdots[dotstyle=*,linecolor=xdxdff](4.48,2.71)
\rput[bl](4.56,2.85){\xdxdff{$b$}}
\psdots[dotstyle=*,linecolor=xdxdff](-2.08,4.55)
\rput[bl](-1.99,4.68){\xdxdff{$C$}}
\psdots[dotstyle=*,linecolor=xdxdff](2.26,8.07)
\rput[bl](6.98,7.09){\xdxdff{$D$}}
\rput[bl](-1.83,4.34){\qqwuqq{$\alpha$}}
\end{scriptsize}
\end{pspicture*}
\end{proof}
\end{document}








