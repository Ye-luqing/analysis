\documentclass[a4paper]{article} 
\usepackage{amsmath,amsfonts,bm}
\usepackage{hyperref}
\usepackage{amsthm} 
\usepackage{geometry}
\usepackage{amssymb}
\usepackage{yhmath}
\usepackage{pstricks-add}
\usepackage{framed,mdframed}
\usepackage{graphicx,color} 
\usepackage{mathrsfs,xcolor} 
\usepackage[all]{xy}
\usepackage{fancybox} 
\usepackage{xeCJK}
\newtheorem*{theorem}{定理}
\newtheorem*{lemma}{引理}
\newtheorem*{corollary}{推论}
\newtheorem*{exercise}{习题}
\newtheorem*{example}{例}
\geometry{left=2.5cm,right=2.5cm,top=2.5cm,bottom=2.5cm}
\setCJKmainfont[BoldFont=STXihei]{FZFangSong-Z02S}
\renewcommand{\today}{\number\year 年 \number\month 月 \number\day 日}
\newcommand{\D}{\displaystyle}\newcommand{\ri}{\Rightarrow}
\newcommand{\ds}{\displaystyle} \renewcommand{\ni}{\noindent}
\newcommand{\pa}{\partial} \newcommand{\Om}{\Omega}
\newcommand{\om}{\omega} \newcommand{\sik}{\sum_{i=1}^k}
\newcommand{\vov}{\Vert\omega\Vert} \newcommand{\Umy}{U_{\mu_i,y^i}}
\newcommand{\lamns}{\lambda_n^{^{\scriptstyle\sigma}}}
\newcommand{\chiomn}{\chi_{_{\Omega_n}}}
\newcommand{\ullim}{\underline{\lim}} \newcommand{\bsy}{\boldsymbol}
\newcommand{\mvb}{\mathversion{bold}} \newcommand{\la}{\lambda}
\newcommand{\La}{\Lambda} \newcommand{\va}{\varepsilon}
\newcommand{\be}{\beta} \newcommand{\al}{\alpha}
\newcommand{\dis}{\displaystyle} \newcommand{\R}{{\mathbb R}}
\newcommand{\N}{{\mathbb N}} \newcommand{\cF}{{\mathcal F}}
\newcommand{\gB}{{\mathfrak B}} \newcommand{\eps}{\epsilon}
\renewcommand\refname{参考文献}\renewcommand\figurename{图}
\usepackage[]{caption2} 
\renewcommand{\captionlabeldelim}{}
\begin{document}
\title{\huge{\bf{有两个焦点的卡西尼曲线的一个性质}}} \author{\small{叶卢
    庆\footnote{叶卢庆(1992---),男,杭州师范大学理学院数学与应用数学专业
      本科在读,E-mail:h5411167@gmail.com}}\\{\small{杭州师范大学理学院}}}
\maketitle
我们把平面上满足到两个不同定点的距离乘积为定值这个条件的所有点形成的集
合叫做有两个焦点的卡西尼曲线.两个定点叫做卡西尼曲线的焦点.下面我
们来证明,有两个焦点的卡西尼曲线具有如下性质:

\begin{theorem}
\textbf{对于有两个焦点的卡西尼曲线,当确定卡西尼曲线上的任意一点后,
该点的任意小去心邻域都含有该卡西尼曲线上的点.}
\end{theorem}
当 $n=2$ 时,有三种情况.令 $A,B$
为平面上的两个定点,$C$ 到 $A,B$ 的距离乘积为 $|AC||BC|$.分别体现
在图\eqref{fig:1} ,图 \eqref{fig:2} 和 图 \eqref{fig:3} 中.\\

在图 \eqref{fig:1} 中,以 $C$ 为圆心做一个任意的半径比较小的圆
(圆的半径比 $C$ 到直线 $AB$ 的距离小),且经过 $C$ 作 $AB$ 的垂线,垂线与
圆交于 $G,F$.则易得
\begin{equation}
  \label{eq:1}
 |GA||GB|> |CA||CB|>|FA||FB|.
\end{equation}
对于圆周上的任意一点 $K$,当 $K$ 变动时,易得 $|KA||KB|$ 是连续变动的,也
即,$|KA||KB|$ 是关于 $K$ 的位置连续的.因此根据介值原理,必定在圆弧
$\wideparen{GF}$(在左边)或者圆弧 $\wideparen{FG}$(在右边) 上分别存在点
$K'$ ,$K''$,使得 $|K'A||K'B|=|CA||CB|$,$|K''A||K''B|=|CA||CB|$.\\


\begin{figure}[h]
  \newrgbcolor{qqwuqq}{0 0.39 0}
\psset{xunit=1.0cm,yunit=1.0cm,algebraic=true,dotstyle=o,dotsize=3pt 0,linewidth=0.8pt,arrowsize=3pt 2,arrowinset=0.25}
\begin{pspicture*}(-2.78,-5.44)(20.54,6.74)
\pspolygon[linecolor=qqwuqq,fillcolor=qqwuqq,fillstyle=solid,opacity=0.1](4.31,-1.52)(3.89,-1.54)(3.91,-1.97)(4.33,-1.95)
\psline(4.12,2.6)(2.42,-2.02)
\psline(4.12,2.6)(10.64,-1.7)
\pscircle(4.12,2.6){1.21}
\psline(2.42,-2.02)(10.64,-1.7)
\psline(4.12,2.6)(4.33,-1.95)
\psline(4.18,1.39)(2.42,-2.02)
\psline(4.18,1.39)(10.64,-1.7)
\psline(4.06,3.81)(2.42,-2.02)
\psline(4.06,3.81)(10.64,-1.7)
\psline(4.12,2.6)(4.06,3.81)
\begin{scriptsize}
\psdots[dotstyle=*,linecolor=blue](2.42,-2.02)
\rput[bl](1.96,-1.94){\blue{$A$}}
\psdots[dotstyle=*,linecolor=blue](10.64,-1.7)
\rput[bl](10.72,-1.58){\blue{$B$}}
\psdots[dotstyle=*,linecolor=blue](4.12,2.6)
\rput[bl](4.2,2.72){\blue{$C$}}
\psdots[dotstyle=*,linecolor=blue](4.18,1.39)
\rput[bl](4.26,1.5){\blue{$F$}}
\psdots[dotstyle=*,linecolor=blue](4.06,3.81)
\rput[bl](4.14,3.94){\blue{$G$}}
\end{scriptsize}
\end{pspicture*}
  \caption{}
  \label{fig:1}
\end{figure}

在图 \eqref{fig:2} 中,点 $C$ 位于 直线 $AB$ 上且位于线段 $AB$ 之外时,
易得
\begin{equation}
  \label{eq:2}
  |AE||BE|<|AC||BC|<|AF||BF|.
\end{equation}
因此在圆弧 $\wideparen{EF}$ 和 $\wideparen{FE}$ 上分别存在 $K',K''$,使
得 $|AK'||BK'|=|AC||BC|=|AK''||BK''|$.\\

\begin{figure}[h]
\newrgbcolor{xdxdff}{0.49 0.49 1}
\psset{xunit=1.0cm,yunit=1.0cm,algebraic=true,dotstyle=o,dotsize=3pt 0,linewidth=0.8pt,arrowsize=3pt 2,arrowinset=0.25}
\begin{pspicture*}(-1.01,-6.92)(17.65,1.4)
\pscircle(11.27,-2.73){0.59}
\psline(3.3,-2.76)(11.86,-2.73)
\begin{scriptsize}
\psdots[dotstyle=*,linecolor=blue](3.3,-2.76)
\rput[bl](3.35,-2.68){\blue{$A$}}
\psdots[dotstyle=*,linecolor=blue](9.07,-2.74)
\rput[bl](9.13,-2.66){\blue{$B$}}
\psdots[dotstyle=*,linecolor=xdxdff](11.27,-2.73)
\rput[bl](11.33,-2.64){\xdxdff{$C$}}
\psdots[dotstyle=*,linecolor=darkgray](10.68,-2.73)
\rput[bl](10.74,-2.66){\darkgray{$E$}}
\psdots[dotstyle=*,linecolor=darkgray](11.86,-2.73)
\rput[bl](11.91,-2.64){\darkgray{$F$}}
\end{scriptsize}
\end{pspicture*}
  \caption{}\label{fig:2}
\end{figure}

当点 $C$ 位于直线 $AB$ 上且位于线段 $AB$ 之间时,如图 \eqref{fig:3},则
可得
$$
|AD||BD|=(|AC|-|DC|)(|BC|+|DC|)=|AC||BC|+|DC|(|AC|-|BC|)-|DC|^2.
$$
令 $|DC|$ 足够小,则当 $|AC|\geq |BC|$ 时可得 $|AD||BD|<|AC||BC|$.当
$|AC|< |BC|$ 时可得 $|AD||BD|>|AC||BC|$,根据对称性,此时必有
$|AE||BE|<|AC||BC|$.而且我们必有  $|AI||BI|>|AC||BC|$.可见,在弧
$\wideparen{DI}$ 或者 $\wideparen{IE}$ 上必定存在 $K$,使得
$|AK||BK|=|AC||BC|$.\\

这样我们就分别分析了三种仅有的情况.由于以 $C$ 为圆心所做的圆的半径是任
意小的,因此我们就证明了定理.
\begin{figure}[h]
\newrgbcolor{xdxdff}{0.49 0.49 1}
\newrgbcolor{qqwuqq}{0 0.39 0}
\psset{xunit=1.0cm,yunit=1.0cm,algebraic=true,dotstyle=o,dotsize=3pt 0,linewidth=0.8pt,arrowsize=3pt 2,arrowinset=0.25}
\begin{pspicture*}(-0.61,-5.53)(17.13,2.38)
\pspolygon[linecolor=qqwuqq,fillcolor=qqwuqq,fillstyle=solid,opacity=0.1](8.18,-0.47)(7.9,-0.48)(7.9,-0.75)(8.18,-0.75)
\psline(2.32,-0.84)(11.1,-0.7)
\pscircle(8.18,-0.75){1.02}
\psline(8.16,0.27)(8.2,-1.77)
\begin{scriptsize}
\psdots[dotstyle=*,linecolor=blue](2.32,-0.84)
\rput[bl](2.38,-0.76){\blue{$A$}}
\psdots[dotstyle=*,linecolor=blue](11.1,-0.7)
\rput[bl](11.16,-0.62){\blue{$B$}}
\psdots[dotstyle=*,linecolor=xdxdff](8.18,-0.75)
\rput[bl](8.18,-0.65){\xdxdff{$C$}}
\psdots[dotstyle=*,linecolor=xdxdff](9.2,-0.73)
\rput[bl](9.25,-0.65){\xdxdff{$E$}}
\psdots[dotstyle=*,linecolor=darkgray](7.16,-0.76)
\rput[bl](6.87,-0.67){\darkgray{$D$}}
\psdots[dotstyle=*,linecolor=darkgray](8.16,0.27)
\rput[bl](8.22,0.35){\darkgray{$H$}}
\psdots[dotstyle=*,linecolor=darkgray](8.2,-1.77)
\rput[bl](8.25,-1.69){\darkgray{$I$}}
\end{scriptsize}
\end{pspicture*}
  \caption{}
  \label{fig:3}
\end{figure}
\end{document}








