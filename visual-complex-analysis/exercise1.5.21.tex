\documentclass[a4paper]{article} 
\usepackage{amsmath,amsfonts,bm}
\usepackage{hyperref}
\usepackage{amsthm} 
\usepackage{geometry}
\usepackage{amssymb}
\usepackage{pstricks-add}
\usepackage{framed,mdframed}
\usepackage{graphicx,color} 
\usepackage{mathrsfs,xcolor} 
\usepackage[all]{xy}
\usepackage{fancybox} 
% \usepackage{CJKutf8}
\usepackage{xeCJK}
\newtheorem*{theorem}{定理}
\newtheorem{lemma}{引理}
\newtheorem{corollary}{推论}
\newtheorem*{exercise}{习题}
\newtheorem{example}{例}
\geometry{left=2.5cm,right=2.5cm,top=2.5cm,bottom=2.5cm}
\setCJKmainfont[BoldFont=Adobe Heiti Std R]{Adobe Song Std L}
\renewcommand{\today}{\number\year 年 \number\month 月 \number\day 日}
\newcommand{\D}{\displaystyle}
\newcommand{\ds}{\displaystyle} \renewcommand{\ni}{\noindent}
\newcommand{\pa}{\partial} \newcommand{\Om}{\Omega}
\newcommand{\om}{\omega} \newcommand{\sik}{\sum_{i=1}^k}
\newcommand{\vov}{\Vert\omega\Vert} \newcommand{\Umy}{U_{\mu_i,y^i}}
\newcommand{\lamns}{\lambda_n^{^{\scriptstyle\sigma}}}
\newcommand{\chiomn}{\chi_{_{\Omega_n}}}
\newcommand{\ullim}{\underline{\lim}} \newcommand{\bsy}{\boldsymbol}
\newcommand{\mvb}{\mathversion{bold}} \newcommand{\la}{\lambda}
\newcommand{\La}{\Lambda} \newcommand{\va}{\varepsilon}
\newcommand{\be}{\beta} \newcommand{\al}{\alpha}
\newcommand{\dis}{\displaystyle} \newcommand{\R}{{\mathbb R}}
\newcommand{\N}{{\mathbb N}} \newcommand{\cF}{{\mathcal F}}
\newcommand{\gB}{{\mathfrak B}} \newcommand{\eps}{\epsilon}
\renewcommand\refname{参考文献}\newcommand{\degre}{\ensuremath{^\circ}}
\begin{document}
\title{\huge{\bf{利用几何变换证明Van Aubel定理}}} \author{\small{叶卢
    庆\footnote{叶卢庆(1992---),男,杭州师范大学理学院数学与应用数学专业
      本科在读,E-mail:h5411167@gmail.com}}\\{\small{杭州师范大学理学院,浙
      江~杭州~310036}}}
\maketitle
所谓 Van Aubel定理,叙述如下:
\begin{theorem}
如图,任意一个四边形 $ABCD$ 的四条边上向外各作一个正方形,正方形的中心分
别为 $M,N,O,P$.则 $NP$ 垂直 $MO$ 且 $|NP|=|MO|$.\\
\psset{xunit=1.0cm,yunit=1.0cm,algebraic=true,dotstyle=o,dotsize=3pt 0,linewidth=0.8pt,arrowsize=3pt 2,arrowinset=0.25}
\begin{pspicture*}(0,-8.34)(25.11,5.95)
\pspolygon[fillcolor=black,fillstyle=solid,opacity=0.1](5.84,1.26)(9.08,1.82)(10.74,-1.48)(4.6,-0.61)
\pspolygon[fillcolor=black,fillstyle=solid,opacity=0.1](4.6,-0.61)(5.84,1.26)(3.97,2.5)(2.73,0.63)
\pspolygon[fillcolor=black,fillstyle=solid,opacity=0.1](5.84,1.26)(9.08,1.82)(8.51,5.06)(5.28,4.5)
\pspolygon[fillcolor=black,fillstyle=solid,opacity=0.1](9.08,1.82)(10.74,-1.48)(14.04,0.19)(12.38,3.48)
\pspolygon[fillcolor=black,fillstyle=solid,opacity=0.1](10.74,-1.48)(4.6,-0.61)(3.73,-6.75)(9.87,-7.62)
\psline(5.84,1.26)(9.08,1.82)
\psline(9.08,1.82)(10.74,-1.48)
\psline(10.74,-1.48)(4.6,-0.61)
\psline(4.6,-0.61)(5.84,1.26)
\psline(4.6,-0.61)(5.84,1.26)
\psline(5.84,1.26)(3.97,2.5)
\psline(3.97,2.5)(2.73,0.63)
\psline(2.73,0.63)(4.6,-0.61)
\psline(5.84,1.26)(9.08,1.82)
\psline(9.08,1.82)(8.51,5.06)
\psline(8.51,5.06)(5.28,4.5)
\psline(5.28,4.5)(5.84,1.26)
\psline(9.08,1.82)(10.74,-1.48)
\psline(10.74,-1.48)(14.04,0.19)
\psline(14.04,0.19)(12.38,3.48)
\psline(12.38,3.48)(9.08,1.82)
\psline(10.74,-1.48)(4.6,-0.61)
\psline(4.6,-0.61)(3.73,-6.75)
\psline(3.73,-6.75)(9.87,-7.62)
\psline(9.87,-7.62)(10.74,-1.48)
\psline(7.18,3.16)(7.24,-4.11)
\psline(4.28,0.95)(11.56,1)
\psline(5.84,1.26)(10.74,-1.48)
\psline(7.18,3.16)(8.29,-0.11)
\psline(8.29,-0.11)(7.24,-4.11)
\psline(4.28,0.95)(8.29,-0.11)
\psline(11.56,1)(8.29,-0.11)
\begin{scriptsize}
\psdots[dotstyle=*,linecolor=blue](5.84,1.26)
\rput[bl](5.94,1.4){\blue{$A$}}
\psdots[dotstyle=*,linecolor=blue](9.08,1.82)
\rput[bl](9.18,1.96){\blue{$B$}}
\psdots[dotstyle=*,linecolor=blue](10.74,-1.48)
\rput[bl](10.82,-1.32){\blue{$C$}}
\psdots[dotstyle=*,linecolor=blue](4.6,-0.61)
\rput[bl](4.7,-0.48){\blue{$D$}}
\psdots[dotstyle=*,linecolor=darkgray](3.97,2.5)
\rput[bl](4.07,2.64){\darkgray{$E$}}
\psdots[dotstyle=*,linecolor=darkgray](2.73,0.63)
\rput[bl](2.83,0.76){\darkgray{$F$}}
\psdots[dotstyle=*,linecolor=darkgray](8.51,5.06)
\rput[bl](8.62,5.2){\darkgray{$G$}}
\psdots[dotstyle=*,linecolor=darkgray](5.28,4.5)
\rput[bl](5.36,4.63){\darkgray{$H$}}
\psdots[dotstyle=*,linecolor=darkgray](14.04,0.19)
\rput[bl](14.13,0.32){\darkgray{$I$}}
\psdots[dotstyle=*,linecolor=darkgray](12.38,3.48)
\rput[bl](12.47,3.63){\darkgray{$J$}}
\psdots[dotstyle=*,linecolor=darkgray](3.73,-6.75)
\rput[bl](3.83,-6.6){\darkgray{$K$}}
\psdots[dotstyle=*,linecolor=darkgray](9.87,-7.62)
\rput[bl](9.98,-7.47){\darkgray{$L$}}
\psdots[dotstyle=*,linecolor=darkgray](4.28,0.95)
\rput[bl](4.37,1.09){\darkgray{$M$}}
\psdots[dotstyle=*,linecolor=darkgray](7.18,3.16)
\rput[bl](7.28,3.3){\darkgray{$N$}}
\psdots[dotstyle=*,linecolor=darkgray](11.56,1)
\rput[bl](11.65,1.14){\darkgray{$O$}}
\psdots[dotstyle=*,linecolor=darkgray](7.24,-4.11)
\rput[bl](7.33,-3.97){\darkgray{$P$}}
\psdots[dotstyle=*,linecolor=darkgray](13.72,1.74)
\rput[bl](13.83,1.89){\darkgray{$U$}}
\psdots[dotstyle=*,linecolor=darkgray](8.29,-0.11)
\rput[bl](8.38,0.04){\darkgray{$F_1$}}
\end{scriptsize}
\end{pspicture*}
\end{theorem}
\begin{proof}[\textbf{证明}]
如下图所示,我们发现 $A$ 绕着 $N$ 沿着逆时针方向旋转了 $\frac{\pi}{2}$,到达
$B$,$B$ 再沿着 $O$ 逆时针旋转 $\frac{\pi}{2}$,到达 $C$,这等效于 $A$ 绕
着点 $F_1$ 逆时针旋转 $\pi$,其中 $F_1$ 是圆 $AD_1C$ 的圆心,也就是线段
$AC$ 的中点.\\
\newrgbcolor{qqwuqq}{0 0.39 0}
\psset{xunit=1.0cm,yunit=1.0cm,algebraic=true,dotstyle=o,dotsize=3pt 0,linewidth=0.8pt,arrowsize=3pt 2,arrowinset=0.25}
\begin{pspicture*}(0,-8.36)(30.07,8.92)
\pspolygon[linecolor=qqwuqq,fillcolor=qqwuqq,fillstyle=solid,opacity=0.1](3.74,1.03)(4.27,0.75)(4.55,1.28)(4.02,1.56)
\pspolygon[linecolor=qqwuqq,fillcolor=qqwuqq,fillstyle=solid,opacity=0.1](10.78,2.2)(11.07,1.68)(11.6,1.98)(11.3,2.5)
\pscircle(4.02,1.56){3.27}
\psline(4.02,1.56)(2.47,-1.32)
\pscircle(11.3,2.5){5.05}
\psline(6.9,0.01)(11.3,2.5)
\psline(4.02,1.56)(6.9,0.01)
\psline(11.3,2.5)(13.79,-1.9)
\psline(6.42,3.79)(6.9,0.01)
\psline(6.42,3.79)(2.47,-1.32)
\psline(6.42,3.79)(13.79,-1.9)
\psline(2.47,-1.32)(13.79,-1.9)
\pscircle(8.13,-1.61){5.66}
\begin{scriptsize}
\psdots[dotstyle=*,linecolor=blue](4.02,1.56)
\rput[bl](4.15,1.74){\blue{$N$}}
\psdots[dotstyle=*,linecolor=blue](11.3,2.5)
\rput[bl](11.41,2.68){\blue{$O$}}
\psdots[dotstyle=*,linecolor=blue](2.47,-1.32)
\rput[bl](2.59,-1.15){\blue{$A$}}
\psdots[dotstyle=*,linecolor=blue](6.9,0.01)
\rput[bl](7.01,0.18){\blue{$B$}}
\rput[bl](4.06,0.95){\qqwuqq{$90\textrm{\degre}$}}
\psdots[dotstyle=*,linecolor=blue](13.79,-1.9)
\rput[bl](13.9,-1.72){\blue{$C$}}
\rput[bl](11.07,1.88){\qqwuqq{$90\textrm{\degre}$}}
\psdots[dotstyle=*,linecolor=darkgray](6.42,3.79)
\rput[bl](6.53,3.95){\darkgray{$D_1$}}
\psdots[dotstyle=*,linecolor=darkgray](8.13,-1.61)
\rput[bl](8.23,-1.44){\darkgray{$F_1$}}
\end{scriptsize}
\end{pspicture*}

因此点 $N$ 绕着本身逆时针旋转 $\frac{\pi}{2}$,再绕着点 $O$ 逆时针旋转
$\frac{\pi}{2}$,也等效于点 $N$ 绕着 $F_1$ 逆时针旋转 $\pi$.如下图所
示.$|ON|=|ON'|$ 且 $ON$ 垂直于 $ON'$.且 $F_1$ 是线段 $NN'$ 的中点.因此
易得 $|F_1N|=|F_1O|$ 且 $F_1N$ 垂直于 $F_1O$.同理, $F_1P$ 垂直于 $F_1M$ 且两者长度相等.因此三角形 $F_1NP$ 是由三
角形 $F_1OM$ 绕着点 $F_1$ 顺时针旋转 $\frac{\pi}{2}$ 得到的,因此
$|NP|=|MO|$ 且 $NP$ 垂直于 $MO$.
\newrgbcolor{qqwuqq}{0 0.39 0}
\psset{xunit=1.0cm,yunit=1.0cm,algebraic=true,dotstyle=o,dotsize=3pt 0,linewidth=0.8pt,arrowsize=3pt 2,arrowinset=0.25}
\begin{pspicture*}(-4.3,-5.88)(23.02,6.3)
\pspolygon[linecolor=qqwuqq,fillcolor=qqwuqq,fillstyle=solid,opacity=0.1](8.37,4.02)(8.49,3.61)(8.9,3.73)(8.78,4.14)
\psline(3.32,2.52)(8.78,4.14)
\psline(8.78,4.14)(10.4,-1.32)
\psline(3.32,2.52)(10.4,-1.32)
\psline(8.78,4.14)(6.86,0.6)
\begin{scriptsize}
\psdots[dotstyle=*,linecolor=blue](3.32,2.52)
\rput[bl](3.4,2.64){\blue{$N$}}
\psdots[dotstyle=*,linecolor=blue](8.78,4.14)
\rput[bl](8.86,4.26){\blue{$O$}}
\psdots[dotstyle=*,linecolor=blue](10.4,-1.32)
\rput[bl](10.48,-1.2){\blue{$N'$}}
\rput[bl](8.54,3.74){\qqwuqq{$90\textrm{\degre}$}}
\psdots[dotstyle=*,linecolor=darkgray](6.86,0.6)
\rput[bl](6.94,0.72){\darkgray{$F_1$}}
\end{scriptsize}
\end{pspicture*}
\end{proof}
\end{document}








