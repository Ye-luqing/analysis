\documentclass[a4paper]{article} 
\usepackage{amsmath,amsfonts,bm}
\usepackage{hyperref}
\usepackage{amsthm} 
\usepackage{geometry}
\usepackage{amssymb}
\usepackage{pstricks-add}
\usepackage{framed,mdframed}
\usepackage{graphicx,color} 
\usepackage{mathrsfs,xcolor} 
\usepackage[all]{xy}
\usepackage{fancybox} 
\usepackage{xeCJK}
\newtheorem*{theorem}{定理}
\newtheorem*{lemma}{引理}
\newtheorem*{corollary}{推论}
\newtheorem*{exercise}{习题}
\newtheorem*{example}{例}
\newtheorem*{remark}{注}
\geometry{left=2.5cm,right=2.5cm,top=2.5cm,bottom=2.5cm}
\setCJKmainfont[BoldFont=Adobe Heiti Std R]{Adobe Song Std L}
\renewcommand{\today}{\number\year 年 \number\month 月 \number\day 日}
\newcommand{\D}{\displaystyle}\newcommand{\ri}{\Rightarrow}
\newcommand{\ds}{\displaystyle} \renewcommand{\ni}{\noindent}
\newcommand{\pa}{\partial} \newcommand{\Om}{\Omega}
\newcommand{\om}{\omega} \newcommand{\sik}{\sum_{i=1}^k}
\newcommand{\vov}{\Vert\omega\Vert} \newcommand{\Umy}{U_{\mu_i,y^i}}
\newcommand{\lamns}{\lambda_n^{^{\scriptstyle\sigma}}}
\newcommand{\chiomn}{\chi_{_{\Omega_n}}}
\newcommand{\ullim}{\underline{\lim}} \newcommand{\bsy}{\boldsymbol}
\newcommand{\mvb}{\mathversion{bold}} \newcommand{\la}{\lambda}
\newcommand{\La}{\Lambda} \newcommand{\va}{\varepsilon}
\newcommand{\be}{\beta} \newcommand{\al}{\alpha}
\newcommand{\dis}{\displaystyle} \newcommand{\R}{{\mathbb R}}
\newcommand{\N}{{\mathbb N}} \newcommand{\cF}{{\mathcal F}}
\newcommand{\gB}{{\mathfrak B}} \newcommand{\eps}{\epsilon}
\renewcommand\refname{参考文献}
\begin{document}
\title{\huge{\bf{习题2.9.6}}} \author{\small{叶卢
    庆\footnote{叶卢庆(1992---),男,杭州师范大学理学院数学与应用数学专业
      本科在读,E-mail:h5411167@gmail.com}}\\{\small{杭州师范大学理学院,浙
      江~杭州~310036}}}
\maketitle
\begin{exercise}[2.9.6.4]
若两曲线在 $p\neq 0$ 处相交成角 $\phi$,则它们在映射 $z\to \omega=z^2$
下的象也在 $\omega=p^2$ 处以同样的角 $\phi$ 相交.  
\end{exercise}
\begin{proof}[\bf{证明}]
我们来看映射 $z\to z^2$ 意味着什么.它把点$z=(z_1,z_2)$
变为点$z'=(z_1^2-z_2^2,2z_1z_2).$
易得
$$
\frac{\pa z}{\pa z_1}=(1,0),\frac{\pa z}{\pa z_2}=(0,1),
$$
$$
\frac{\pa z'}{\pa z_1}=(2z_1,2z_2),\frac{\pa z'}{\pa z_2}=(-2z_2,2z_1).
$$
由于
$$
\begin{pmatrix}
  2z_1\\
2z_2
\end{pmatrix}\begin{pmatrix}
  -2z_2&2z_1
\end{pmatrix}=0=\begin{pmatrix}
  1\\
0\\
\end{pmatrix}\begin{pmatrix}
  0&1
\end{pmatrix},
$$
且
$$
\left |\frac{\pa z'}{\pa z_1}\right|=\left|\frac{\pa z'}{\pa
    z_2}\right|\neq 0,
$$
因此命题得证.
\end{proof}
\begin{remark}
  我们探究一下 $z\to z^3$ 是不是有这样的性质.$z\to z^3$ 把点
  $(z_1,z_2)$ 变为 $(z_1^3-3z_2^2z_{1},3z_1^2z_{2}-z_2^3)$.易得 $z\to
  z^3$ 也是具有这样的性质的.
\end{remark}
\end{document}








