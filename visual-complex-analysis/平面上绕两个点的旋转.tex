\documentclass[a4paper]{article} 
\usepackage{amsmath,amsfonts,bm}
\usepackage{hyperref}
\usepackage{amsthm} 
\usepackage{geometry}
\usepackage{amssymb}
\usepackage{pstricks-add}
\usepackage{framed,mdframed}
\usepackage{graphicx,color} 
\usepackage{mathrsfs,xcolor} 
\usepackage[all]{xy}
\usepackage{fancybox} 
\usepackage{xeCJK}
\newtheorem*{theorem}{定理}
\newtheorem*{lemma}{引理}
\newtheorem*{corollary}{推论}
\newtheorem*{exercise}{习题}
\newtheorem*{example}{例}
\geometry{left=2.5cm,right=2.5cm,top=2.5cm,bottom=2.5cm}
\setCJKmainfont[BoldFont=Adobe Heiti Std R]{Adobe Song Std L}
\renewcommand{\today}{\number\year 年 \number\month 月 \number\day 日}
\newcommand{\D}{\displaystyle}\newcommand{\ri}{\Rightarrow}
\newcommand{\ds}{\displaystyle} \renewcommand{\ni}{\noindent}
\newcommand{\pa}{\partial} \newcommand{\Om}{\Omega}
\newcommand{\om}{\omega} \newcommand{\sik}{\sum_{i=1}^k}
\newcommand{\vov}{\Vert\omega\Vert} \newcommand{\Umy}{U_{\mu_i,y^i}}
\newcommand{\lamns}{\lambda_n^{^{\scriptstyle\sigma}}}
\newcommand{\chiomn}{\chi_{_{\Omega_n}}}
\newcommand{\ullim}{\underline{\lim}} \newcommand{\bsy}{\boldsymbol}
\newcommand{\mvb}{\mathversion{bold}} \newcommand{\la}{\lambda}
\newcommand{\La}{\Lambda} \newcommand{\va}{\varepsilon}
\newcommand{\be}{\beta} \newcommand{\al}{\alpha}
\newcommand{\dis}{\displaystyle} \newcommand{\R}{{\mathbb R}}
\newcommand{\N}{{\mathbb N}} \newcommand{\cF}{{\mathcal F}}
\newcommand{\gB}{{\mathfrak B}} \newcommand{\eps}{\epsilon}
\renewcommand\refname{参考文献}
\begin{document}
\title{\huge{\bf{平面上绕两个点的旋转}}} \author{\small{叶卢
    庆\footnote{叶卢庆(1992---),男,杭州师范大学理学院数学与应用数学专业
      本科在读,E-mail:h5411167@gmail.com}}\\{\small{杭州师范大学理学院,浙
      江~杭州~310036}}}
\maketitle
如图,平面上有 $A,C$ 两点.点 $B$ 先绕着点 $A$ 逆时针旋转 $\angle BAD$
至点 $D$,再绕着点 $C$ 逆时针旋转 $\angle DCE$ 至点 $E$.则易得这两个操
作可以复合成如下:点 $B$ 绕着点 $K$ 逆时针旋转到点 $E$,其中 $K$ 是圆
$BFE$ 的圆心.\\
\newrgbcolor{xdxdff}{0.49 0.49 1}
\psset{xunit=1.0cm,yunit=1.0cm,algebraic=true,dotstyle=o,dotsize=3pt 0,linewidth=0.8pt,arrowsize=3pt 2,arrowinset=0.25}
\begin{pspicture*}(0,-6.28)(24.11,7.11)
\pscircle(4.84,2.16){3.11}
\psline(7.31,0.27)(11.76,2.2)
\pscircle(11.76,2.2){4.85}
\psline(3.67,-0.72)(4.84,2.16)
\psline(4.84,2.16)(7.31,0.27)
\psline(11.76,2.2)(14.09,-2.06)
\psline(7.28,4.08)(7.31,0.27)
\psline(7.28,4.08)(3.67,-0.72)
\psline(7.28,4.08)(14.09,-2.06)
\pscircle(8.94,-0.93){5.27}
\psline(3.67,-0.72)(8.94,-0.93)
\psline(8.94,-0.93)(14.09,-2.06)
\begin{scriptsize}
\psdots[dotstyle=*,linecolor=blue](4.84,2.16)
\rput[bl](4.93,2.3){\blue{$A$}}
\psdots[dotstyle=*,linecolor=blue](3.67,-0.72)
\rput[bl](3.76,-0.59){\blue{$B$}}
\psdots[dotstyle=*,linecolor=blue](11.76,2.2)
\rput[bl](11.86,2.34){\blue{$C$}}
\psdots[dotstyle=*,linecolor=xdxdff](7.31,0.27)
\rput[bl](7.39,0.4){\xdxdff{$D$}}
\psdots[dotstyle=*,linecolor=xdxdff](14.09,-2.06)
\rput[bl](14.19,-1.93){\xdxdff{$E$}}
\psdots[dotstyle=*,linecolor=darkgray](7.28,4.08)
\rput[bl](7.37,4.21){\darkgray{$F$}}
\psdots[dotstyle=*,linecolor=darkgray](8.94,-0.93)
\rput[bl](9.02,-0.81){\darkgray{$K$}}
\end{scriptsize}
\end{pspicture*}
\newrgbcolor{xdxdff}{0.49 0.49 1}
\psset{xunit=1.0cm,yunit=1.0cm,algebraic=true,dotstyle=o,dotsize=3pt 0,linewidth=0.8pt,arrowsize=3pt 2,arrowinset=0.25}
\begin{pspicture*}(0,-7.44)(25.79,7.68)
\pscircle(4.84,2.16){3.11}
\psline(7.31,0.27)(11.76,2.2)
\pscircle(11.76,2.2){4.85}
\psline(3.67,-0.72)(4.84,2.16)
\psline(4.84,2.16)(7.31,0.27)
\psline(11.76,2.2)(13.54,6.72)
\psline(7.28,4.08)(7.31,0.27)
\psline(7.28,4.08)(3.67,-0.72)
\psline(7.28,4.08)(13.54,6.72)
\pscircle(15.01,-5.5){12.31}
\psline(3.67,-0.72)(15.01,-5.5)
\psline(15.01,-5.5)(13.54,6.72)
\begin{scriptsize}
\psdots[dotstyle=*,linecolor=blue](4.84,2.16)
\rput[bl](4.93,2.32){\blue{$A$}}
\psdots[dotstyle=*,linecolor=blue](3.67,-0.72)
\rput[bl](3.77,-0.56){\blue{$B$}}
\psdots[dotstyle=*,linecolor=blue](11.76,2.2)
\rput[bl](11.86,2.34){\blue{$C$}}
\psdots[dotstyle=*,linecolor=xdxdff](7.31,0.27)
\rput[bl](7.42,0.43){\xdxdff{$D$}}
\psdots[dotstyle=*,linecolor=xdxdff](13.54,6.72)
\rput[bl](13.62,6.86){\xdxdff{$E$}}
\psdots[dotstyle=*,linecolor=darkgray](7.28,4.08)
\rput[bl](7.39,4.23){\darkgray{$F$}}
\psdots[dotstyle=*,linecolor=darkgray](15.01,-5.5)
\rput[bl](15.11,-5.36){\darkgray{$K$}}
\end{scriptsize}
\end{pspicture*}
\end{document}








