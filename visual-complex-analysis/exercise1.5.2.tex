\documentclass[a4paper]{article} 
\usepackage{amsmath,amsfonts,bm}
\usepackage{hyperref}
\usepackage{amsthm} 
\usepackage{geometry}
\usepackage{amssymb}
\usepackage{pstricks-add}
\usepackage{framed,mdframed}
\usepackage{graphicx,color} 
\usepackage{mathrsfs,xcolor} 
\usepackage[all]{xy}
\usepackage{fancybox} 
\usepackage{xeCJK}
\newtheorem*{theorem}{定理}
\newtheorem*{lemma}{引理}
\newtheorem*{corollary}{推论}
\newtheorem*{exercise}{习题}
\newtheorem*{example}{例}
\geometry{left=2.5cm,right=2.5cm,top=2.5cm,bottom=2.5cm}
\setCJKmainfont[BoldFont=Adobe Heiti Std R]{Adobe Song Std L}
\renewcommand{\today}{\number\year 年 \number\month 月 \number\day 日}
\newcommand{\D}{\displaystyle}\newcommand{\ri}{\Rightarrow}
\newcommand{\ds}{\displaystyle} \renewcommand{\ni}{\noindent}
\newcommand{\pa}{\partial} \newcommand{\Om}{\Omega}
\newcommand{\om}{\omega} \newcommand{\sik}{\sum_{i=1}^k}
\newcommand{\vov}{\Vert\omega\Vert} \newcommand{\Umy}{U_{\mu_i,y^i}}
\newcommand{\lamns}{\lambda_n^{^{\scriptstyle\sigma}}}
\newcommand{\chiomn}{\chi_{_{\Omega_n}}}
\newcommand{\ullim}{\underline{\lim}} \newcommand{\bsy}{\boldsymbol}
\newcommand{\mvb}{\mathversion{bold}} \newcommand{\la}{\lambda}
\newcommand{\La}{\Lambda} \newcommand{\va}{\varepsilon}
\newcommand{\be}{\beta} \newcommand{\al}{\alpha}
\newcommand{\dis}{\displaystyle} \newcommand{\R}{{\mathbb R}}
\newcommand{\N}{{\mathbb N}} \newcommand{\cF}{{\mathcal F}}
\newcommand{\gB}{{\mathfrak B}} \newcommand{\eps}{\epsilon}
\renewcommand\refname{参考文献}
\begin{document}
\title{\huge{\bf{习题1.5.2}}} \author{\small{叶卢
    庆\footnote{叶卢庆(1992---),男,杭州师范大学理学院数学与应用数学专业
      本科在读,E-mail:h5411167@gmail.com}}\\{\small{杭州师范大学理学院,浙
      江~杭州~310036}}}
\maketitle
\begin{exercise}[1.5.2]
求解三次方程 $x^3=3px+2q$,其中 $p,q\in \mathbf{R}$, 可如下进行:一个希望有用的变换 $x=s+t$ 并导出,如果 $st=p$,且 $s^3+t^3=2q$,则
  此 $x$ 为三次方程之根.
    \begin{align*}
      s^3+t^3+3st(s+t)=3p(s+t)+2q
    \end{align*}
因此令 $st=p$,$s^3+t^3=2q$,根据对称性,不妨令
$$
s^3=q+\sqrt{q^2-p^3},t^3=q-\sqrt{q^2-p^3}.
$$
因此可得
$$
s_{1}=\sqrt[3]{q+\sqrt{q^2-p^3}},s_2=\omega
\sqrt[3]{q+\sqrt{q^2-p^3}},s_3=\omega^2 \sqrt[3]{q+\sqrt{q^2-p^3}},
$$
$$
t_1=\sqrt[3]{q-\sqrt{q^2-p^3}},t_2=\omega\sqrt[3]{q-\sqrt{q^2-p^3}},t_3=\omega^2\sqrt[3]{q-\sqrt{q^2-p^3}}.
$$
由于 $st$ 是实数,因此 $x=s_1+t_1$ 或 $s_2+t_3$,$s_3+t_2$.
\end{exercise}
\end{document}








