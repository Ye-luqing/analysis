\documentclass[a4paper]{article} 
\usepackage{amsmath,amsfonts,bm}
\usepackage{hyperref}
\usepackage{amsthm} 
\usepackage{geometry}
\usepackage{amssymb}
\usepackage{pstricks-add}
\usepackage{framed,mdframed}
\usepackage{graphicx,color} 
\usepackage{mathrsfs,xcolor} 
\usepackage[all]{xy}
\usepackage{fancybox} 
% \usepackage{CJKutf8}
\usepackage{xeCJK}
\newtheorem{theorem}{定理}
\newtheorem{lemma}{引理}
\newtheorem{corollary}{推论}
\newtheorem*{exercise}{习题}
\newtheorem{example}{例}
\geometry{left=2.5cm,right=2.5cm,top=2.5cm,bottom=2.5cm}
\setCJKmainfont[BoldFont=Adobe Heiti Std R]{Adobe Song Std L}
\renewcommand{\today}{\number\year 年 \number\month 月 \number\day 日}
\newcommand{\D}{\displaystyle}
\newcommand{\ds}{\displaystyle} \renewcommand{\ni}{\noindent}
\newcommand{\pa}{\partial} \newcommand{\Om}{\Omega}
\newcommand{\om}{\omega} \newcommand{\sik}{\sum_{i=1}^k}
\newcommand{\vov}{\Vert\omega\Vert} \newcommand{\Umy}{U_{\mu_i,y^i}}
\newcommand{\lamns}{\lambda_n^{^{\scriptstyle\sigma}}}
\newcommand{\chiomn}{\chi_{_{\Omega_n}}}
\newcommand{\ullim}{\underline{\lim}} \newcommand{\bsy}{\boldsymbol}
\newcommand{\mvb}{\mathversion{bold}} \newcommand{\la}{\lambda}
\newcommand{\La}{\Lambda} \newcommand{\va}{\varepsilon}
\newcommand{\be}{\beta} \newcommand{\al}{\alpha}
\newcommand{\dis}{\displaystyle} \newcommand{\R}{{\mathbb R}}
\newcommand{\N}{{\mathbb N}} \newcommand{\cF}{{\mathcal F}}
\newcommand{\gB}{{\mathfrak B}} \newcommand{\eps}{\epsilon}
\renewcommand\refname{参考文献}
\begin{document}
\title{\huge{\bf{习题1.5.12}}} \author{\small{叶卢
    庆\footnote{叶卢庆(1992---),男,杭州师范大学理学院数学与应用数学专业
      本科在读,E-mail:h5411167@gmail.com}}\\{\small{杭州师范大学理学院,浙
      江~杭州~310036}}}
\maketitle
\begin{exercise}
  用图形求出以下两个方程
$$
\hbox{Re}\left(\frac{z-1-i}{z+1+i}\right)=0
$$
与
$$
\hbox{Im}\left(\frac{z-1-i}{z+1+i}\right)=0
$$
的轨迹.
\end{exercise}
\begin{proof}[\textbf{证明}]
第一个方程对应于下图.\\
\psset{xunit=0.5cm,yunit=0.5cm}
\begin{pspicture*}(-10.61,-5.12)(12.63,5.24)
\psgrid[subgriddiv=0,gridlabels=0,gridcolor=lightgray](0,0)(-10.61,-5.12)(12.63,5.24)
\psset{xunit=1.0cm,yunit=1.0cm,algebraic=true,dotstyle=o,dotsize=3pt 0,linewidth=0.8pt,arrowsize=3pt 2,arrowinset=0.25}
\psaxes[labelFontSize=\scriptstyle,xAxis=true,yAxis=true,Dx=0.5,Dy=0.5,ticksize=-2pt 0,subticks=2]{->}(0,0)(-5.31,-2.56)(6.32,2.62)
\pscircle(0,0){1.41}
\parametricplot[linewidth=5.2pt]{0.7853981633974483}{3.9269908169872414}{1*1.41*cos(t)+0*1.41*sin(t)+0|0*1.41*cos(t)+1*1.41*sin(t)+0}
\parametricplot[linewidth=5.2pt]{-2.356194490192345}{0.7853981633974483}{1*1.41*cos(t)+0*1.41*sin(t)+0|0*1.41*cos(t)+1*1.41*sin(t)+0}
\begin{scriptsize}
\psdots[dotstyle=*,linecolor=blue](1,1)
\rput[bl](1.03,1.05){\blue{$A$}}
\psdots[dotstyle=*,linecolor=blue](-1,-1)
\rput[bl](-0.97,-0.95){\blue{$B$}}
\psdots[dotstyle=*,linecolor=darkgray](0,0)
\rput[bl](0.04,0.05){\darkgray{$C$}}
\end{scriptsize}
\end{pspicture*}

第二个方程对应于\\
\psset{xunit=0.5cm,yunit=0.5cm}
\begin{pspicture*}(-10.61,-5.12)(12.63,5.24)
\psgrid[subgriddiv=0,gridlabels=0,gridcolor=lightgray](0,0)(-10.61,-5.12)(12.63,5.24)
\psset{xunit=1.0cm,yunit=1.0cm,algebraic=true,dotstyle=o,dotsize=3pt 0,linewidth=0.8pt,arrowsize=3pt 2,arrowinset=0.25}
\psaxes[labelFontSize=\scriptstyle,xAxis=true,yAxis=true,Dx=0.5,Dy=0.5,ticksize=-2pt 0,subticks=2]{->}(0,0)(-5.31,-2.56)(6.32,2.62)
\psline(-1,-1)(1,1)
\psplot{-5.31}{6.32}{(-0-2*x)/-2}
\begin{scriptsize}
\psdots[dotstyle=*,linecolor=blue](1,1)
\rput[bl](1.03,1.05){\blue{$A$}}
\psdots[dotstyle=*,linecolor=blue](-1,-1)
\rput[bl](-0.97,-0.95){\blue{$B$}}
\psdots[dotstyle=*,linecolor=darkgray](0,0)
\rput[bl](0.04,0.05){\darkgray{$C$}}
\end{scriptsize}
\end{pspicture*}
\end{proof}
\end{document}








