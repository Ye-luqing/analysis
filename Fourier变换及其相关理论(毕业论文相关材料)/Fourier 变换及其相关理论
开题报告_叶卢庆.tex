\documentclass{beamer} 
\usetheme{AnnArbor}
\usepackage[euler-digits]{eulervm}
\usefonttheme{professionalfonts}
\usepackage{xeCJK}
\usepackage{pstricks-add}
\setCJKmainfont[BoldFont=FZBeiWeiKaiShu-S19S]{FZQiTi-S14S}
\newtheorem{lemm}{引理}
\newtheorem{theo}{定理}
\newtheorem{rema}{注}
\newtheorem{coro}{推论}
\newtheorem{defi}{定义}
\newtheorem{exer}{题目}
\newtheorem{exam}{例}
\newtheorem{question}{问题}
\renewcommand{\today}{\number\year 年 \number\month 月 \number\day 日}
\begin{document}
\title{Fourier变换及其相关理论} \subtitle{开题报告}
\author{叶卢庆}\institute[杭州师范大学]{杭州师范大学理学院数学112班\\
  \medskip 
指导老师:谢剑}
\begin{frame}
  \titlepage
\end{frame}
\begin{frame}
  \frametitle{选题背景与意义}
  \begin{itemize}
  \item Fourier分析博大精深.横跨代数,微分方程,分析,数论等多个数学领
    域.在力学,光学等物理领域有深刻应用.在图像处理,信号处理等工程领域也
    有广泛应用.\pause
  \item 我的学年论文是写有关Fourier变换的理论,希望毕业论文能在学年论
    文的基础上进行深化和扩展.
  \end{itemize}
\end{frame}
\begin{frame}
  \frametitle{研究的基本内容和拟解决的问题}
  \begin{itemize}
\item 研究Fourier分析的早期历史.古希腊的Ptolemy为了解释行星运
  动中出现的逆行等不规则现象而创造的“本轮”和“均轮”学说,颇有Fourier分析的味道.后来正式的
  发展里出现了Euler,d'Alembert,Daniel Bernoulli,Lagrange,Fourier,Dirichlet,Abel,……等
  一长串光辉的名字.可以说,Fourier分析深刻地主导和影响了19世纪和20世纪分析学
  的发展.特别要参考Fourier的著作,是他把这门学科发扬光大.\pause
\item 探究线性代数的相关理论,比如酉变换,最小二乘法,特征值和Fourier分析理论的
  紧密联系.\pause
\item Fourier分析和微分方程的联系.以及相应的离散Fourier分析与差分方程
  的联系.\pause
\item 离散Fourier分析与多边形之间的关系.以及利用Fourier分析解等周问题(Hurwitz).
  \end{itemize}
\end{frame}
\begin{frame}
  \frametitle{研究的基本内容和拟解决的问题}
\begin{itemize}
  \item 学习与探究为什么通过离散Fourier变换能导出四次以及四次以下方程的根式解(Lagrange).同样
  的变换对于五次方程为什么会失效.
\end{itemize}
\end{frame}
\begin{frame}
  \frametitle{研究的基本内容和拟解决的问题}
      \begin{exam}
$x^3+ax^2+bx+c=0$的三个根$x_1,x_2,x_3$满足关系式
\begin{equation}\label{eq:1}
\begin{cases}
  x_1+x_2+x_3=-a\\
x_1x_2+x_2x_3+x_3x_1=b\\
x_1x_2x_3=-c
\end{cases},
\end{equation}
$\omega=e^{\frac{2\pi i}{3}}$.三个根经过离散Fourier 变换
$$
\begin{pmatrix}
  r_{1}\\
r_{2}\\
r_{3}\\
\end{pmatrix}=
\begin{pmatrix}
  1&1&1\\
1&\omega&\omega^{2}\\
1&\omega^{2}&\omega
\end{pmatrix}
\begin{pmatrix}
  x_1\\
x_2\\
x_3
\end{pmatrix}
$$
后,代入方程组\eqref{eq:1},便能求出$r_1,r_2,r_3$,再通过离散Fourier逆变换,便能求出
$x_1,x_2,x_3$.这背后到底是什么机制?
  \end{exam}
\end{frame}
\begin{frame}
  \frametitle{研究的基本内容和拟解决的问题}
  \begin{itemize}
  \item 学习与研究 Fourier 分析在弦振动,热传导,$n$个自由度的系统的微振动等物理问
    题中的应用.
  \end{itemize}
\end{frame}
\begin{frame}
  \frametitle{研究方法及措施}
\begin{itemize}
\item  看书,上网查阅相关资料和文献.思考.记录笔记和感悟.最后整理成文.\pause
\item 我想论述的问题并不是孤立的.它们之间存在有机的联系.比
  如,Fourier分析,微分方程,积分方程会和相关的物理问题,以及线性代数理
  论有紧密关联.在介绍Fourier分析的历史的时候,会尽
  量和数学理论结合在一起阐述,使得数学的历史和数学理论相得益彰.因此我的研究方
  法就是注重各个主题之间的联系.估计会遇到的困难是,心有余而时间和能力不足.因此,会量力而行.时间若足
  够,就多写一点.时间若不够,就少写一点.
\end{itemize}
\end{frame}
\begin{frame}
  \frametitle{研究工作的步骤与进度}
  \begin{itemize}
  \item 在上交论文的倒数第二个星期撰写好论文,并交由导师审阅.之前一直准
    备.
  \end{itemize}
\end{frame}
\begin{frame}
  \frametitle{END}
\begin{center}
谢谢观看.
\end{center}
\end{frame}
\end{document}




\begin{frame}
  \frametitle{主要参考文献}
  \begin{itemize}
\item  Elias M. Stein,Rami Shakarchi.Fourier Analysis[M].Princeton
  University Press,2003.
\item Geogi P.Tolstov.Fourier Series[M].New York:Dover Publication.
\item Joseph Fourier.桂质亮译.热的解析理论[M].北京:北京大学出版
  社,2008
\item Richard Courant.朱言钧译.柯氏微积分学[M].上海:中华书局,1952.
\item  陶哲轩. 王昆扬译.陶哲轩实分析[M].北京:人民邮电出版
    社,2008.
\item 福田武雄.穆鸿基译.差分方程[M].上海:上海科学技术出版社,1981.
\item M.Klein.张理京,张锦炎等译.古今数学思想[M].上海:上海科学技
  术出版社,2009.
\item Harold M. Edwards.Galois theory[M].Springer,1984.
\item 吴大猷.古典动力学[M].北京:科学出版社,1983.
\item A.P.French.徐绪笃译.振动与波.北京:人民教育出版社,1981.
  \end{itemize}
\end{frame}